\usepackage{tablefootnote}
\usepackage{amssymb}
\usepackage{amsmath}
\usepackage{amsthm}

\renewcommand{\thefootnote}{\arabic{footnote}}
% Macro for 'List of Symbols', 'List of Notations' etc...
\def\listofsymbols{
    \newpage
\chapter*{Lista de Símbolos\hfill}
\addcontentsline{toc}{chapter}{Lista de Símbolos}
\begin{tabbing}
% YOU NEED TO ADD THE FIRST ONE MANUALLY TO ADJUST THE TABBING AND SPACES
$n$~~~~~~~~~~\=\parbox{5in}{Vector size\dotfill \pageref{symbol:nml}}\\
%ADD THE REST OF SYMBOLS WITH THE HELP OF MACRO

%% se añaden nuevos simbolos con el macro \newsymbol y se hace referecnia
% al simbolo utilizando \addsymbol{symbol:LABEL}

\newsymbol (x_i, y_i): {coordenadas que representan un punto}{symbol:xy_i}

\end{tabbing}

    \clearpage{}
}
\def\newsymbol #1: #2#3{$#1$ \> \parbox{5in}{#2 \dotfill \pageref{#3}}\\}
\def\addsymbol#1{\label{#1}}
% Para las imagenes en grilla
% custom commands
\newcommand{\foreign}[1]{{\it #1}}
\DeclareMathOperator*{\argmax}{arg\,max}
\algsetup{}

\newcounter{eqn}
\renewcommand*{\theeqn}{\alph{eqn}}
\newcommand{\num}{\refstepcounter{eqn}\text{\theeqn}\;}
\newcommand{\initbox}{\setcounter{eqn}{0}}

\newtheorem{definition}{Definición}

\newcommand{\figref}[1]{Figura \ref{#1}}
\newcommand{\tabref}[1]{Tabla \ref{#1}}
\newcommand{\secref}[1]{sección \ref{#1}}
\renewcommand{\algref}[1]{Algoritmo \ref{#1}}


\makeatletter
\newcommand{\putindeepbox}[2][0.7\baselineskip]{{%
    \setbox0=\hbox{#2}%
    \setbox0=\vbox{\noindent\hsize=\wd0\unhbox0}
    \@tempdima=\dp0
    \advance\@tempdima by \ht0
    \advance\@tempdima by -#1\relax
    \dp0=\@tempdima
    \ht0=#1\relax
    \box0
}}
\makeatother

%Traducción al español del paquete algorithmic%
\floatname{algorithm}{Algoritmo}
\renewcommand{\listalgorithmname}{Lista de algoritmos}
\renewcommand{\algorithmicrequire}{\textbf{Entrada:}}
\renewcommand{\algorithmicensure}{\textbf{Salida:}}
\renewcommand{\algorithmicend}{\textbf{fin}}
\renewcommand{\algorithmicif}{\textbf{si}}
\renewcommand{\algorithmicthen}{\textbf{entonces}}
\renewcommand{\algorithmicelse}{\textbf{si no}}
\renewcommand{\algorithmicelsif}{\algorithmicelse,\ \algorithmicif}
\renewcommand{\algorithmicendif}{\algorithmicend\ \algorithmicif}
\renewcommand{\algorithmicfor}{\textbf{para}}
\renewcommand{\algorithmicforall}{\textbf{para todo}}
\renewcommand{\algorithmicdo}{\textbf{hacer}}
\renewcommand{\algorithmicendfor}{\algorithmicend\ \algorithmicfor}
\renewcommand{\algorithmicwhile}{\textbf{mientras}}
\renewcommand{\algorithmicendwhile}{\algorithmicend\ \algorithmicwhile}
\renewcommand{\algorithmicloop}{\textbf{repetir}}
\renewcommand{\algorithmicendloop}{\algorithmicend\ \algorithmicloop}
\renewcommand{\algorithmicrepeat}{\textbf{repetir}}
\renewcommand{\algorithmicuntil}{\textbf{hasta que}}
\renewcommand{\algorithmicprint}{\textbf{imprimir}}
\renewcommand{\algorithmicreturn}{\textbf{retorna}}
\renewcommand{\algorithmictrue}{\textbf{cierto }}
\renewcommand{\algorithmicfalse}{\textbf{falso }}
\renewcommand{\algorithmiccomment}{\textbf{comentario : }}
\renewcommand{\algorithmicprocedure}{\textbf{Procedimiento }} 
\renewcommand{\algorithmicrequire}{\textbf{Entrada:}}

%Colores customizados
\definecolor{muybajo}{HTML}{FFF8A6}%
\definecolor{bajo}{HTML}{FAD151}%
\definecolor{normal}{HTML}{F2A82E}%
\definecolor{alto}{HTML}{A95124}%
\definecolor{muyalto}{HTML}{630C0E}%

\algsetup{indent=4em,linenosize=\small, linenodelimiter=.}

% Secciones hasta nivel 3
\setcounter{secnumdepth}{3}
