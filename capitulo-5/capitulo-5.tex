
\chapter{HARDroid: Reconocedor de Actividades Humanas}

\label{chap5:hardroid}

\section{Introducción}

\label{sec51:intro}En este capítulo introducimos un sistema que clasifica
la actividades humanas ambulatorias utilizando teléfonos móviles inteligentes
con la plataforma \abbr{Android}\texttrademark: \emph{HARDroid}.
El sistema es una adaptación del diseño utilizado por sistemas existentes
como \emph{Google Play Services} \cite{Google2013l}, específicamente
la funcionalidad para reconocer actividades humanas utilizando los
conceptos y técnicas expuestos en capítulos precedentes. 

El sistema \emph{HARDroid} está implementado en base a dos componentes
principales: una interfaz de programación que expone las facilidades
y un servicio en ejecución que realiza las funciones principales del
sistema. La interfaz para programadores (\abbr{API}) proporciona
una firma de funciones bien definidas junto con la documentación adecuada
para que toda aplicación móvil de terceros utilice como un componente
externo la solución de un sistema \abbr{HAR}. El servicio de trasfondo
en la plataforma \abbr{Android} es una aplicación para teléfonos
móvil independiente que implementa algoritmos de reconocimiento de
actividades humanas.

La propuesta de este trabajo se basa en un diseño desacoplado para
contar con una implementación genérica y extensible de un sistema
\abbr{HAR}. Como en todo sistema de \emph{software}, el diseño adecuado
posibilita la evolución y mantenimiento del mismo sin afectar el funcionamiento
de otras aplicaciones móviles clientes dependientes. 

Las siguientes secciones se organizan de la siguiente manera: primeramente
en la sección \ref{sec52:dise=0000F1o} damos una introducción de
las consideraciones de diseño y metodología utilizadas para la construcción
de los componentes de software. Se empieza con los conceptos generales
de Ingeniería de Software hasta concluir con los detalles técnicos
de la plataforma \abbr{Android\emph{}}. La siguiente sección \ref{sec53:arquitectura}
incluye una vista general del sistema explicando su arquitectura.
La sección \ref{sec54:hardroid} conforma el núcleo principal de este
proyecto donde una implementación \abbr{HAR} en forma de aplicación
móvil desacoplado es presentado. También, en la sección \ref{sec55:activity}
se explica el desenvolvimiento de una herramienta para realizar experimentos
y evaluar los resultados asociados a la solución. Por último, en la
sección \ref{sec56:conclusion} se discuten los resultados preliminar
obtenidos como motivo de los componentes de software construidos.

\section{Diseño General}

\label{sec52:dise=0000F1o}La construcción del sistema \emph{HARDroid
}está enmarcado dentro del ecosistema de aplicaciones móviles. Desde
el punto de vista conceptual del desarrollo de aplicaciones móviles
es necesario enfocarse en dos aspectos principales para el diseño:
el medio y el contexto.

Una aplicación móvil puede presentarse en diversos medios que corresponden
con el enfoque técnico \cite{Fling2009}, es decir: un sitio Web Móvil,
una aplicación Web Móvil, SMS, Juegos, controles utilitarios\footnote{\emph{Widgets}}
y aplicaciones nativas. Las aplicaciones nativas son uno de los medios
más utilizados debido a la rica experiencia de usuario y capacidades
que pueden ser explotadas en los dispositivos, ya que disponen de
gran soporte en la plataforma subyacente. Por ejemplo, en la plataforma
para teléfonos móviles \abbr{Android}\emph{ }se dispone de las capacidades
disponibles del dispositivo como almacenamiento, ubicación, sensores
y además un medio de distribución certificado como ser el \emph{Google
Play Store} \cite{Google2016p}. 

Por otro lado, está el contexto de la aplicación que se refiere a
la experiencia que el usuario final obtiene al utilizar el sistema
móvil. Para esto el sistema móvil procesa la información de contexto
que rodea al usuario dando una interacción distinta a los sistemas
convencionales. A continuación se listan los tipos de contexto comunes
utilizados en las aplicaciones móviles y un breve ejemplo de su funcionalidad
\cite{Fling2009}:
\begin{itemize}
\item \emph{Utilidad}: Calculadora, Conversión de monedas.
\item \emph{Localización}: Mapas, Registro de actividades físicas.
\item \emph{Informativo}: Buscar información relevante.
\item \emph{Productividad}: Comprar productos y pagar servicios
\item \emph{Inmersión}: Juegos 
\end{itemize}
Estos tipos de contexto se pueden combinar para crear mejores experiencias
de uso de la aplicación móvil. El trabajo desarrollado en esta tesis
se adecua al modelo de aplicaciones móviles de contexto. Se busca
proveer un componente de \emph{Utilidad} que puede ser combinado con
diferentes aplicaciones móviles desarrollados por terceros y pueda
ser mantenido de forma colaborativa. 

\subsection{Criterios de diseño}

La implementación de este trabajo, así como la mayoría de los sistemas
de información, se rige bajo el principio de diseño \emph{basado en
componentes} donde se busca separar la complejidad de un sistema en
módulos y que estos interactúen entre sí. Los módulos en el diseño
de sistemas mejoran la flexibilidad y comprensión mientras se reduce
el tiempo de desarrollo de los mismos \cite{Parnas1972}.

Una de las técnicas más comunes de diseño de sistemas\emph{ }es la
separación por capas (\emph{Layering}, en inglés\emph{)} para dividir
un sistema complejo \cite{Fowler2002} en diferentes niveles. 

\subsubsection{Arquitecturas en capas}

Cuando un sistema se construye en términos de capas, los módulos se
organizan en niveles apilados como en un pastel, donde cada capa se
soporta sobre la capa baja subyacente. En este sentido, la capa superior
utiliza varios servicios definidos en capas inferiores, pero la capa
inferior desconoce y no depende de las capas en niveles superiores.
Las capas definen concretamente las responsabilidades y encapsulan
las funcionalidades soportadas en cada nivel.

La popularidad de la división de sistemas en módulos por capas se
volvió más relevante con el auge de los sistemas tipo \emph{Cliente-Servidor}.
Estos sistemas fueron inicialmente concebidos como de dos (2) capas:
el cliente se encarga de la interfaz de usuario y la lógica de la
aplicación, y el servidor usualmente es una bases de datos relacional.
En la \figref{fig5:cliente_servidor} se muestra una representación
de la arquitectura en dos capas. Para sistemas sencillos que desplieguen
información y actualicen datos, este modelo es el más adecuado. Sin
embargo, a medida que se construyen sistemas más complejos, el problema
radica mantener la lógica central del sistema: las reglas de negocio,
validaciones, cálculos, etc. El modelo de dos capas propone ubicar
la lógica central embebida en la interacciones de la Interfaz de Usuario,
la capa cliente, o almacenar los mismos en procedimientos almacenados
en la basa de datos, la capa servidor.

\begin{figure}[h]
\begin{centering}
\includegraphics[width=0.8\columnwidth]{capitulo-5/graphics/cliente_servidor}
\par\end{centering}
\caption[Modelo de dos capas]{\label{fig5:cliente_servidor}Modelo de dos capas}
\end{figure}

Debido a las limitaciones del enfoque Cliente-Servidor, se puede extender
el mismo con la ayuda del paradigma de Orientación a Objetos para
construir sistemas con arquitecturas de tres (3) capas. Estas capas
son comúnmente conocidas como \cite{Fowler2002}: Presentación, Dominio
y Recursos. En la \tabref{tab5:tres_capas} se definen las responsabilidades
correspondientes a cada capa. 

\begin{table}
\begin{centering}
\begin{tabular}[t]{|l|>{\raggedright}p{0.5\columnwidth}|}
\hline 
Capa & Responsabilidades\tabularnewline
\hline 
\hline 
Presentación & Provisión de servicios a sistemas externos, despliegue de información
e interacción con el usuario.\tabularnewline
\hline 
Dominio & \multirow{1}{0.5\columnwidth}{Lógica y procesamiento del sistema.}\tabularnewline
\hline 
Recursos & Comunicación con bases de datos, sistemas externos, integración, transacciones
y otros componentes.\tabularnewline
\hline 
\end{tabular}
\par\end{centering}
\caption[Modelo de tres capas]{\label{tab5:tres_capas}Modelo de tres capas}
\end{table}

Independientemente del tipo de sistema de información a ser construido
dividir el mismo en capas lógicas, dividir el sistema en partes separadas,
permite reducir el acoplamiento entre diferentes módulos, inclusive
si todos los módulos se ejecutan en la misma máquina física. En la
\figref{fig5:tres_capas} se muestra la arquitectura de tres capas
discutida anteriormente.

\begin{figure}[h]
\begin{centering}
\includegraphics[width=0.8\columnwidth]{capitulo-5/graphics/arqui_tres_capas}
\par\end{centering}
\caption[Modelo de tres capas]{\label{fig5:tres_capas}Modelo de tres capas}

\end{figure}


\subsubsection{Patrones de diseño}

Los patrones de diseño son parte del paradigma de orientación a objetos
ideados para solucionar problemas comunes determinados en un contexto
particular \cite{Shalloway2004}. Una definición acertada se cita
a continuación.
\begin{quotation}
<<\emph{Cada patrón describe un problema que ocurre una y otra vez
en nuestro entorno, y también describe la solución principal al problema,
de tal manera que la solución pueda ser utilizada millares de veces,
sin tener que duplicar el trabajo de pensar cómo resolver el problema.>>}
\cite{Alexander1977}
\end{quotation}
Para nuestro objetivo de implementación, se utilizan los patrones
de diseño para organizar el \emph{Dominio}, la lógica principal del
sistema. La metodología de dividir el\emph{ Dominio} por medio de
los patrones \cite{Fowler2002}:
\begin{enumerate}
\item \emph{Service Layer}: capa de servicios
\item \emph{Domain Model}: modelo del dominio
\end{enumerate}
La capa de servicios es el punto de interacción entre la \emph{Presentación}
y el\emph{ Dominio}, por lo que actúa como proveedor de una interfaz
clara (comúnmente una \abbr{API} de programación). Este patrón define
los límites del sistema con una capa de servicios que establece el
conjunto de operaciones disponibles y coordina el proceso de cada
operación. La capa de servicios puede ser tan gruesa o tan fina como
se requiera, puede contener objetos de servicio con reglas de negocio,
manejo de transacciones y seguridad.

Por otro lado, el modelo del dominio da soporte a la capa de servicios.
Este patrón define objetos de dominio que tienen incorporados datos
y comportamiento, estos representan de manera significativa el dominio
del problema a resolver. Debido a que la lógica de negocio de un sistema
puede ser bastante compleja, estos objetos están diseñados para manejar
las diversas combinaciones de reglas y lógica del sistema.

\subsubsection{Guías Generales}

En este trabajo fueron utilizadas las guías generales para construcción
de sistemas orientados a objetos. Algunos de los principios más relevantes
que han sido considerados durante el diseño de nuestra solución son
\cite{Albin2003}:
\begin{itemize}
\item Modularidad
\item Orientación a Objetos
\item Reusabilidad
\item Ocultamiento de Información
\item Abstracción
\end{itemize}
Siguiendo estos principios se ha logrado una implementación exitosa
de los componentes y aplicaciones móviles clientes de verificación.
Además, se ha enfocado el desarrollo con miras a la colaboración basada
en la comunidad de código abierto por medio de una Licencia Pública
Apache, Versión 2.0 (\emph{Apache License Version 2.0}, en inglés)
\cite{GimenezYegros2016c}.

\subsection{Metodología de desarrollo}

En toda labor de construcción de sistemas de información es necesario
definir una metodología de diseño para encarar problemas y tecnologías
complejas. La metodología utilizada en este trabajo es el diseño descendente
(\emph{top-down desing}), que permite organizar el diseño del sistema
en niveles de abstracción para reducir la complejidad general del
sistema \cite{Albin2003}. 

En esta metodología se empieza definiendo la funcionalidad esperada
del sistema como lo requiere el cliente, la capa más alta, y sigue
paso a paso hasta refinar el diseño en capas más bajas a medida que
se avanzan con detalles específicos. El diseño empieza poniendo énfasis
en las funcionalidades esperadas del sistema cuando este se encuentre
en operación. Luego, se continua con el diseño de la lógica de negocios
que soportan a estas funcionalidades, y para por ultimo enfocarse
en los recursos necesarios por la capa lógica del sistema.

Como aspecto práctica de la metodología se han identificado las siguientes
tareas específicas de diseño que ayudan a comprender y describir mejor
el desarrollo de este trabajo:
\begin{enumerate}
\item Diseñar la arquitectura de la solución.
\item Diseñar la capa de servicios, o interfaz del sistema enfocada para
programadores (\abbr{API}) con un conjunto de llamadas a exponer.
\item Diseño de los recursos que conforman el dominio al servicio.
\item Implementación del servicio reconocedor.
\item Implementación de un componente de evaluación de la solución.
\end{enumerate}
Parte de la metodología consistió en definir la interfaz del sistema
en base a las mejores prácticas de la industria y de aplicaciones
móviles basadas en contexto basados en el lenguaje de programación
\abbr{Java} y la tecnología basada en \abbr{Android}. Se basó la
solución en implementaciones existentes \cite{Google2016m} ya que
las buenas prácticas y mejores diseños se heredan con ventajas.

\subsection{Tecnología}

Esta sección está dedicada a describir la tecnología principal utilizada
para la implementación del diseño de este capítulo, y la mayor parte
del desarrollo de este trabajo. Uno de los criterios claves para la
selección de teléfonos móviles inteligentes apropiados para este trabajo,
además de los sensores, es el sistema operativo (\abbr{OS}, en inglés)
móvil. 

El sistema operativo móvil se encarga del manejo de recursos del dispositivo
y el control de operación de la aplicaciones móviles. Estos sistemas
tiene características similares a los convencionales utilizados en
las computadoras de escritorio (\abbr{PC}, en inglés) , su función
principal es proveer servicios comunes a las aplicaciones y programas,
y además administrar los recursos de \emph{hardware}. Adicionalmente,
los sistemas operativos para teléfonos móviles están orientados al
uso eficiente de energía debido a las limitaciones de batería y características
de movilidad. 

En este trabajo se propone una herramienta que está dirigida a los
teléfonos móviles inteligentes basados en \abbr{Android}. En los
siguientes apartados se da una vista general de \abbr{Android} como
componente tecnológico de este trabajo.

\subsubsection{Plataforma}

El sistema operativo móvil \abbr{Android} es una plataforma de código
abierto con un entorno de desarrollo de distribución publica que además
incluye una interfaz de programación (\abbr{API}, en inglés) completa
con acceso al \emph{hardware} del teléfono y los sensores internos.
Esto facilidad permite que las herramientas de aprendizaje automático
y sus procedimientos puedan ser construidos de manera más simple. 

\abbr{Android} es una plataforma abierta para dispositivos móviles
encabezada por \emph{Google} en junto con la \emph{Open Handset Alliance
\cite{OHA2008}} con el objetivo de innovar en el entorno móvil, mejorando
la experiencia de usuario a un menor costo \cite{Gargenta2014}. El
principal aporte que ha hecho la plataforma es que por primera vez
una plataforma abierta logra separar el \emph{software} y el \emph{hardware}
subyacente. Esto permite que un gran número diverso de dispositivos
ejecuten las mismas aplicaciones, creando un ecosistema más variado
para desarrolladores y consumidores. 

\subsubsection{Arquitectura}

La plataforma \abbr{Android} está soportada sobre el núcleo (\abbr{Kernel},
en inglés) de \abbr{Linux} como se muestra en la \figref{fig5:android-stack}.
El proyecto \abbr{Linux} es una iniciativa de código abierto utilizado
ampliamente en el ámbito de tecnología desde hace más de una década.
La capa más baja de la arquitectura hereda las principales funcionalidades
esperadas en los sistemas operativos modernos: portabilidad, seguridad
y las capacidades funcionales.

\begin{figure}[H]
\begin{centering}
\includegraphics[width=0.8\columnwidth]{capitulo-5/graphics/android_stack}
\par\end{centering}
\caption[Arquitectura de Android]{\label{fig5:android-stack}Arquitectura de \abbr{Android}}
\end{figure}

Adicionalmente, la plataforma ha mejorado el núcleo introduciendo
un número de extensiones especiales para el entorno móvil. Las extensiones
agregan capacidades funcionales mejoradas para administrar la energía,
ya que los móviles utilizan batería, mecanismos para llamados a procedimientos
remotos rápidos, y aislamiento de aplicaciones para mejor seguridad
\cite{Gargenta2014}. Las extensiones son módulos que fueron agregados
y son listados a continuación con su nomenclatura en inglés \cite{Schreiber2011}: 
\begin{itemize}
\item \emph{Alarm }
\item \emph{Ashmem}
\item \emph{Binder}
\item \emph{Power Management}
\item \emph{Low Memory Killer}
\item \emph{Kernel debugger} y \emph{Kernel} \emph{logger}
\end{itemize}
Parte de la arquitectura de la plataforma son los componentes que
están soportados sobre el núcleo y en la siguiente sección se habla
con más detalles acerca de los mismos.

\subsubsection{Componentes}

Los componentes de la plataforma mostrados en la \figref{fig5:android-stack}
se agrupan en las siguientes partes \cite{Gargenta2014} con su nomenclatura
en inglés:
\begin{itemize}
\item \textbf{Native Layer}: Este componente llamado capa nativa, es un
conjunto base de funcionalidades implementadas en \texttt{C/C++} que
no forman parte del núcleo sino del espacio de usuario del sistema.
Este grupo está compuesto de varias partes como: abstracciones de
\emph{hardware} (\abbr{HAL}), librerías nativas, servicios del sistema
y herramientas básicas utilitarias.
\item \textbf{Dalvik}: Es una máquina virtual (\abbr{DVM}) de propósito
específico diseñada especialmente para \abbr{Android} para correr
aplicaciones programadas en el lenguaje \abbr{Java} \cite{Ehringer2010}.
A diferencia de la máquina virtual de Java estándar (\abbr{JVM}),
el diseño se hizo pensando en restricciones específicas de los entornos
móviles, como consumo de energía y capacidad limitada de recursos
como \abbr{CPU} y Memoria. Además, la \abbr{JVM} posee restricciones
de licencia comercial.
\item \textbf{Application Framework}: Este componente provee un entorno
de programación con varias librerías y servicios para construir aplicaciones
nativas. Este componente es el mejor documentado y mejor cubierto
en toda la plataforma ya que permite a los desarrolladores ser creativos
y construir aplicaciones que puedan ser distribuirlas al mercado.
Son parte de este componente las librerías \abbr{Java} genéricas
y específicas para \abbr{Android} \cite{OHA2008r}, como también
servicios (o gestores de recursos) que facilitan las capacidades útiles
de la plataforma como ubicación, sensores, conectividad, etc. 
\item \textbf{Applications}: Las aplicaciones son el punto de acceso principal
de la plataforma y se soporta sobre los componentes arriba mencionados.
La función principal de la aplicación es proveer una utilidad al usuario
final según lo descrito inicialmente en la sección \ref{sec52:dise=0000F1o}.
Las aplicaciones pueden estar instaladas de fábrica en los teléfonos
móviles o pueden ser adquiridos por los usuarios en los mercados de
distribución de aplicaciones \abbr{Android}.
\end{itemize}

\subsection{Entorno Android}

En esta sección, se exponen las facilidades de \abbr{Android} desde
el punto de vista del entorno de programación que ofrece, con énfasis
en las características que este ofrece para crear aplicaciones contextuales.
Este apartado es una vista general de alto nivel acerca de los modelos
de computación y comunicación disponibles para el desarrollador.

\subsubsection{Componentes de Computación}

Los bloques principales de una aplicación móvil son componentes que
\abbr{Android} provee para construir una aplicación contextual de
utilidad. La idea de este apartado es presentar a grandes rasgos los
componentes disponibles y como estos se combinan entre sí para formar
una aplicación. Cada aplicación está compuesta de cuatro (4) componentes
distintos, donde cada uno tiene un tema en especial \cite{Gargenta2014}
y son descritos a continuación con su nomenclatura en inglés:
\begin{itemize}
\item \textbf{\emph{Activity}}: Este componente representa la interfaz de
usuario de la aplicación, responsable de desplegar la información
y capturar las interacciones del usuario. Este componente no persiste
información, son tareas de computo efímeros de corta duración en primer
plano donde el sistema operativo puede requerir parar su ejecución
y pasarlos a un modo de reposo frecuentemente.
\item \textbf{\emph{Service}}: Este componente ofrece un mecanismo para
ejecutar tareas de computo de larga duración que se realicen en segundo
plano, sin una interfaz de usuario asociada. Un servicio solo puede
ser parado si el sistema se queda sin recursos de memoria.
\item \textbf{\emph{Content Provider}}: Este componente provee una interfaz
para compartir datos entre aplicaciones. Su función es la persistencia
de datos por medio de una \abbr{API} que sigue el principio \abbr{CRUD},
una forma de acceso parecida al de bases de datos \abbr{SQL}. Los
datos pueden ser persistidos en archivos locales o recursos remotos
en red.
\item \textbf{\emph{Broadcast Receiver}}: Este componente provee un mecanismo
para publicación y suscripción a eventos del sistema basado en el
patrón de diseño \emph{Observer} \cite{Shalloway2004}. El receptor
es una tarea de computo latente que se activa con la ocurrencia de
un evento para lo cual el receptor está suscrito. Los eventos tienen
la forma de un concepto descrito en la parte de componentes de comunicación.
\end{itemize}
Estos componentes en su conjunto conforman una aplicación móvil. Estos
son bloques básicos que están conectados de manera suelta y están
contenidos en un entorno de aplicación que corresponde a un proceso
del sistema operativo que comparten los mismos recursos: memoria,
archivos, \abbr{CPU}, etc..

\paragraph{Componente de Comunicación}

El intercambio de datos es una necesidad común entre partes de un
sistema. Este intercambio es posible en \abbr{Android} por medio
de un mecanismo de comunicación entre procesos (\abbr{IPC}), o comunicación
entre componentes si son partes de la misma aplicación \cite{Schreiber2011}. 

La comunicación opera por medio del componente (de nomenclatura inglesa)
\emph{Intent} que es una representación de las operaciones a ser realizadas.
Este consiste en una estructura que contiene un identificador en formato
\abbr{URI}, una acción asociada y datos anexos que son enviados en
un mensajes al sistema de comunicación entre procesos para su procesamiento. 

\begin{figure}
\caption[Componentes de Computo y Comunicación]{\label{fig5:mensajes-ipc}Componentes de Computo y Comunicación}
\end{figure}

En la \figref{fig5:mensajes-ipc} se muestran los mensajes de interacción
comunes entre los componentes del entorno \abbr{Android}. El sistema
de comunicación se basa en el mecanismo de \emph{IPC Binder \cite{Schreiber2011}}
que se discute con más detalle durante la descripción del desenvolvimiento
de este trabajo\emph{.}

\section{Arquitectura del proyecto}

\label{sec53:arquitectura}

\section{HARDroid: Servicio de Reconocimiento de Actividades}

\label{sec54:hardroid}

\subsection{Librería API}

\subsection{Servicio BINDER}

\subsection{Clasificador DEX}

\section{ActivitySurvey: Aplicación de Encuesta}

\label{sec55:activity}

\section{Conclusión}

\label{sec56:conclusion}
