
\chapter{Sistemas de Reconocimiento de Actividades }

\label{chap:sistemas-de-reconocimiento}

Este capítulo describe la arquitectura común de un sistema de reconocimiento
de actividades. Se describe detalladamente los componentes del sistema
además de sus funciones principales de manera general. 

\section{Introducción}

Los sistemas de reconocimiento de actividades humanas son similares
a cualquier aplicación de aprendizaje automático. Según trabajos publicados
con anterioridad\cite{LaraLabrador2013}, los sistemas de reconocimiento
comparten una misma estructura de componentes y poseen las mismas
fases realizadas en dos etapas, entrenamiento y evaluación, así como
está descrito en el Capítulo 2 {[}ira al 2{]}. A pesar de que siguen
la misma guía de diseño puede que algunas implementaciones no incluyan
todos los componentes requeridos, es decir, un sistema puede implementar
simplemente los componentes necesarios para la etapa de evaluación
dejando de lado cualquier tarea en etapa de entrenamiento a realizarse
manualmente. 

\section{Componentes}

Actualmente, dentro del marco teórico del estudio de los sistemas
de reconocimiento, se han identificado unos componentes mínimos requeridos
para realizar las etapas de aprendizaje y predicción de manera automatizada\cite{choudhury2008mobile}.
Un sistema de reconocimiento de actividades posee tres componentes
o módulos principales que son:
\begin{itemize}
\item un \emph{módulo recolector de datos sensoriales} que recolecta toda
la información relevante para las actividades humanas utilizando por
supuesto sensores. Ej. el acelerómetro, giroscopio, brújula, etc.
\item un \emph{módulo de procesamiento y selección de muestras} que procesa
los datos en bruto para adecuarlos muestras con variables significativas
que permitan discriminar las actividades a reconocer
\item un \emph{módulo de clasificación }que utiliza las muestras extraídas
para inferir qué actividad probable está realizando un individuo en
un determinado instante.
\end{itemize}
Las responsabilidades y detalles de cada módulo se describen en los
siguientes apartados.

\subsection{Recolector de datos}

\subsection{Procesamiento y selección de muestras}

\subsection{Clasificación}

\section{Capacidades}

Existen un conjunto de características deseables que deben ser satisfechas
para la construcción efectiva de los sistemas de reconocimiento. Estas
características abordan cuestiones de diseño importantes que conciernen
a la calidad y al funcionamiento del sistema:
\begin{enumerate}
\item Portabilidad, el sistema utiliza sensores adjuntos a los individuos,
por ejemplo el sensor de aceleración, y no debe obstruir las actividades
cotidianas de una persona durante su uso para evitar que esto atente
contra la adopción masiva del sistema. 
\item Conectividad, el sistema debe poder transferir de manera confiable
los datos recolectados y procesados a un componente remoto. 
\item Almacenamiento, el sistema debe persistir datos recolectados y procesados
de manera local para mantener la calidad de los mismos y minimizar
la cantidad transferida a otro componente remoto.
\item Procesamiento, el sistema debe realizar tareas de procesamiento y
transformación de datos para producir información relacionada al reconocimiento
de actividades.
\item Ubiquidad, el sistema debe operar en cualquier condición y contexto
en que la persona se encuentre sin interferir u obligar a la persona
a interactuar con el sistema.
\item Uso de energía, el sistema debe preservar el uso de energía en los
dispositivos móviles que están implementados. La lectura de datos
de los sensores, el procesamiento y la conectividad no deben incurrir
en gastos excesivos de energía para que el sistema pueda operar.
\item Privacidad, el sistema debe de mantener de manera confidencial los
datos recolectados y producidos durante la adopción masiva del sistema,
además de avisar sobre la utilización de datos sensibles que puedan
requerir consentimiento del usuario.
\end{enumerate}

