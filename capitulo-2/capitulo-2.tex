
\chapter{Reconocimiento de Actividades}

%##Capítulo 2. Reconocimiento de Actividades
%* Introducción
%* Estructuras de datos de los Sistemas de Información Geográfica
%\input{capitulo-2/definicion-sig.tex}

\section{Introducción}
El conocimiento del contexto de usuario es una de las nuevas aplicaciones y servicios móviles en el área de la computación ubicua.
En general, el contexto de usuario significa su actividad, ubicación, preferencia, su situación, emoción, etc. Con los teléfonos móviles cada vez más potentes, la mayoría de los puntos mencionados del contexto se determinan con componentes de detección integrados en los teléfonos móviles, como acelerómetro, GPS, micrófono, bluetooth, cámara, etc. En esencia, los teléfonos móviles pueden crear redes de sensores móviles que son capaces de recoger datos de los sensores sobre la vida diaria de un usuario, es decir, quien es, lo que está haciendo el usuario, donde se encuentra, ya quien esta ? 

En este trabajo, investigamos posibilidades y viabilidades de tener un servicio para tener contexto que controla la actividad física diaria de los usuarios con teléfonos móviles.


<Aca hablamos de sensores ambientables vs wearable>

<Aca hablamos del las ventajas de realizar el recononocimeinto con dispositivos moviles>

