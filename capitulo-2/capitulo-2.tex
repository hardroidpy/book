
\chapter{Reconocimiento de Actividades}

%##Capítulo 2. Reconocimiento de Actividades
%* Introducción
%* Estructuras de datos de los Sistemas de Información Geográfica
%\input{capitulo-2/definicion-sig.tex}

\section{Introducción}
El conocimiento del contexto de usuario es una de las nuevas aplicaciones y servicios móviles en el área de la computación ubicua.
En general, el contexto de usuario significa su actividad, ubicación, preferencia, su situación, emoción, etc. Con los teléfonos móviles cada vez más potentes, la mayoría de los puntos mencionados del contexto se determinan con componentes de detección integrados en los teléfonos móviles, como acelerómetro, GPS, micrófono, bluetooth, cámara, etc. En esencia, los teléfonos móviles pueden crear redes de sensores móviles que son capaces de recoger datos de los sensores sobre la vida diaria de un usuario, es decir, quien es, lo que está haciendo el usuario, donde se encuentra, ya quien esta ? 

En este trabajo, investigamos posibilidades y viabilidades de tener un servicio para obtener el contexto de la actividad física diaria de los usuarios con teléfonos móviles.

\section{Actividades Humanas}
El diseño de un sistema HAR depende totalmente de las actividades definidas que van a ser reconocida. Por lo tanto, el cambio del conjunto de actividades que el sistema reconoce convierte al problema en uno completamente diferente al anterior.

Teniendo en cuenta esto, las distintas publicaciones, tenemos estos siete distintos grupos de actividades en la siguiente tabla.

\newcommand{\rr}{\raggedright}
\newcommand{\tn}{\tabularnewline}

\begin{table}[htbp]
	\begin{center}
		\begin{tabular}{|l|p{9cm}|}
			\hline
			\textbf{Grupo} & \textbf{Actividades} \\
			\hline \hline
			Ambulatoria & Caminar, correr, sentarse, pararse, quedarse quieto, acostarse, subir escaleras, descender escaleras, usar escaleras mecánicas, usar elevador.\\ \hline
			Transporte & Andar en bus, bicicleta y conducir \\ \hline
			En el teléfono & Enviar mensajes de texto y hacer llamadas \\ \hline
			Actividades diarias & Comer, beber, trabajar en la PC, mirar TV, leer, cepillarse los dientes, aspirar el piso, y otros. \\ \hline
			Ejercitarse & Alzar pesas, bicicleta estática, remo y otros. \\ \hline
			Militares & Arrastrarse, en cuclillas, abrir la puerta \\ \hline
			Parte superior del cuerpo & Masticar, hablar, mover la cabeza, tragar líquidos, mirar. \\ \hline
		\end{tabular}
		\caption{Grupos de Actividades.}
		\label{tabla:sencilla}
	\end{center}
\end{table}


\section{Sensores ambientales vs Sensores <<Wearables>>}

En la problemática del reconocimiento de la actividad que esta realizando una persona uno de los puntos importantes a considerar es la elección de los sensores a utilizar. Los sensores pueden medir signos vitales (ritmo cardíaco, temperatura del cuerpo, presión arterial), el ambiente (intensidad de luz, temperatura, niveles de sonido), movimiento (aceleración, velocidad), y posición (localización global o en interiores).
Sobre la localización de los sensores respecto a la persona, algunos autores \cite{ReyesOrtiz2015} diferencian entre ambientales, cuando los sensores esta ubicados de manera estatica en el ambiente que rodea a la persona, y <<wearables>> cuando lo sensores se usan o están conectados al cuerpo de la persona.

\subsection{Sensores Ambientales}

Los sensores ambientales, también denominados externos o de entorno, son un conjunto de dispositivos que se encuentran en el medio ambiente que miden propiedades físicas del entorno donde las personas se encuentras e interactúan. Existe una amplia variedad de sensores ambientales, como micrófonos, cámaras de vídeo, sensores de presencia, termómetros y sensores de profundidad(Kinect). Para este tipo de reconocimiento tenemos por ejemplo a \cite{Poppe2007} que realiza un análisis de movimientos humanos utilizando cámaras de vídeo.

\subsection{Sensores <<Wearables>>}

Los sensores <<Wearables>> son usados para obtener señales directamente en una persona, estos pueden estar unidos a varias partes del cuerpo, como ser la cintura, muñeca, pecho, pierna, y cabeza(agregar referencia un paper con dispositivo en brazo/pierna/etc) pero también pueden formar parte de la ropa o embebido en otro accesorio de uso común, como relojes, anteojos, o teléfonos móviles. Como característica tienen una batería que proporciona la energía para poder operar, y algunos cuentan con conexión inalámbrica wifi/bluetooh para la transmisión de los datos obtenidos.

Señales del movimientos y físicas son obtenidas por los sensores, como ser temperatura de la piel, frecuencia cardíaca, conductividad, y posicionamiento global (GPS), posicionamiento en interiores y movimientos del cuerpo. Todos estas mediciones son utilices para tener una constante información del estado de una persona en cualquier momento.

Después hay que mencionar que dentro de los wearables están los que usan sensores acelerómetro en varias partes del cuerpo y los que usan acelerómetro de teléfonos moviles.

Otros autores como \cite{karmul2010} extraen al reconocimiento con cámaras de vídeo como una clasificación diferente a los ambientales. 

\section{Dispositivos Moviles}
<<<Acá hablamos del las ventajas de realizar el reconocimiento con dispositivos móviles ya que tiene sensores>>>


\section{Metodología HAR}
<<<Acá hablamos de tipo de actividades, las fases del reconocimiento de actividades (Colección de datos, procesamiento de datos o extracción de feature, modelado o aprendizaje, y evaluación e inferencia), y por ultimo La definición matemática del problema. >>>

\subsection{Fases del Reconocimiento}

\subsection{Definición del Problema}
Teniendo en cuenta todo esto puntos podemos definir el problema como 

\newtheorem{defi}{Definición}

\begin{defi}(HARP)
	Dado un conjunto $S = \{S_{0},...,S_{k-1}\} $ de $k$ series de tiempo, cada una medida de cada atributo, y todo definido en un intervalo de tiempo $I =  \left [ t_{\alpha}, t_{\omega} \right ]$ el objetivo es encontrar un sub intervalo de tiempo $\left\langle I_{0},...,I_{r-1} \right\rangle $ en $I$, basado en datos de $S$ y conjunto de etiquetas que representan la actividad realizada durante cada intervalo $I_{j}$ (Ej. sentado, caminando, corriendo, etc.). Esto implica que cada intervalo $I_{j}$ son consecutivos, no vacíos, no superpuestos y tal que \displaystyle\bigcup_{r-1}^{j=0}{I_j = I }
\end{defi}

Note que el HARP no es factible una resolución determinista. El numero de combinaciones de valores de atributos y actividades puede ser muy grande a veces infinito; y encontrar los puntos de transicion es complicado

