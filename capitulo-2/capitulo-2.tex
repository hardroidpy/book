
\chapter{Reconocimiento de Actividades}

%##Capítulo 2. Reconocimiento de Actividades
%* Introducción
%* Estructuras de datos de los Sistemas de Información Geográfica
%\input{capitulo-2/definicion-sig.tex}

\section{Introducción}
El conocimiento del contexto de usuario es una de las nuevas aplicaciones y servicios móviles en el área de la computación ubicua.
En general, el contexto de usuario significa su actividad, ubicación, preferencia, su situación, emoción, etc. Con los teléfonos móviles cada vez más potentes, la mayoría de los puntos mencionados del contexto se determinan con componentes de detección integrados en los teléfonos móviles, como acelerómetro, GPS, micrófono, bluetooth, cámara, etc. En esencia, los teléfonos móviles pueden crear redes de sensores móviles que son capaces de recoger datos de los sensores sobre la vida diaria de un usuario, es decir, quien es, lo que está haciendo el usuario, donde se encuentra, ya quien esta ? 

En este trabajo, investigamos posibilidades y viabilidades de tener un servicio para obtener el contexto de la actividad física diaria de los usuarios con teléfonos móviles.

\section{Sensores ambientales vs Sensores <<Wearables>>}

En la problemática del reconocimiento de la actividad que esta realizando una persona uno de los puntos importantes a considerar es la elección de los sensores a utilizar. Los sensores pueden medir signos vitales (ritmo cardíaco, temperatura del cuerpo, presión arterial), el ambiente (intensidad de luz, temperatura, niveles de sonido), movimiento (aceleración, velocidad), y posición (localización global o en interiores).
Sobre la localización de los sensores respecto a la persona, algunos autores \cite{ReyesOrtiz2015} diferencian entre ambientales, cuando los sensores esta ubicados de manera estatica en el ambiente que rodea a la persona, y <<wearables>> cuando lo sensores se usan o están conectados al cuerpo de la persona.

\subsection{Sensores Ambientales}

Los sensores ambientales, también denominados externos o de entorno, son un conjunto de dispositivos que se encuentran en el medio ambiente que miden propiedades físicas del entorno donde las personas se encuentras e interactúan. Existe una amplia variedad de sensores ambientales, como micrófonos, cámaras de vídeo, sensores de presencia, termómetros y sensores de profundidad(Kinect). Para este tipo de reconocimiento tenemos por ejemplo a \cite{Poppe2007} que realiza un análisis de movimientos humanos utilizando cámaras de vídeo.

\subsection{Sensores <<Wearables>>}

Los sensores <<Wearables>> son usados para obtener señales directamente en una persona, estos pueden estar unidos a varias partes del cuerpo, como ser la cintura, muñeca, pecho, pierna, y cabeza(agregar referencia un paper con dispositivo en brazo/pierna/etc) pero también pueden formar parte de la ropa o embebido en otro accesorio de uso común, como relojes, anteojos, o teléfonos móviles. Como característica tienen una batería que proporciona la energía para poder operar, y algunos cuentan con conexión inalámbrica wifi/bluetooh para la transmisión de los datos obtenidos.

Señales del movimientos y físicas son obtenidas por los sensores, como ser temperatura de la piel, frecuencia cardíaca, conductividad, y posicionamiento global (GPS), posicionamiento en interiores y movimientos del cuerpo. Todos estas mediciones son utilices para tener una constante información del estado de una persona en cualquier momento.



Después hay que mencionar que dentro de los wearables están los que usan sensores acelerómetro en varias partes del cuerpo y los que usan acelerómetro de teléfonos moviles.

Otros autores como \cite{karmul2010} extraen al reconocimiento con cámaras de vídeo como una clasificación diferente a los ambientales. 


<<<Acá hablamos del las ventajas de realizar el reconocimiento con dispositivos móviles ya que tiene sensores>>>

<<<Acá hablamos de como las otras publicaciones proponen el metodo de realizar el HAR>>>



