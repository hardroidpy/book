\documentclass{./util/unathesis}

% paquetes recomendados
\usepackage{amsmath,amsthm}
\usepackage{textcomp}
\usepackage[T1]{fontenc}
\usepackage[spanish]{babel}
\usepackage[utf8]{inputenc}
\usepackage{csquotes}
\usepackage{enumerate}
\usepackage{enumitem}
\usepackage{caption}
\usepackage{subcaption} 
\usepackage{algorithm}
\usepackage{algpseudocode}
%\usepackage[style=alphabetic, uniquename=full, sorting=none, backend=biber, natbib=true, backref=true]{biblatex}
\usepackage[authoryear]{natbib}
\usepackage{listings}
% para la lista de simbolos
\usepackage{array} %for vertical thick lines in tables
\usepackage{longtable}
\usepackage{multirow} %multirow tables
\usepackage{nicefrac} %for fractions like 1/4
\usepackage[table]{xcolor} %para colorear las tablas
% para las tablas

%se importan las configuraciones custmizdas realizadas.
\usepackage{tablefootnote}
\usepackage{amssymb}
\renewcommand{\thefootnote}{\arabic{footnote}}
% Macro for 'List of Symbols', 'List of Notations' etc...
\def\listofsymbols{
    
\nomenclature{HAR}{Reconocimiento de Actividades Humanas, por sus siglas en inglés}

    \clearpage{}
}
\def\newsymbol #1: #2#3{$#1$ \> \parbox{5in}{#2 \dotfill \pageref{#3}}\\}
\def\addsymbol#1{\label{#1}}
% Para las imagenes en grilla
% custom commands
\newcommand{\foreign}[1]{{\it #1}}
\DeclareMathOperator*{\argmax}{arg\,max}
%\algsetup{}

\usepackage{amsmath}

\newcounter{eqn}
\renewcommand*{\theeqn}{\alph{eqn}}
\newcommand{\num}{\refstepcounter{eqn}\text{\theeqn}\;}
\newcommand{\initbox}{\setcounter{eqn}{0}}

\newcommand{\figref}[1]{Figura \ref{#1}}
\newcommand{\tabref}[1]{Tabla \ref{#1}}
\newcommand{\secref}[1]{sección \ref{#1}}
\newcommand{\algref}[1]{Algoritmo \ref{#1}}


\makeatletter
\newcommand{\putindeepbox}[2][0.7\baselineskip]{{%
    \setbox0=\hbox{#2}%
    \setbox0=\vbox{\noindent\hsize=\wd0\unhbox0}
    \@tempdima=\dp0
    \advance\@tempdima by \ht0
    \advance\@tempdima by -#1\relax
    \dp0=\@tempdima
    \ht0=#1\relax
    \box0
}}
\makeatother

%Traducción al español del paquete algorithmic%
\floatname{algorithm}{Algoritmo}
\renewcommand{\listalgorithmname}{Lista de algoritmos}
\renewcommand{\algorithmicrequire}{\textbf{Entrada:}}
\renewcommand{\algorithmicensure}{\textbf{Salida:}}
\renewcommand{\algorithmicend}{\textbf{fin}}
\renewcommand{\algorithmicif}{\textbf{si}}
\renewcommand{\algorithmicthen}{\textbf{entonces}}
\renewcommand{\algorithmicelse}{\textbf{si no}}
\renewcommand{\algorithmicelsif}{\algorithmicelse,\ \algorithmicif}
\renewcommand{\algorithmicendif}{\algorithmicend\ \algorithmicif}
\renewcommand{\algorithmicfor}{\textbf{para}}
\renewcommand{\algorithmicforall}{\textbf{para todo}}
\renewcommand{\algorithmicdo}{\textbf{hacer}}
\renewcommand{\algorithmicendfor}{\algorithmicend\ \algorithmicfor}
\renewcommand{\algorithmicwhile}{\textbf{mientras}}
\renewcommand{\algorithmicendwhile}{\algorithmicend\ \algorithmicwhile}
\renewcommand{\algorithmicloop}{\textbf{repetir}}
\renewcommand{\algorithmicendloop}{\algorithmicend\ \algorithmicloop}
\renewcommand{\algorithmicrepeat}{\textbf{repetir}}
\renewcommand{\algorithmicuntil}{\textbf{hasta que}}
\renewcommand{\algorithmicprint}{\textbf{imprimir}}
\renewcommand{\algorithmicreturn}{\textbf{retorna}}
\renewcommand{\algorithmictrue}{\textbf{cierto }}
\renewcommand{\algorithmicfalse}{\textbf{falso }}
\renewcommand{\algorithmiccomment}{\textbf{comentario : }}

%Colores customizados
\definecolor{muybajo}{HTML}{FFF8A6}%
\definecolor{bajo}{HTML}{FAD151}%
\definecolor{normal}{HTML}{F2A82E}%
\definecolor{alto}{HTML}{A95124}%
\definecolor{muyalto}{HTML}{630C0E}%

\algsetup{indent=4em,linenosize=\small, linenodelimiter=.}

% Cabeceras
\pagestyle{fancy}
\renewcommand{\chaptermark}[1]{\markboth{\MakeUppercase{\thechapter. #1 }}{}}
\renewcommand{\sectionmark}[1]{\markright{\thesection\ #1}}
\fancyhf{}
\fancyhead[L]{\ifthenelse{\isodd{\value{page}}}{}{\leftmark}}
\fancyhead[R]{\ifthenelse{\isodd{\value{page}}}{\rightmark}{}}
\fancyfoot[C]{\thepage}
\renewcommand{\headrulewidth}{0.5pt}
\renewcommand{\footrulewidth}{0pt}
\setlength{\headheight}{15pt}
\addtolength{\headheight}{0.5pt}
\fancypagestyle{plain}{
  \fancyhead{}
  \renewcommand{\headrulewidth}{0pt}
}


\makenomenclature

\makeatletter

% datos de la tesis
\degree{Informática}
\advisor{Prof. Ing.}{Joaquin Lima \\
Juan Talavera}
\logosource{./graphics/logo.png}
%\hbadness=10000
%\hfuzz=50pt
%\setcitestyle{round}

\makeatother

\begin{document}

\title{Reconocimiento de actividades humanas con un enfoque colaborativo}
\author{Alberto Gimenez \and Santiago A. Yegros Z.} 

% PDF
\hypersetup{pdftitle=Reconocimiento de actividades humanas con un enfoque colaborativo,
 pdfauthor={Alberto Gimenez, Santiago A. Yegros Z.},
 pdfsubject={Tesis de Grado}}

\maketitle     % esto hace las portadas
\frontmatter

% Agradecimientos

\chapter*{Agradecimientos}

A mis padres por estar siempre presente y brindarme la posibilidad de estudiar una carrera.

A nuestros tutores Ing. Joaquin Lima y Ing. Juan Talavera por acompañarnos y ayudarnos a llevar a cabo este trabajo.

A mi compañero de tesis, Alberto, con quien pudimos lograr sacar este trabajo de grado.

A todos los profesores que de alguna forma nos guiaron durante las materias.

A los voluntarios que nos regalaron un poco de su tiempo y esfuerzo físico para realizar las pruebas.

\begin{flushright}
	Santiago A. Yegros.Z
\end{flushright}

A mi madre Braulia por brindarme la oportunidad de tener una educación y seguir la carrera.

A mi esposa Paola y mi hija Valeria por el ánimo y las fuerzas necesarias para terminar este trabajo.

A los profesores Ing. Joaquin Lima y Ing. Juan Talavera por la guía y mentoría durante el desarrollo de este proyecto.

A los compañeros y amigos que con su entusiasmo hicieron posible culminar las pruebas.

\begin{flushright}
	Alberto Giménez
\end{flushright}


\chapter*{Resumen}

In spanish.

\chapter*{Abstract}

En ingles


% los siguientes comandos producen índices.

% Tabla de contenidos
\tableofcontents	
% Lista de figuras
\listoffigures
% Lista de tablas
\listoftables
% Lista de algoritmos
\listofalgorithms

%
\nomenclature{HAR}{Reconocimiento de Actividades Humanas, por sus siglas en inglés}


\mainmatter  % inician los capitulos de la tesis

\nomenclature{Android}{Android es un sistema operativo móvil desarrollado por Google. Está basado en el kernel de Linux.}

\nomenclature{HAR}{Reconocimiento de Actividades Humanas, por sus siglas en inglés} 

\nomenclature{iOS}{Es un sistema operativo móvil creado y desarrollado por Apple Inc.}

\nomenclature{GPS}{Sistema de Geoposicionamiento Global, por sus siglas en inglés}

\nomenclature{API}{Interfaces de Programación de Aplicaciones, por sus siglas en inglés}

\nomenclature{SaaS}{Software como Servicio}

\nomenclature{DT}{Árboles de decisión, por sus siglas en inglés}

\nomenclature{SVM}{Máquinas de Vectores de Soporte, por sus siglas en inglés.}

\nomenclature{HRM}{Monitor de ritmo cardiaco por sus siglas en inglés.}


% incluye aqui los capitulos (un archivo .tex por capitulo)

\chapter{Introducción}

\label{chap1:introduccion}

Hace apenas una década, que los teléfonos móviles pasaron de ser simples
aparatos de comunicación, a ser dispositivos de información que se
han extendido en todo el mundo como el dispositivo electrónico más
utilizado de manera diaria. Desde la aparición de los teléfonos móviles
modernos (\emph{Smartphones}, o teléfonos inteligentes) con pantalla
táctil, la convergencia en el uso de teléfonos móviles, las redes
de datos e Internet ha ido aumentando gradualmente \cite{Fling2009}.

La capacidad y prestaciones de los teléfonos móviles modernos han
mejorado drásticamente, ya que fusiona los aspectos de teléfonos móviles
y computadoras portátiles, la promesa esperada desde hace más de una
década \cite{Tanenbaum2010}. Muchas de las tareas asociadas comúnmente
a una computadora personal de escritorio pueden ser realizadas con
un teléfono móvil moderno, pero con la ventaja adicional de que el
móvil está más próximo a las actividades diarias del portador.

Desde los teléfonos móviles modernos se tiene acceso a la información
pero en un contexto móvil. El contexto puede darse de dos formas,
la primera forma es el valor que los usuarios mismos generan a partir
de las circunstancias en la que están involucrados. La información
provee un contexto que permite al usuario entender mejor el momento
que está experimentando. Por otro lado, el contexto también es determinado
por la acción que el usuario realizar en su entorno \cite{Fling2009}.
En este último caso, el contexto físico de un usuario involucra a
su ubicación y su actividad humana. 

El reconocimiento de actividades humanas (\abbr{HAR}, \emph{Human
Activity Recognition}) es un tópico de investigación que busca diseñar
algoritmos que provean información acerca del contexto de uno o más
individuos a partir de datos ambiguos capturados en su entorno \cite{Bao2004}.
Reconocer el contexto es un componente primordial de los sistemas
inteligentes y cognitivos, este forma parte de un proceso más complejo,
donde contribuye en las etapas de análisis y captura de datos a través
de sensores externos o adjuntos a los individuos \cite{ReyesOrtiz2015,Chen2012}.
Los avances recientes en las tecnologías de computación móvil y sensores
tales como: la miniaturización, bajo consumo, buenas prestaciones,
buena conectividad y el procesamiento de datos, hizo que proliferara
el uso de los teléfonos móviles modernos con sistemas inteligentes
que monitorizan las acciones del usuario en su vida diaria. 

Diseñar sistemas móviles inteligentes, o aplicaciones móviles de contexto,
que reconozcan las actividades de un individuo, Ej. si está caminando,
está corriendo, está quieto o moviéndose en algún vehículo tiene diversos
motivos en la actualidad\cite{CampuzanoLopez2015,Google2013l}. Tal
es la motivación, que ha aumentado la popularidad de las aplicación
móviles de contexto con funcionalidades en el ámbito del cuidado personal,
la movilidad y la asistencia en la vida diaria que requieren de estas
capacidades. Por citar ejemplos, en la actualidad existen diversas
alternativas en el mercado, como las proveídas por las grandes compañías
\emph{Google}, \emph{Sony} y \emph{Apple}, que son principales fabricantes
de teléfonos móviles e impulsores aplicaciones móviles basadas en
contexto.

Desde el año 2013, \emph{Google} ha proveído como complemento para
la plataforma \emph{\abbr{Android} }\cite{Google2005a} el producto
\emph{Google Play Services}, este dispone de una librería para el
reconocimiento de actividades humanas exclusivamente para desarrolladores
de aplicaciones móviles \cite{Google2013l}. Saliendo al paso, \emph{Sony}
ha lanzado una suite completa (aplicaciones móviles, pulseras y relojes)
para mantenerse en forma acompañados del producto \emph{Lifelog} \cite{Sony2016l}.
Este también es capaz de reconocedor de actividades físicas y registrar
las acciones de sus usuarios simplemente portando el teléfono en cualquier
lugar. Además, \emph{Apple} para su plataforma\emph{ \abbr{iOS}}
\cite{Apple2007i} ha dispuesto recientemente de una librería para
desarrolladores llamada \emph{HealthKit} \cite{Apple2016h}, con un
enfoque similar a los productos anteriormente mencionados. 

Las plataformas para aplicaciones \emph{\abbr{Android}} e \emph{\abbr{iOS}}
proveen un ecosistema abierto para crear aplicaciones personalizadas,
estas conducen a la expansión y el crecimiento de aplicaciones contextuales
\cite{Tanenbaum2010}. Sin embargo, a pesar de promoverse plataformas
abiertas, existe aún una carencia de proyectos colaborativos de código
abierto que contribuyan al desarrollo de sistemas inteligentes de
contexto, a excepción de ciertos precedentes (véase \cite{Kwapisz2011,LaraLabrador2013})
que han hecho esfuerzos iniciales u otras iniciativas como \cite{FUNF2016}
y \cite{SensingKit2016} orientados a la colaboración.

Desde que los teléfonos móviles modernos conectados constantemente
Internet el acceso a la Web\footnote{World Wide Web} se ha popularizando.
Esto impulsó a que la computación móvil haya crecido enormemente,
y también ha crecido el uso de la Web (cfr. \cite{NYTimes2008iph}).
Esto fomenta un ámbito de trabajo colaborativo, distribuido y móvil
donde un grupo masivo de personas pueden aportar a un propósito común,
ya sea como sujetos de estudio en el reconocimiento de actividades
humanas o en el desarrollo de herramientas para aplicaciones móviles
contextuales.

\section{Planteamiento del problema}

\label{sec11:planteamiento}

El reconocimiento de actividades humanas es un tópico de investigación
que data desde los noventa (90), y a la actualidad sigue atrayendo
bastante interés ya que abarca varias áreas novedosas de estudio.
Algunas áreas de estudio que se pueden mencionar son la computación
ubicua, la computación móvil, y la computación contextual; también
la seguridad por vigilancia, las viviendas asistidas y los ambientes
inteligentes, etc.\cite{Chen2012}. 

Reconocer actividades humanas consiste en detectar las acciones o
interacciones con el entorno que llevan acabo las personas utilizando
sensores. Las actividades humanas pueden ser catalogados en distintos
tipos y según el nivel de detalle \cite{Chen2012}. Poniendo énfasis
en las actividades básicas que un individuo puede realizar podemos
destacar las acciones físicas más simples, como lo son las actividades
ambulatorias y de transportación. Además existen otras acciones más
complejas como la postura, los movimientos en ejercicio aeróbico u
otras más seculares, Ej. cepillarse los dientes, mirar televisión,
comer, beber, hablar por teléfono, etc. \cite{LaraLabrador2013}.

El problema del reconocimiento de actividades humanas conlleva un
proceso complejo que se resume en la utilización de técnicas de aprendizaje
automático supervisado, con un modelo de inferencia construido en
dos etapas: entrenamiento y evaluación \cite{LaraLabrador2013,Kwapisz2011}.
Es un proceso se puede resumir en un flujo iterativo que consiste
en \cite{Bao2004}:
\begin{itemize}
\item Recolectar datos etiquetados de un conjunto de individuos de forma
supervisada. 
\item Extraer variables predictoras con atributos relevantes al estudio.
\item Construir un modelo de predicción de actividades humanas.
\item Detectar con cierta precisión el conjunto de actividades previamente
etiquetadas utilizando como entrada un conjunto de datos no etiquetados.
\item Retroalimentar y/o evaluar el modelo para obtener una mejora continua.
\end{itemize}
La investigación en el área de los sistemas \abbr{HAR} utilizando
teléfonos móviles ha progresado de manera sostenida debido a los avances
tecnológicas de sensores de bajo costo, las redes inalámbricas de
alta velocidad y los dispositivos móviles inteligentes \cite{Chen2012}.
A pesar de los avances tecnológicos, aún existen importantes desafíos
por resolver como: seleccionar medidas relevantes, recolectar datos
en forma no invasiva, analizar métodos de inferencia alternativos
y extraer muestras significativas \cite{LaraLabrador2013}. Además
de la restricción al recolectar datos en condiciones realistas y tener
la flexibilidad para soportar nuevos individuos sin necesidad entrenar
el modelo permanentemente se debe diseñar el sistema bajo condiciones
especiales de ahorro de energía y procesamiento \cite{ReyesOrtiz2015}. 

De las cuestiones citadas anteriormente, en este trabajo ponemos énfasis
en cuatro puntos que enmarcan el problema a resolver y cuya justificación
responde a utilizar los teléfonos móviles como plataforma de reconocimiento
de actividades humanas.

\section{Justificación}

\label{sec12:justificaciuxf3n}

Para construir sistemas \abbr{HAR} se requiere utilizar uno o más
sensores que midan datos del entorno de un usuario. Los teléfonos
móviles modernos poseen múltiples sensores de bajo consumo, con el
tamaño y las condiciones adecuadas para la capturar de manera ubicua.
Comúnmente los teléfonos modernos se componen de sensores de: localización
(\abbr{GPS}, \emph{Global Positioning System}), brújula, aceleración,
audio (micrófonos) y vídeo (cámaras), luminosidad, temperatura, barómetro
y otros dependiendo del fabricante y/o accesorios adicionales\cite{Kwapisz2011}.

El sensor de aceleración\footnote{acelerómetro} mide el movimiento
en dos o tres ejes y puede ser combinado con otros para detectar la
orientación del dispositivo. Estas medidas proveen información crucial
para el reconocimiento de actividades humanas con un bajo consumo
de energía. El sensor de localización y podómetro (contador de pasos)
también son ideales para su uso pero poseen más restricciones asociadas.
Adicionalmente, los teléfonos modernos tienen capacidad de almacenamiento
y procesamiento, una buena conectividad por lo que son plataformas
ideales para construir sistemas de reconocimiento autónomos.

La combinación de capacidad, tamaño diminuto y bajo costo hace que
los dispositivos sean portados a cualquier lugar por sus usuarios,
esto permite recolectar los datos en condiciones realistas, una ventaja
ideal que tener que obtener muestras supervisadas en un laboratorio
\cite{Bao2004}. 

La información sobre el estado de arte de esta área de investigación
es bastante extensa y abarca las técnicas de reconocimiento, los métodos
de captura y el procesamiento de datos de sensoriales\cite{LaraLabrador2012,Kwapisz2011}.
Sin embargo, a pesar de estar definida la arquitectura los sistemas
\abbr{HAR}, aún existe un hueco de contar con un componente de software
para teléfonos móviles que pueda ser utilizado sin depender de librerías
privadas Ej. \emph{Google Play Services} API \cite{Google2016l} (\abbr{API},
\emph{Application Programming Interface}), o servicios alojados en
Internet (\abbr{SaaS}, \emph{Software as a Service}), o bien aplicaciones
distribuidas por terceros Ej. \emph{Lifelog}.

El modelo de desarrollo de software libre hace posible tener un enfoque
colaborativo en dos dimensiones, por un lado el software como tal
Ej. una librería de reconocimiento, y por otro el los de datos sensoriales
utilizadas para el entrenamiento del modelo de predicción de actividades.

\section{Alcance y Objetivos}

\label{sec13:alcance-y-objetivos}

Este trabajo se centra en el reconocimiento de actividades humanas
con teléfonos móviles inteligentes utilizando sensores y aportar una
componente de librería de código abierto que sea libremente distribuida.
Durante la recolección de datos y las pruebas se desarrollará adicionalmente
una aplicación móvil que demuestre la efectividad de la librería como
componente de software independiente. También se evaluará el sistema
de reconocimiento de actividades basado en aprendizaje automático.

Se incluye la revisión del estado del arte de la materia con el objetivo
de comprender las metodologías de reconocimiento actualmente empleadas.
Luego, se construirá una librería cliente y un servicio basado en
\emph{\abbr{Android}} que atienda a las llamadas de procedimiento
del sistema de reconocimiento de actividades. 

Dentro del sistema de reconocimiento se incluirá un modelo colaborativo
capaz de actualizarse constantemente, a nuestro criterio, esto mejorará
las predicciones basándose en un conjunto de datos de entrenamiento
proveniente de las personas que colaboren de forma anónima y en condiciones
reales.

Además se implementará un servicio Web de captura de muestras con
predicciones para evaluar y actualizar el modelo de predicción para
futuras versiones. El modelo de aprendizaje se basará en aprendizaje
automático por árboles de decisión (\abbr{DT}, \emph{Decision Trees}).
Finalmente, para evaluar la metodología se creará una aplicación móvil
para recolectar muestras y evaluar el resultado de predicción utilizando
el sistema de reconocimiento propuesto.

\subsection{Objetivo General}

\label{objetivo-general}

Implementar un sistema de reconocimiento de actividades humanas con
teléfonos móviles cuyo principal aporte sea una librería de código
abierto libremente distribuidos.

\subsection{Objetivos Específicos}

\label{objetivos-especuxedficos}
\begin{enumerate}
\item \label{enu:obe1}Definir el estado del arte sobre reconocimiento de
actividades humanas (\abbr{HAR}). 
\item \label{enu:obe2}Comprender las técnicas de recolección de datos en
entornos restringidos para bajo consumo energía. 
\item \label{enu:obe3}Comprender el procesamiento de señales de datos de
aceleración inercial para identificar variables significativas de
entrenamiento. 
\item \label{enu:obe4}Comprender la clasificación por aprendizaje automático
en entornos restringidos para bajo consumo energía. 
\item \label{enu:obe5}Diseñar un sistema de reconocimiento de actividades
que comprenda la recolección de muestras de manera colaborativa y
predicción de actividades en-línea. 
\item \label{enu:obe6}Aportar un componente de software de código abierto
para uso en teléfonos móviles inteligentes. 
\end{enumerate}

\section{Organización del trabajo}

\label{sec14:organizaciuxf3n-del-trabajo}

El trabajo está organizado en la siguiente manera: en el Capítulo
2 se presenta define el marco teórico del área describiendo los conceptos
principales del reconocimiento de actividades humanas, sus características,
dificultades, y las metodologías existentes. El Capítulo 3 definimos
el aprendizaje automático, los tipos de aprendizajes, y las distintas
técnicas o enfoques aplicados, dando énfasis en los arboles de decisión.
Estos capítulos representan el estado del arte de este trabajo.

El Capítulo 4 se presenta en detalle la construcción de un sistema
de reconocimiento para teléfonos móviles. Continuando, en el Capítulo
5 se discute el modelo y la implementación de un reconocedor de actividades
colaborativo, su diseño, arquitectura, y tecnologías utilizadas.

En el Capítulo 6 se detallan los protocolos de los experimentos realizados,
y en el Capítulo 7 los resultados obtenidos durante el estudio. Para
finalizar, en el Capítulo 8 se incluyen las conclusiones y los posibles
trabajos futuros. 
 	%Cap 1: Introducción
%Parte II: Marco referencia (Background): 

\chapter{Marco Teórico}

\label{chap2:marco-teorico}

\section{Introducción}

\label{sec21:introduccion}En este capitulo, se resumen las cuatro
áreas principales de estudio relacionados a los sistemas \abbr{HAR}
con miras a definir el problema del reconocimiento de actividades
humanas desde un punto de vista teórico. La descripción de estas áreas
detallan las aplicaciones que motivan los sistemas \abbr{HAR} y también
los mecanismos que llevan a una implementación factible. 

La primera sección, se introducen las aplicaciones de contexto que
son el principal marco de trabajo del estudio de sistemas \abbr{HAR}.
En síntesis, nuestro enfoque es detectar las actividades físicas para
dar información de contexto en las actividades diarias de un individuo
durante su locomoción. Las siguientes secciones dan todos los elementos
requeridos para implementar los sistemas \abbr{HAR}. Desde el punto
de vista de implementación se discuten los sensores y los teléfonos
móviles inteligentes. La última sección define el marco teórico del
reconocimiento de actividades humanas en base al estado de arte.

\section{Aplicaciones de contexto}

\label{sec22:contexto}En la actualidad construir aplicaciones interactivas
requieren tener en cuenta el aspecto del contexto del usuario principalmente
por la importancia de este asunto debido a que este cambia con mucha
frecuencia, como ocurren en la computación móvil y ubicua. El contexto
de un usuario es un medio adicional para la interacción entre computadora
y humano, además de otros los métodos convencionales de entrada de
datos. Este nuevo medio abre nuevas posibilidades de comunicación
y producción de nuevos servicios en computación. 

Pero, ¿qué es el contexto?, citando otras fuentes podemos definir
el contexto como: \textquotedbl{}\emph{... cualquier información que
pueda ser utilizada para caracterizar la situación de una entidad.
Un entidad es una persona, lugar, o objeto que es considerado relevante
en la interacción entre el usuario y la aplicación, incluyendo al
usuario y la aplicación}\textquotedbl{} \cite{Dey2000}. De manera
específica, el contexto de un usuario es el estado de su información
física, emocional o social, sin dejar de lado cualquier otra situación
en que el usuario esté involucrado y sea relevante para la aplicación.

Debido a la libertad de movilidad presentes en la computación móvil
y ubicua, es primordial construir aplicaciones que conozcan el contexto
de sus usuarios (\emph{context-aware}, o aplicaciones de contexto).
Debido a que el entorno de ejecución cambia con cierto dinamismo,
el rango de situaciones posibles del usuario se amplia y por lo tanto
se requiere que los servicios proveídos por una aplicación se adapten
para mejorar la interacción entre el usuario y el computador. 

Los tipos de contexto prácticos más importantes en las aplicaciones
de contexto con computación móvil son la ubicación, la identidad,
la actividad y el tiempo. Esta caracterización permite a los diseñadores
de aplicaciones escoger el contexto más relevante para su uso.

\section{Sensores}

\label{sec23:sensores} En la problemática del reconocimiento de actividades
realizada por un individuo, uno de los temas importantes a tratar
es la elección de los sensores utilizados. Se han utilizando sensores
variados para extraer información acerca de las actividades de un
individuo. Los sensores pueden medir signos vitales (ritmo cardíaco,
temperatura del cuerpo, presión arterial), señales del ambiente (intensidad
de luz, temperatura, niveles de sonido), el movimiento (aceleración,
velocidad), y la posición (localización global o en interiores). 

Con respecto a la disposición de los sensores en relación a los usuarios,
algunos autores \cite{ReyesOrtiz2015,LaraLabrador2013} diferencian
entre ambientales, los sensores están ubicados de fija en el ambiente
que rodea a al individuo, y \emph{wearables} cuando los sensores están
puestos o conectados al cuerpo de la persona.

\subsection{Sensores Ambientales}

Los sensores ambientales, también denominados externos o de entorno,
son un conjunto de dispositivos que se encuentran en el entorno y
miden propiedades físicas del mismo, a las personas que rodean, y
la interacción entre los mismos. Existe una amplia variedad de sensores
ambientales, como micrófonos, cámaras de vídeo, sensores de presencia,
termómetros y sensores de profundidad (Ej. \abbr{Kinect}). 

Existen varios trabajos que explorar el uso de sensores ambientales
para reconocimiento de actividades. Entre ellos el lo expuesto por
\cite{Poppe2007} que realiza un análisis de movimientos humanos utilizando
cámaras de vídeo. A pesar de ser bastantes efectivos los sistemas
basados en vídeos existen grandes limitaciones para procesamiento
en tiempo real y la privacidad de los usuarios.

\subsection{Sensores Wearables}

Los sensores \emph{wearables} (o de atuendo) son utilizados para obtener
señales directamente de los usuarios, los sensores pueden estar anexos
a varias partes del cuerpo, como la cintura, la muñeca, el pecho,
la pierna, y la cabeza \cite{Bao2004} como también pueden formar
parte de alguna vestimenta. También, pueden estar embebidos en algún
accesorio de uso común, como relojes, anteojos y los teléfonos móviles. 

Como característica adicional tienen autonomía gracias al uso de baterías
que proporcionan energía para poder operar, y además algunos cuentan
con conexiones inalámbricas (Ej. \abbr{WIFI}/\abbr{Bluetooth}) para
la transmisión de las señales. Las señales de movimiento y fisiológicas
obtenidas por los sensores pueden ser temperatura de la piel, frecuencia
cardíaca, conductividad, posicionamiento global (\abbr{GPS}) y movimientos
del cuerpo. Todos estas mediciones son útiles para tener una constante
información del estado de una persona en cualquier momento.

Los sensores anexos directamente a un individuo se clasifican comúnmente
según los siguientes grupos \cite{LaraLabrador2013}:
\begin{description}
\item [{Movimiento}] miden datos inerciales como la aceleración y la orientación
respecto a un marco de referencia relativo al dispositivo que contiene
los sensores. El acelerómetro, giroscopio y la brújula son los sensores
más comunes y utilizados para reconocimiento de actividades con un
bajo consumo de energía y buena precisión de reconocimiento \cite{Bao2004,LaraLabrador2012}.
\item [{Ubicación}] miden datos obtenidos con las redes celulares 3G y
los satélites de navegación \abbr{GPS}. Provee información de contexto
bastante relevante acerca de la posición del individuo, además de
ciertas medidas de movimiento pero con un consumo moderado de energía.
\item [{Fisiología}] miden signos vitales del individuo como el ritmo cardíaco
(\abbr{HRM}, \emph{Hearth Rate Monitor}), la temperatura del cuerpo,
el ritmo de respiración, entre otros.
\item [{Ambiente}] miden datos externos que rodean al individuo como el
nivel de ruido, la humedad y/o la temperatura. Los sensores de luz,
cámara, micrófonos y termómetros miden estas señales. 
\end{description}

\section{Dispositivos móviles}

\label{sec24:dispositivos-moviles} Los dispositivos móviles, tales
como teléfonos móviles, reproductores de música o relojes inteligentes,
han comenzado ya hace un par de años en incorporar diversos sensores.
Debido a su tamaño reducido de estos dispositivos inteligentes, su
enorme capacidad de procesamiento, la posibilidad de recibir y enviar
datos; y su omnipresencia en nuestra sociedad de hoy, lo hacen dispositivos
de preferencia para utilizarlo en la vida diaria de un usuario.

En este trabajo, investigamos posibilidades y viabilidades de tener
un servicio para obtener el contexto de la actividad física diaria
de los usuarios con teléfonos móviles.

%% TODO: Definir mejor los teléfonos móviles, no tanto los sensores

\section{Aprendizaje Automático}

\label{sec25:aprendizaje-automatico}Una de las técnicas de reconocimiento
de actividades humanas consiste en encontrar un modelo para descubrir
información a partir de los datos, es decir utilizar algoritmos de
aprendizaje automático (\abbr{ML}, \emph{Machine Learning}) para
construir modelos que puedan inferir actividades a partir de una gran
cantidad de datos medidos con anterioridad junto con el comportamiento
deseado del usuario\cite{Chen2012}. Esta técnica involucra la creación
de modelos de clasificación probabilistas o estadísticos, seguidos
por los procesos de entrenamiento y aprendizaje.

El aprendizaje automático implica la utilización de datos como conjunto
inicial de entrenamiento para entrenar un algoritmo, uno de muchos
existentes, como redes de Bayes, máquinas de soporte-vector \abbr{SVM},
Árboles de decisión \abbr{DT}, Modelos de Markov ocultos\abbr{HMM},
y otros \cite{Rajaraman2011} (véase siguiente sección). Las ventajas
de utilizar este enfoque de reconocimiento es su capacidad de manejar
información temporal y con cierto grado de incertidumbre, pero su
desventaja es que requiere una cantidad grande de datos de entrenamiento,
por lo que puede sufrir de problemas de inicio lento y escasez de
datos.

\section{Reconocimiento de Actividades Humanas}

El campo de estudio del \abbr{HAR} incluye las metodologías para
comprender el comportamiento humano a partir de interpretar atributos
derivados de diversas fuentes \cite{Bao2004,Poppe2007}, como por
ejemplo el movimiento con sensores, ubicación u otras señales fisiológicas. 

El objetivo es identificar las acciones llevadas acabo por una persona
en base a observaciones realizadas sobre el mismo en el entorno que
se desenvuelve. Las aplicaciones en computación móvil y ubicua explotan
el contexto del usuario haciendo uso de sistemas \abbr{HAR} como
una herramienta tecnológica. Para tener un conocimiento acabado de
este trabajo, en esta sección se expone una descripción general de
aspectos clave como la definición del problema, el proceso de reconocimiento
estándar, las actividades estudiadas y las técnicas de aprendizaje.

\subsection{Definición del Problema}

\label{sec261:definicion-har}De manera a establecer el marco teórico
de estudio del problema \abbr{HAR} en este apartado se describe una
definición formal del problema. Considerando el objetivo y los elementos
de reconocimiento podemos definir el problema como\cite{LaraLabrador2013}:

\newtheorem{defi}{Definición}

\begin{defi}(Problema \abbr{HAR}) Dado un conjunto $S=\{S_{0},...,S_{k-1}\}$
de $k$ series de tiempo, cada una con una medida particular de cada
atributo, y definidas en el intervalo de tiempo $I=\left[t_{\alpha},t_{\omega}\right]$,
el objetivo es encontrar una partición temporal (sub-intervalo de
tiempo) $\left\langle I_{0},...,I_{r-1}\right\rangle $ en $I$, basado
en los datos de $S$ y el conjunto de etiquetas que representan la
acción realizada durante cada intervalo $I_{j}$ (Ej. quieto, caminando,
corriendo, etc.). 

Esto implica que cada intervalo $I_{j}$ son consecutivos, no vacíos,
no superpuestos y que ${\displaystyle \bigcup_{r-1}^{j=0}{I_{j}=I}}$
\end{defi}

Se asumen que las acciones consideradas no son realizadas simultáneamente,
es decir la persona no realiza la acción de correr y caminar al mismo
tiempo. Además, se debe notar que el problema \abbr{HAR} no es factible
a ser resuelto con una solución determinista. El numero de combinaciones
de valores de atributos y acciones puede ser muy grande, inclusive
infinito; y encontrar los puntos de transición es complejo teniendo
en cuenta que se desconoce la duración cada acción. 

Es por esta razón que las metodologías de aprendizaje automático son
utilizadas como proceso para reconocer actividades humanas por medio
de la técnica clasificación. Debido a la utilización de aprendizaje
automático para la resolución del problema, se requiere la siguiente
definición relajada del problema \abbr{HAR} descrito anteriormente: 

\newtheorem{defi}{Definición}

\begin{defi}(Problema \abbr{HAR} relajado) Dado (1) un conjunto
$W=\{W_{0},...,W_{m-1}\}$ de $m$ ventanas de tiempo del mismo tamaño,
donde cada una está total o parcialmente etiquetada, y que cada $W_{i}$
contiene un conjunto de series de tiempo $S_{i}=\{S_{i,0},...,S_{i,k-1}\}$
para cada $k$ atributos medidos, y (2) un conjunto $A\text{=}\{a_{0},...,a_{n-1}\}$
de etiquetas de actividades, el objetivo es encontrar una función
$f\colon S_{i}\rightarrow A$ que sea evaluada para todos los valores
posibles de $S_{i}$, tal que $f(S_{i})$ es lo más próximo a la acción
realizada durante $W_{i}$ \end{defi}

Considerar la utilización de esta definición relajada introduce un
error en el modelo durante las ventanas de transición entre actividades,
debido a que, en una ventana de tiempo una persona puede estar realizando
más de una acción. Sin embargo, el número de ventanas en transición
es menor al número total de ventanas por lo que el error introducido
por relajar el problema no es significativo para la mayoría de las
aplicaciones.

\subsection{Proceso de Reconocimiento }

\label{sec262:proceso-har}Al igual que en otras aplicaciones de aprendizaje
automático, el proceso de reconocimiento se divide en dos etapas bien
conocidas, la de entrenamiento y las pruebas (o evaluación).

\begin{figure}[!htbp]
\centering \includegraphics[width=0.7\linewidth]{capitulo-2/graphics/harsystem}
\caption[Flujo HAR]{Flujo general del Reconocimiento de Actividades Humanas}
\label{fig:harsystem} 
\end{figure}

En la \figref{fig:harsystem} se visualiza las fases comunes de las
dos etapas \cite{LaraLabrador2013}. La etapa de entrenamiento requiere
inicialmente un conjunto de datos recolectados en una serie de tiempo
con los atributos medidos a partir de individuos que realizan cada
actividad. Las series se dividen en ventanas de tiempo para aplicar
la extracción de muestras filtrando así la información relevante en
las señales en bruto. 

Más adelante, se utilizan métodos de aprendizaje para generar un modelo
de reconocimiento de actividades a partir del conjunto de datos colectado
a través de las características calculadas. Del mismo modo, para la
etapa de prueba o evaluación, se recogen datos durante una ventana
de tiempo, que se utiliza para extraer las mismas características
utilizadas en el modelo, estas se evalúan en el modelo de aprendizaje
previamente entrenado, generando una etiqueta de la actividad predicha.

En las siguientes secciones detallamos el objeto de la clasificación
(actividades humanas) y las funciones de cada etapa.

\subsection{Actividades Humanas}

\label{sec263:actividades-humanas} El diseño e implementación de
un sistema \abbr{HAR} depende totalmente de las actividades que serán
reconocidas. Por lo tanto, el cambio del conjunto de actividades que
un sistema reconoce convierte al problema en uno completamente distinto
a otro sistema construido con el mismo propósito.

Teniendo en cuenta esta razón, y de acuerdo a distintas publicaciones,
presentamos siete grupos distintos de actividades agrupados en la
siguiente tabla.

\begin{table}[htbp]
\centering{}%
\begin{tabular}{|l|p{9cm}|}
\hline 
\textbf{Grupo}  & \textbf{Actividades} \tabularnewline
\hline 
\hline 
Ambulatoria  & Caminar, correr, sentarse, pararse, quedarse quieto, acostarse, subir
escaleras, descender escaleras, usar escaleras mecánicas, usar elevador.\tabularnewline
\hline 
Transporte  & Andar en bus, bicicleta y conducir \tabularnewline
\hline 
En el teléfono  & Enviar mensajes de texto y hacer llamadas \tabularnewline
\hline 
Actividades diarias  & Comer, beber, trabajar en la PC, mirar TV, leer, cepillarse los dientes,
aspirar el piso, y otros. \tabularnewline
\hline 
Ejercitarse  & Alzar pesas, bicicleta estática, remo y otros. \tabularnewline
\hline 
Militares  & Arrastrarse, en cuclillas, abrir la puerta \tabularnewline
\hline 
Parte superior del cuerpo  & Masticar, hablar, mover la cabeza, tragar líquidos, mirar. \tabularnewline
\hline 
\end{tabular}\caption{Grupos de Actividades.}
\label{tabla:sencilla} 
\end{table}

Las actividades pueden separase en varios grupos de acuerdo a la duración
y la complejidad del evento. Los eventos cortos son movimientos de
transición y movimientos en base a gestos. Los eventos al que nos
enfocamos en este trabajo son aquellos que se componen de las actividades
básicas de larga duración que se caracterizan por las acciones continuas
y cíclicas de un individuo \cite{ReyesOrtiz2015}. Nuestro estudio
no se basa en actividades complejas que sean una secuencia de actividades
básicas y eventos cortos.

Definimos como objetivo de estudio detectar actividades básicas ambulatorias
y de transporte, de larga duración y sin cambios bruscos de transición.

\subsection{Colección de Datos}

La definición del método de colección de datos es un punto importante
en un sistema \abbr{HAR}. Según como se realiza la observación del
individuo existen: ambientes realistas, que son los ideales pero no
siempre es posible realizar este tipo de colectas. También existen
los ambientes cuasi realistas que se realizan en laboratorios simulando
las condiciones reales de las actividades. Por otro lado tenemos los
ambientes totalmente controlados en laboratorio.

Una falla en el diseño de un sistema \abbr{HAR} se puede dar por
no considerar las condiciones reales de las actividades, tales como
actividades no tenidas en cuenta, calibración de sensores, ruido,
etc.

Otra de las consideraciones a tener en cuenta en este punto es la
cantidad de individuos para realizar la colección, es recomendable
el mayor cantidad de individuos en distintos tipos de edades y condiciones
físicas.

\subsection{Extracción de Muestras}

Para cualquier problema de aprendizaje automático, la selección de
características se refiere al proceso de selección de un conjunto
significativo de características que aporten relevancia a la capacidad
de discriminación en un algoritmo de aprendizaje. Por otro lado, la
extracción de características, tiene como objetivo disminuir la cantidad
de características a utilizar mediante distintas transformaciones
entre ellas para obtener nuevas características reducidas sin perder
información relevante del conjunto de datos originales. La selección
y extracción de características también permite reducir los tiempos
de procesamiento en la fase de entrenamiento y aumenta el rendimiento
en la fase de evaluación

Dependiendo de la aplicación, las características requeridas para
la extracción de la información relevante pueden variar. En el caso
particular de \abbr{HAR}, una representación reducida de los datos
del sensor se puede utilizar como la entrada del algoritmo de reconocimiento.
Esto se logra mediante medición de la señal del sensor en varios dominios,
pudiendo ser en tiempo y frecuencia.

\subsection{Aprendizaje e Inferencia}

Varios enfoques de aprendizaje automático se han desarrollado a lo
largo de los años para resolver el problema de \abbr{HAR}. En su
mayoría a través de algoritmos de aprendizaje supervisado aunque también
se han propuesto métodos semi-supervisados y no supervisados.

Modelos de Bayes y basados en frecuencia han sido bien cubiertos en
toda la literatura \abbr{HAR}. Implican modelos basados en reglas
como Arboles de decisión \abbr{DT} y Selvas Aleatorias \abbr{RF},
algunos con un enfoques geométricos como vecinos cercanos \abbr{k-NN},
redes neuronales \abbr{ANN} y máquinas de soporte-vector \abbr{SVM},
y los métodos de clasificación probabilistas, por ejemplo clasificadores
de Bayes \abbr{NB}, y modelos ocultos de Markov \abbr{HMM}.

Otros aspectos relevantes para la selección del algoritmo de modelo
de aprendizaje incluyen: el consumo de energía, los requisitos de
memoria, interpretabilidad y complejidad de computo, etc. Estos aspectos
se agudizan si se utilizan dispositivos inteligentes .Como cuestión
de ejemplo, árboles de decisión podrían ser preferidos cuando se requiere
simplicidad en su implementación y \abbr{SVM} para aplicaciones de
alto rendimiento\cite{ReyesOrtiz2015}.

En el capitulo \ref{chap:Aprendizaje-Automatico} se detalla todo
lo referente a métodos de aprendizaje y los utilizados en este trabajo.

\section{Conclusión}

\label{sec27:conclusion}Este capitulo abarcó los tópicos primordiales
del estudio del reconocimiento de las actividades humanas. Los conceptos
descritos sirven de base de conocimiento para entender el problema
de \abbr{HAR} y además permiten tener una vista general de los componentes
principales para resolver el problema. Primeramente se cubre la motivación
en base a las aplicaciones, luego los medios disponibles para construir
estos sistemas: sensores, teléfonos móviles y aprendizaje automático,
y finalmente la definición metodológica y teórica del problema. 
 	%Cap 2: Reconocimiento de Actividades 

\chapter{Aprendizaje Automático}
\label{chap:Aprendizaje-Automatico}

\section{Introducción}
En el contexto de aprendizaje automático, los patrones deben ser descubiertos a partir de una serie de muestras que son denominadas instancias. El conjunto de entrada se denomina conjunto de entrenamiento. En nuestro caso específico, cada instancia es un vector de características extraída de señales en una determinada ventana de tiempo. Las muestras en el conjunto de entrenamiento pueden o no ser etiquetadas, es decir, tener asociada a una clase conocida (por ejemplo, caminar, correr, etc.). En algunos casos, tener datos etiquetados no es factible, ya que puede requerir un experto para examinar manualmente los ejemplos y asignar una etiqueta en base a su experiencia.

El proceso de aprendizaje es generalmente tedioso, caro y consume mucho tiempo en muchas aplicaciones de minería de datos. Existen dos enfoques de aprendizaje, es decir, aprendizaje supervisado y no supervisado, que se ocupan de datos etiquetados y no etiquetados, respectivamente. Puesto que un sistema de reconocimiento de la actividad humana debe devolver un resultado como caminar, sentarse, correr, etc., la mayoría de los sistemas de \abbr{HAR} utilizan algoritmos de aprendizaje supervisados. De hecho, podría ser muy difícil de discriminar actividades en un contexto completamente sin supervisión. Algunos otros sistemas funcionan de una manera semisupervisada en donde parte de los datos están sin etiqueta.

\section{Aprendizaje supervisado}
(FALTA DEFINICION DE MACHINE LEARNING)

\label{sec3:aprendizaje}De manera a establecer un marco teórico para los algoritmos de aprendizaje automático supervisados \cite{Rajaraman2011} se incluye la siguiente definición formal.

\label{def3:clasificacion}\newtheorem{defs}{Definición}
\begin{defs}(Clasificador (\emph{Machine Learning}, \abbr{ML})) Un algoritmo \abbr{ML} recibe un conjunto entrenamiento que se compone de varios pares $(\boldsymbol{x},y)$ conocidos como instancias de entrenamiento, donde
\begin{itemize}
\item $\boldsymbol{x}$ es un \emph{vector} de valores, llamado vector característico, e
\item $\boldsymbol{y}$ es la \emph{etiqueta}, el valor de clasificación para $\boldsymbol{x}$.
\end{itemize}
El objetivo del proceso \abbr{ML} es encontrar una función $y=f(\boldsymbol{x})$ cuya predicción de $\boldsymbol{y}$ asociada a valores desconocidos de $\boldsymbol{x}$ sea la mejor. El valor de $\boldsymbol{y}$ corresponde a valores arbitrarios pero es común encontrase con los siguientes casos:
\begin{enumerate}
\item $\boldsymbol{y}$ es un número del conjunto de los reales. Este caso corresponde a un problema de regresión.
\item $\boldsymbol{y}$ es un valor booleano verdadero-falso, también representado como $+1$ y $-1$. Este caso corresponde a una clasificación binaria. 
\item $\boldsymbol{y}$ es miembro de un conjunto finito discreto donde cada valor representa una clase particular. El problema es llamado de clasificación de múltiples clases.
\end{enumerate}
\end{defs}

El aprendizaje supervisado, comúnmente conocido como clasificación de múltiples clases discretas, ha sido un campo muy productivo que da lugar a un gran número de algoritmos. En la \tabref{tab3:clasificadores} se resumen los clasificadores más importantes en el estudio de \abbr{HAR}s según su tipo.
\begin{table}
\begin{centering}
\begin{tabular}{|l|l|}
\hline 
		Tipo 						& Clasificador							\\
\hline 
\hline 
		Arboles de Decisión 		& C4.5, ID3								\\
\hline 
		Clasificadores de Bayes 	& Redes de Bayes, \emph{Naïve} Bayes	\\
\hline 
		Basados en Instancia 		& k-vecinos proximos					\\
\hline 
		Transformación de dominio 	& Máquinas de soporte-vector			\\
\hline 
		Redes neuronales 			& Perceptron de múltiples capas			\\
\hline 
		Modelos de Markov 			& Modelos ocultos de Markov				\\
\hline
\end{tabular}
\par\end{centering}
\caption[Algoritmos de \abbr{ML}]{\label{tab3:clasificadores} Algoritmos de aprendizaje automático}
\end{table}

En las siguientes secciones nos concentraremos en un enfoque basado en arboles de decisión (Decision Tree, \abbr{DT}) para construir la función $f$. La función $f$, que corresponde a un árbol de múltiples niveles donde cada nodo aplica una función a $\boldsymbol{x}$ y determina sobre qué nodo hijo procede la búsqueda. Cada nodo posee un número arbitrario de nodos hijos. Los árboles de decisión son preferibles para las clasificación binaria o de múltiples clases, especialmente cuando la dimensión del vector característicos no es muy grande \cite{Rajaraman2011}.

\section{Arboles de Decisión}
El árbol de decisión es un método de aprendizaje predictivo sobre un conjunto de tuplas o instancias etiquetadas. Un árbol de decisión es una estructura de árbol similar a un diagrama de flujo, donde cada nodo interno (nodo no hoja) denota una prueba de un atributo, y cada rama representa el camino de a seguir luego de la evaluación, y cada nodo hoja es una etiqueta o clase.

\begin{figure}[!htbp]
	\centering
	\includegraphics[width=0.7\linewidth]{capitulo-3/graphics/ad_2}
	\caption[Árbol de decisión]{Árbol de decisión}
	\label{fig:arbolEjemplo}
\end{figure}

En la \figref{fig:arbolEjemplo} se muestra un típico ejemplo de árbol de decisión que representa el concepto de jugar o no Golf, teniendo en cuenta variables climática representadas en los nodos internos, y los nodos hojas denotan las decisiones de jugar o no al Golf. 

\section{Algoritmo C4.5}
El trabajo de J.R. Quinlan (citar y ver referencia para C4.5) propone un algoritmo de construcción de arboles de decisión denominado C4.5, el cual a su vez es una mejora o extensión al algoritmo ID3 (citar la referencia ID3). Este algoritmo genera un árbol de decisión a partir del paticionamiento recursivo de los datos de entrada. El árbol se construye mediante la estrategia \emph{profundidad-primero (depth-first)}.

Además, el algoritmo C4.5 utiliza una técnica heurística conocida como \emph{proporción de ganancia (gain-ratio)}. Que es una medida basada en información que considera diferentes números y probabilidades de los resultados de las pruebas. El algoritmo así, considera todas las pruebas posibles que puede dividir el conjunto de datos para seleccionar la prueba que le haya generado mayor ganancia de información. Para cada atributo discreto, se considera una prueba con $N$ resultados, siendo $N$ el numero de valores posibles que puede tomar el atributo. Para cada atributo continuo, se realiza la prueba binaria (1, 0) sobre cada uno de los valores que puede tomar el atributo de los datos. 

El algoritmo C4.5, ilustrado en el Algoritmo \ref{algoC45}, presenta las siguientes características:

\begin{itemize}
	\item Permite trabajar con valores continuos para los atributos, en dos ramas según se cumpla una de las siguientes condiciones  $ A_{i} <= N $ o $ A_{i} > N $ . 
	\item Por lo general los arboles generados por el mismo se caracterizan por ser menos frondosos, ya que cada hoja cubre una distribución de clases y no una clase en particular.
	\item Utiliza el método \emph{divide y vencerás} para generar el árbol de decisión inicial a partir de un conjunto de datos de entrenamiento.
	\item Se basan en la utilización del criterio de proporción de ganancia, definido como $ I(X_{i},C) / I(X_{i})  $. De esta manera consigue evitar que variables con mayor numero de categorías salgan beneficiadas en la selección. 
	\item Su implementación típica es recursiva.
\end{itemize}

\begin{algorithm}
	\begin{algorithmic}[1]
		\Require Conjunto de datos etiquetados $D$
		\Procedure{C4.5}{$ D $}
			\If {$D > \textit{es puro o cumple el criterio de parada} $} 
				\State\textit{Termina}
			\EndIf
			\ForAll{$a \in D $}
				\State $\textit{Computar información de division en a }$
			\EndFor
			\State $ a_{best} =$ Mejor atributo de división respecto al criterio 
			\State $ Arbol =$ Crear un nodo de decisión con $ a_{best} $ en la raíz 
			\State $ D_{v} =$ Introducir sub-conjunto de $D$ basado en división $ a_{best} $
			\ForAll{$ D_{v} $}
				\State $ Arbol_{v} = C4.5(D_{v}) $
				\State Unir $ Arbol_{v} $ al correspondiente arco del Árbol
			\EndFor
			\State 
			\Return $ Arbol $
		\EndProcedure
	\end{algorithmic}
	\caption{\label{algoC45}Árbol de Decisión - C4.5}
\end{algorithm}


\section{Información de Ganancia / Entropía} (REVISAR COMPLETO)
Tanto el algoritmo C4.5 como su predecesor el algoritmo ID3 utilizan la entropía como medida de selección para cada atributo.
En donde, dado un nodo $N$ que representa las tuplas de la partición $D$, el atributo con mayor valor de ganancia es elegido para la división de $N$. Con este enfoque, el atributo reduce al mínimo la información necesaria para clasificar las tuplas de las particiones resultantes y refleja la menor aleatoriedad o impureza de la partición. Este enfoque minimiza el número esperado de  ensayos necesarios para clasificar una tupla dada y garantiza un que se encuentre un árbol simple, pero no necesariamente el arbol más simple.

La información de ganancia necesaria para clasificar una tupla en $D$ está dada por:

\begin{equation}
Info(D) = - \displaystyle\sum_{i=1}^{m} p_{i}\log_2(p_{i})\label{eq3:info}
\end{equation}

donde $p_{i}$ es la probabilidad de una tupla arbitraria en $D$ pertenezca a la clase $C_{i}$ y que se estima que es igual a $ \lvert C_{i,D}} \rvert / \lvert D \rvert $. $Info(D)$ es la cantidad media de información necesaria para identificar la etiqueta de una tupla dada en $D$. Nótese, que la información que tenemos se basa solamente en las proporciones de las tuplas de cada clase. $Info(D)$ también se conoce entonces como entropía de $D$.

Ahora, supongamos que se requiera particionar las tuplas en $D$ conforme a los atributos $A$ conformados por $v$ valores distintos $ \{ a_{1},a_{2},...,a_{v} \}$. Si $A$ tiene valores discretos, estos valores se corresponderan directamente con los resultados de una prueba de $v$ sobre $A$. El atributo $A$ se puede utilizar para dividir $D$ en $v$ particiones o subconjuntos, $ \{ D_{1},D_{2},...,D_{v} \}$, donde $D_j$ contiene aquellas tuplas en $D$ que se corresponden con $a_j$ resultados de $A$. Estas particiones se corresponderían con las ramas que crecen a partir del nodo $N$. Idealmente, nos gustaría que esta división pueda producir una clasificación exacta de las tuplas. Es decir, nos gustaría que cada partición sea pura. Sin embargo, es bastante probable que el particionamiento sea impuro (por ejemplo, cuando una partición puede contener una colección de tuplas de diferentes clases en lugar de una sola clase). ¿Cuánto más información sería todavía necesaria (después de la partición) con el fin de llegar a una clasificación exacta? Esta cantidad se mide por:

\begin{equation}
Info_{A}(D) = \displaystyle\sum_{j=1}^{v} \displaystyle\frac{\lvert D_{j} \rvert}{\lvert D \rvert} \times Info(D_{j})\label{eq3:infoA}
\end{equation}

El termino $\displaystyle\frac{\lvert D_{j} \rvert}{\lvert D \rvert}$ actúa como el peso de la j-esima partición. $Info_{A}(D)$ es la información esperada requerida para clasificar la tupla de $D$ basada en la partición de $A$. Cuando menor sea la información esperada, mayor es la pureza de las particiones.

La información de ganancia se define como la diferencia entre lo requerido de información original (basado en la proporción de clases) y el nuevo requerimiento, obtenido luego de realizar la partición en $A$. Es decir,

\begin{equation}
Gain(A) = Info(D) - Info_{A}(D)\label{eq3:ganancia}
\end{equation}

En otras palabras, $Gain(A)$ nos dice cuanto se ganaría por la ramificación en $A$. Es la reducción esperada en el requisito de información causado por conocer el valor de $A$. El atributo A con ganancia mas alta, $Gain(A)$, se selecciona como atributo de división en el nodo $N$. Esto equivale a decir que queremos particionar en el atributo A que haría la mejor clasificación, por lo que la cantidad de información aun necesaria para clasificar las tuplas sea mínima, $InfoA(D)$.


%%\section{Algoritmo}

\section{Ejemplo de deducción de un nodo}

En la siguiente \tabref{tabla:sencilla2} se presenta un conjunto de entrenamiento $D$, con la clase etiquetada, de una base de datos de compra de productos electrónicos. Para este ejemplo, cada atributo tiene un valor discreto, y los atributos de valores continuos fueron generalizados. La clase \textit{Compra Computador} tiene dos valores posibles: $SI$ y $NO$, y por lo tanto existen dos clases distintas (es decir, $m=2$). Por lo tanto, tenemos la clase $C_1$ que corresponde a $SI$ y la clase $C_2$ corresponde a $NO$.

Existen nueve tuplas de la clase $SI$, y cinco de la clase $NO$. Para encontrar el criterio de división para las tuplas debemos calcular la ganancia de información de cada atributo. Primero usamos la ecuación \ref{eq3:info} para calcular la información esperada necesaria para clasificar una tupla en $D$:

\begin{equation*}
Info(D) = - \frac{9}{14}\log_2(\frac{9}{14}) - \frac{5}{14}\log_2(\frac{5}{14})	= 0.940 bits
\end{equation*}

\begin{table}[htbp]
	\caption{Ejemplo de conjunto de entrenamiento considerado}
	\label{tabla:sencilla2}
	\begin{tabular}{|c|l|l|l|l|c|}
		\hline 
		\textbf{Rid} & \textbf{Edad}    & \textbf{Ingreso} & \textbf{Estudiante} & \textbf{Rating Crediticio} & \textbf{Compra Computador?} \\
		\hline 
		\hline
		1   & Joven   & Alto    & No         & Justa             & NO                 \\ 
		2   & Joven   & Alto    & No         & Excelente         & NO                 \\ 
		3   & Mediana & Alto    & No         & Justa             & SI                 \\ 
		4   & Adulto  & Medio   & No         & Justa             & SI                 \\ 
		5   & Adulto  & Bajo    & Si         & Justa             & SI                 \\ 
		6   & Adulto  & Bajo    & Si         & Excelente         & NO                 \\ 
		7   & Mediana & Bajo    & Si         & Excelente         & SI                 \\ 
		8   & Joven   & Medio   & No         & Justa             & NO                 \\ 
		9   & Joven   & Bajo    & Yes        & Justa             & SI                 \\ 
		10  & Adulto  & Medio   & Si         & Justa             & SI                 \\
		11  & Joven   & Medio   & Si         & Excelente         & SI                 \\
		12  & Mediana & Medio   & No         & Excelente         & SI                 \\
		13  & Mediana & Alto    & Si         & Justa             & SI                 \\
		14  & Adulto  & Medio   & No         & Excelente         & NO                 \\
		\hline
		\hline
	\end{tabular}
\end{table}

Luego, necesitamos calcular la información requerida por cada atributo. Comenzamos con el atributo $Edad$, necesitamos observar la distribución de $SO$ y $NO$ por cada categoría o valor de este atributo. Para la categoría $Joven$, hay dos tuplas $SI$ y tres tuplas $NO$. Para la categoría $Mediana$, hay cuatro tuplas $SI$ y cero tuplas $NO$. Para la categoría $Adulto$, hay tres tuplas $SI$ y dos tuplas $NO$. Usando la ecuación \ref{eq3:infoA}, la información esperada necesaria para clasificar una tupla en $D$ si las tuplas son divididas según el atributo $Edad$ corresponde a:

\begin{equation*}
\begin{split}
Info_{Edad}(D) & = \frac{5}{14} \times ( -\frac{2}{5}\log_2(\frac{2}{5})-\frac{3}{5}\log_2(\frac{3}{5})  ) \\
			   & + \frac{4}{14} \times ( -\frac{4}{4}\log_2(\frac{4}{4})-\frac{0}{4}\log_2(\frac{0}{4})  ) \\
			   & + \frac{5}{14} \times ( -\frac{3}{5}\log_2(\frac{3}{5})-\frac{2}{5}\log_2(\frac{2}{5})  ) \\
			   & = 0.694 bits.
\end{split}
\end{equation*}

Por lo tanto, la ganancia esperada de realizar tal partición seria de acuerdo a la ecuación \ref{eq3:ganancia}:

\begin{equation*}
Gain(Edad) = Info(D) - Info_{Edad}(D) = 0.940 - 0.694 = 0.246 bits
\end{equation*}

De la misma manera, se procede a calcular las ganancias para cada atributo:
\begin{itemize}
  \item $Gain(Ingreso)$ = 0.029 bits, 
  \item $Gain(Estudiante)$ = 0.151 bits, y 
  \item $Gain(RatingCrediticio)$ = 0.048 bits. 
\end{itemize}

Dado que el atributo $Edad$ tiene mayor ganancia de información, es seleccionado para realizar la partición. El nodo $N$ es etiquetado con $Edad$, y se crean una rama por cada valor del atributo. Las tuplas se divinen en consecuencia, como se muestra en la \figref{fig:arbolPartEdad}.

\begin{figure}[!tbph]
	\centering
	\includegraphics[width=0.7\linewidth]{capitulo-3/graphics/dtree_parti}
	\caption[Árbol de decisión]{Árbol de decisión: Particionado en el atributo 'Edad'}
	\label{fig:arbolPartEdad}
\end{figure}

Notese que las tuplas en la partición $Edad = Mediana$ pertenecen todas a la clase $SI$, por lo tanto en esta rama se crea una hoja con etiqueta $SI$. Para las particiones $Edad = Joven$ y $Edad = Adulto$, el proceso de deducción se continua hasta que se generen nodos hoja que clasifican las tuplas en una sola clase. El
 árbol de decisión final deducido mediante algoritmo C4.5 se muestra en la \figref{fig:arbolFinal}.
 
(CREAR FIGURA CON ARBOL FINAL)

\section{Métricas de evaluación y Matriz de Confusión}
En general, la selección del algoritmos de clasificación para HAR ha sido apoyada meramente por evidencias empíricas (referencias). La gran mayoría de los estudios utilizan una validación cruzada con pruebas estadísticas para comparar el rendimiento de los clasificadores para un conjunto de datos determinado, como se visualiza en la \figref{fig:evaluacion}.

\begin{figure}[!tbph]
	\centering
	\includegraphics[width=0.7\linewidth]{capitulo-3/graphics/training-test}
	\caption[Diagrama de Evaluación de Clasificación]{Diagrama de Evaluación de Clasificación}
	\label{fig:evaluacion}
\end{figure}
	
Los resultados de una clasificación de un método en particular se suelen organizar en una matriz de confusión $M_{n \times n}$ para un problema de clasificación con $N$ clases.
En esta matriz, el elemento $M_{ij}$ es el numero instancias de la clase $i$ que fueron clasificados en la clase $j$.
Los siguientes valores se pueden obtener de la matriz de confusión en un problema de clasificación primaria:

\begin{itemize}
	\item Verdaderos Positivo (\emph{True Positive}, \abbr{TP}): El número de casos positivos que fueron clasificados como positivos.
	\item Verdaderos Negativo (\emph{True Negative}, \abbr{TN}): El número de casos negativos que fueron clasificados como negativos.
	\item Falso Positivo (\emph{False Positive}, \abbr{FP}): El número de casos negativos que fueron clasificados como positivos.
	\item Falso Negativo (\emph{False Negative}, \abbr{FN}): El número de casos positivos que fueron clasificados como negativos.
\end{itemize}

La precisión es la métrica más utilizada para resumir el rendimiento general de la clasificación para todas las clases y se define de la siguiente manera:

\begin{equation}
Exactitud = \frac{TP + TN}{TP + TN + FP + FN}\label{eq3:exactitud}
\end{equation}

La precisión, a menudo denominada valor predictivo positivo, es la proporción de casos positivos clasificados correctamente al número total de casos clasificados como positivos:
\begin{equation}
\mbox{Precisión} = \frac{TP}{TP + FP}\label{eq3:precision}
\end{equation}

La exhaustividad, también llamado tasa positiva verdadera, es la proporción de instancias positivas correctamente clasificadas al número total de instancias positivas:
\begin{equation}
Exhaustividad = \frac{TP}{TP + FN}\label{eq3:exaustividad}
\end{equation}


El Valor-F combina precisión y exhaustividad en un solo valor:
\begin{equation}
Valor-F = 2 \cdot \frac{\mbox{Precisión} \cdot Exhaustividad}{\mbox{Precisión} + Exhaustividad}\label{eq3:valorf}
\end{equation}

Aunque se definen para la clasificación binaria, estas métricas pueden generalizarse para un problema con $N$ clases. En tal caso, una instancia podría ser positiva o negativa según una mejor clase particular, por ejemplo, los positivos podrían ser todas las instancias de ejecución mientras que los negativos serían todas las instancias distintas de la ejecución.

\section{Evaluación de un modelo}

Utilizando el ejemplo anterior en la \tabref{tabla:sencilla2}, supongamos que se tomo una muestra de 10.000 tuplas obteniendose como resultado la \tabref{tabla:MatrizConfusion}.


\begin{table}[htbp]
	\caption{Matriz de Confusión}
	\label{tabla:MatrizConfusion}
	\begin{tabular}{|l|c|c|c|}
		\hline 
		\textit{Clase} & \textit{Com. Computador = SI}    &\textit{Com. Computador = NO} & \textit{Total}  \\
		\hline 
		\textit{Com. Computador = SI}	& 6.954   	& 46    	& 7.000     \\ 
		\hline 
		\textit{Com. Computador = NO}	& 412		& 2.588		& 3.000    	\\ 
		\hline
		\textit{Total}					& 7.366		& 2.634    	& 10.000	\\ 
		\hline
	\end{tabular}
\end{table}


Al evaluar la clase \textit{Compra Computador = SI} tenemos lo siguiente:

\begin{itemize}
	\item Verdaderos Positivo (\abbr{TP}): 6.954
	\item Verdaderos Negativo (\abbr{TN}): 2.588
	\item Falso Positivo (\abbr{FP}): 412
	\item Falso Negativo (\abbr{FN}): 46
\end{itemize}

A partir de estos datos podemos calcular el resto de las métricas:
\begin{equation*}
Exactitud = \frac{TP + TN}{TP + TN + FP + FN} = \frac{6.954 + 2.588}{6.954 + 2.588 + 412 + 46} = \frac{9542}{10000} = 0,9542
\end{equation*}

\begin{equation*}
\mbox{Precisión} = \frac{TP}{TP + FP} = \frac{6.954}{6.954 + 412} = 0,9440
\end{equation*}

\begin{equation*}
Exhaustividad = \frac{TP}{TP + FN} = \frac{6.954}{6.954 + 46} = 0,9934
\end{equation*}

\begin{equation*}
Valor-F = 2 \cdot \frac{\mbox{Precisión} \cdot Exhaustividad}{\mbox{Precisión} + Exhaustividad} 
		= 2 \cdot \frac{\mbox{0,9440} \cdot 0,9934}{\mbox{0,9440} + 0,9934}
		= 0,4840
\end{equation*}

(FAlTA UNA CONCLUSION EXPLICANDO LA MAGNITUD DE LOS VALORES Y EVALUAR SI EL SUPUESTO MODELO GENERADO ES BUENO)

     %Cap 3: Aprendizaje Automático
%Parte III: Desenvolvimiento

\chapter{Sistemas HAR móviles }

\label{chap4:sistemas-de-reconocimiento}

\section{Introducción}

\label{sec41:introduccion}El diseño de sistemas con conocimiento
del contexto promueven una interacción novedosa con los usuarios y
diversas aplicaciones en las áreas de ambientes inteligentes, repuesta
a emergencias, vigilancia y otros \cite{Choudhury2008}. Un sistema
con la capacidad de reconocer las actividades humanas por medio del
uso de sensores empotrados posee mecanismos para crear aplicaciones
de cuidado personal, salud y asistencia inteligente. El requerimiento
primordial de un sistema con una aplicación de contexto es que este
pueda ser portado continuamente como atuendo de sus usuarios (un sistema
\abbr{Wearable}). Por lo tanto, un sistema que acompaña continuamente
al usuario puede interaccionar oportunamente con el mismo ya que este
tiene la capacidad de observar en tiempo real las acciones de su portador.
La ventaja adicional de un sistema de este tipo es que puede ser desactivado
fácilmente o removido de la actividad diaria de su usuario.

En este capítulo se definen los componentes principales de un sistema
de reconocimiento de actividades humanas (sistemas \abbr{HAR}). El
objetivo principal del sistema \abbr{HAR} en conjunto es proveer
módulo base para aplicaciones novedosas de contexto. El módulo debe
ser capaz de reconocer varias actividades realizadas rutinariamente
de diferentes maneras, por diferentes usuarios y en diferentes condiciones
contextuales. Las funciones principales de los componentes descritos
en la primera sección exponen los mecanismos para implementar los
mismos en base trabajos relacionados de \abbr{HAR} \cite{Choudhury2008,ReyesOrtiz2015}. 

La última sección, enumera los requisitos no funcionales para lograr
una aplicación de contexto móvil y ubicua. Por un lado, las características
esperadas en una aplicación de esta naturaleza, y por el otro los
requisitos técnicos de los dispositivos móviles y los sensores empotrados
utilizados como instrumentos\footnote{\emph{hardware}} de implementación.

\section{Componentes}

\label{sec42:componentes}El diseño de la arquitectura de componentes
de un sistema \abbr{HAR} se rige de acuerdo a las guías de implementación
de una aplicación de aprendizaje automático (\abbr{ML}). De acuerdo
al proceso definido en la sección \ref{sec262:proceso-har}, se tiene
en cuenta la misma estructura de componentes y las mismas fases de
procesamiento de información. Además, se debe contemplar que el proceso
se divide en dos etapas: la etapa de entrenamiento y la de evaluación
\cite{LaraLabrador2013}. 

Ambas etapas requieren la implementación de los mismos componentes,
pero un sistema \abbr{HAR} práctico debe contemplar principalmente
la fase de evaluación, ya que el reconocimiento de actividades resulta
de una \emph{predicción} basado en un algoritmo de \abbr{ML} en-linea
(\emph{On-line learning}). Sin embargo, la etapa de entrenamiento
es un elemento clave para el sistema ya que es el punto de partida
para el \emph{aprendizaje} basado en modelo de \abbr{ML} y usualmente
se realiza bajo demanda (\emph{Off-line learning}).

Bajo el marco teórico de los sistemas \abbr{HAR} basados en \abbr{ML},
se han identificado unos componentes comunes para realizar las funcionalidades
de aprendizaje y predicción según \cite{Choudhury2008}. Un sistema
de reconocimiento de actividades posee tres componentes:
\begin{itemize}
\item un \emph{recolector }de medidas
\item un\emph{ procesador }de muestras 
\item un \emph{clasificador }de actividades
\end{itemize}
En la \figref{fig4:componentes-har} se muestra una vista general
de los componentes y sus interrelaciones. Las funcionalidades de cada
componente se describen a continuación. 

\begin{figure}[!tbph]
\centering{}\includegraphics[width=1\linewidth]{capitulo-4/graphics/diagrama_4_1}\caption[Arquitectura de sistema HAR]{\label{fig4:componentes-har}Componentes de los sistemas \abbr{HAR}}
\end{figure}


\section{Recolector de medidas}

\label{sec43:recolector-datos}El primer paso en el proceso de reconocimiento
primeramente consiste en recolectar medidas de señales obtenidas de
los sensores que observan continuamente a los usuarios. Además, se
procede a realizar un registro de manera organizada e indexada con
respecto al tiempo. Para capturar los datos se requieren instrumentos
de medición apropiados como los sensores (\abbr{Wearables}, véase
\ref{sec23:sensores}). Los sensores capturan las señales directamente
de los usuarios por medio de observaciones continuas, al estar anexados
al cuerpo; en la cintura, la muñeca, el pectoral, los muslos o en
la cabeza \cite{Bao2004}. También, los sensores podrían ser portados
por el usuario ya están comúnmente empotrados en dispositivos de uso
regular como los teléfonos móviles modernos, en relojes o lentes inteligentes
\cite{LaraLabrador2012,Choudhury2008}.

A continuación se describe el método común de registro y organización
de los datos obtenidos de las señales continuas, además de algunos
ejemplos de variables relevantes utilizadas.

\subsection{Registro}

El método de registro consiste en capturar las señales de un sensor
y separar las medidas en una o más variables dependiendo del tipo
de sensor. La organización de los registros se realiza con respecto
al tiempo. Es decir, se dispone de un flujo continuo de datos con
una marca de tiempo almacenados de manera secuencial en un medio permanente
para su posterior procesamiento. 

La marca de tiempo usualmente se mide \emph{mili}-segundos y dependiendo
del sensor el intervalo entre medidas puede variar en el mismo orden,
Ej. con tasa de salida de \texttt{60 \abbr{Hz} }se tendrían 60 medidas
en un segundo. 

Las señales de sensores pueden clasificarse de la misma manera que
los grupos citados en la sección \ref{sec23:sensores}, según movimiento,
posición, entorno y fisiológicas. A continuación se describen en detalle
cada grupo.

\subsubsection{Señales de Movimiento}

Los sensores de movimiento proveen señales altamente informativos
para \abbr{HAR} debido a que miden las fuerzas de aceleración y rotación
en tres ejes cuando son portados por sus usuarios. En esta categoría
de sensores se encuentran los acelerómetros y giroscopios. 

Los acelerómetros miden señales de acuerdo a diferentes tipos de movimientos,
incluyendo la aceleración lineal y centrípeta, la gravedad y vibración
en dos o tres dimensiones \cite{Goehl2007}. Las variables medidas
están expresadas en la magnitud de la aceleración ejercida sobre el
dispositivo con respecto la orientación del mismo (Ej. en reposo mide
$-9.8\,m\,s^{-2}$ en dirección al suelo). La señal de la aceleración
dada por $a(t)$ es un vector con respecto al tiempo con tres componentes
en cada eje $(x,y,z)$, cada uno representa una medida $a_{x}$, $a_{y}$
y $a_{z}$. En la \figref{fig4:muestra-ac} se despliegan las medidas
obtenidas por la señal $a(t)$ durante una actividad física determinada.

\begin{figure}[!tbph]
\begin{centering}
\includegraphics[width=1\columnwidth]{capitulo-4/graphics/signal_a3d}
\par\end{centering}
\caption[Señal de aceleración]{\label{fig4:muestra-ac}Señal de aceleración en tres dimensiones.
Los colores corresponden a los ejes de coordenadas representadas en
la siguiente figura.}
\end{figure}

Los giroscopios, o sensores de razón angular, miden señales de la
rapidez de rotación de los objetos en tres dimensiones \cite{Goehl2007}.
Las variables medidas están expresadas en velocidad angular de rotación
del dispositivo con respecto a los ejes de orientación del mismo (Ej.
en movimiento mide \foreignlanguage{english}{$-0.1\,rad\,s^{-1}$}
en relación a un eje fijo). La señal de la velocidad angular dada
por $w(t)$ es un vector con respecto al tiempo con tres componentes
en cada eje $(x,y,z)$, cada uno representa una medida $w_{x}$, $w_{y}$
y $w_{z}$. La orientación de los ejes con respecto al dispositivo
de medición depende del fabricante del mismo. Los teléfonos móviles
modernos poseen la orientación de los ejes de acuerdo a la siguiente
\figref{fig4:axis-phone}.

\begin{figure}[!tbph]
\begin{centering}
\includegraphics[scale=0.7]{capitulo-4/graphics/axis_device}
\par\end{centering}
\caption[Sistemas de coordenadas relativo a dispositivo]{\label{fig4:axis-phone}Ejes de coordenadas relativas a un teléfono
móvil moderno.}

\end{figure}

Las medidas obtenidas son variaciones con respecto a estos tres ejes
donde el valor $0$ está ubicado en el centro del dispositivo. Las
ventajas de los sensores de movimiento es su amplio uso al reconocer
actividades ambulatorias, como los citados en \cite{Bao2004,Kwapisz2011,ReyesOrtiz2015}.
Los acelerómetros y giroscopios poseen un bajo costo de fabricación,
bajo consumo de energía, y además están incluidos comúnmente en los
teléfonos móviles modernos \cite{Google2016s} debido a su utilidad
en mejorar la interacción humano-máquina por medio de gestos. Varios
trabajos publicados evidencian una alta precisión en \abbr{HAR} utilizando
al menos uno de estos sensores \cite{Bao2004,LaraLabrador2012}.

\subsubsection{Señales de Posición}

Los sensores de posición proveen señales con información adicional
que pueden ser utilizados para efectuar \abbr{HAR} y aplicaciones
de contexto con servicios basados en localización. En esta categoría
están los sensores de orientación (o brújula), magnetómetros y \abbr{GPS}
\cite{Google2016s}.

Las señales del \abbr{GPS} permiten acceder a las coordenadas geográficas
globales como el modo de transporte de un individuo, de acuerdo a
la velocidad estimada. Los teléfonos móviles modernos están equipados
con sensores para captar señales del sistema \abbr{GPS} y también
se puede estimar con buena precisión las coordenadas utilizando una
red celular \abbr{GSM}/\abbr{GPRS} y redes \abbr{WIFI} de corto
alcance. 

La señal de localización utiliza dos variables, conocidas como latitud
y longitud, y son medidas en la unidad radian (Ej. latitud $-57.2322\,rad$
y longitud $-25.3442\,rad$). Los valores de latitud y longitud son
coordenadas del \abbr{WGS} cuyos valores globales oscilan entre -180
a 180 en longitud, y -90 a 90 en latitud.

En la \figref{fig4:gps} se muestra una aplicación móvil para \abbr{Android}
que muestra el proceso de triangulación por satélites del sistema
\abbr{GPS}, donde el resultado es la estimación de las coordenadas
globales. 

\begin{figure}[!tbph]
\begin{centering}
\includegraphics[scale=0.8]{capitulo-4/graphics/gps}
\par\end{centering}
\caption[Coordenadas por GPS.]{\label{fig4:gps}Visualización de satélites y coordenadas globales
basada en triagulación GPS.}
\end{figure}

Sin embargo, los \abbr{GPS} tienen cobertura limitada por la dificultad
de obtener señal dentro de edificaciones, o si no está disponible
servicio de red celular o \abbr{WIFI}. También el alto consumo de
energía es un factor importante si las aplicaciones rastrean la localización
en tiempo real. Además, la ubicación de un individuo es particularmente
información sensible para la mayoría de los usuarios y se debe tener
especial atención a no comprometer la privacidad de los datos y tener
el consentimiento del usuario para que este pueda ser rastreado \cite{LaraLabrador2013}.

\subsubsection{Señales del Ambiente}

Los sensores de ambiente miden varios atributos del entorno que rodea
al usuario. Algunas señales como la temperatura del aire, presión
atmosférica, iluminación, humedad, y el ruido pueden proveer información
de utilidad para conocer mejor el habitad particular de un usuario.
En esta categoría están los barómetros, fotómetros, termómetros y
micrófonos.

Los sensores de ambiente solos no proveen información suficiente ya
que los individuos pueden realizar las actividades bajo diversas circunstancias
contextuales en términos de clima, ruido o iluminación. Por lo tanto
estos sensores pueden utilizarse de manera complementaria para detectar
sugestiones adicionales, Ej. el usuario está en el exterior de acuerdo
a la luminosidad, o se encuentra descansando debido a un nivel de
sonido y luminosidad baja \cite{LaraLabrador2013}.

\subsubsection{Señales de Fisiológicas}

Los sensores fisiológicos proveen señales de signos vitales de un
individuo. La información sobre el ritmo cardíaco, tasa de respiración
y temperatura del cuerpo podrían ser combinados para enriquecer el
contexto durante el reconocimiento en ciertas aplicaciones específicas
como las orientadas a la salud \cite{LaraLabrador2013}.

\section{Procesador de muestras}

\label{sec44:proceso-se=0000F1ales}El siguiente paso en el reconocimiento
de actividades consiste en procesar las señales obtenidas por sensores
y extraer características relevantes de los datos en bruto. El modelo
de reconocimiento se construye a partir de un conjunto de muestras
etiquetadas utilizando métodos de aprendizaje automático en la etapa
de entrenamiento. Durante la etapa de evaluación las entradas con
las que un modelo construido opera son muestras no clasificadas pero
generadas con la misma técnica de muestreo utilizada durante el entrenamiento.

El procesador de muestras depende en tres tareas bien diferenciadas
que se realizan automáticamente en ambas etapas del proceso de reconocimiento,
adicionalmente se realiza una tarea manual durante la etapa de entrenamiento
llamada etiquetado. A continuación se detallan cada tarea.

\subsection{Etiquetado}

\label{ssec44:labeling}El proceso de aprendizaje automático requiere
de una cantidad moderada de datos recolectados a partir de usuarios
mientras realizan actividades humanas objetivas a nuestro estudio.
Estos datos deben ser recolectadas y etiquetados utilizando un teléfono
móvil inteligente con una aplicación diseñada para el caso. Ej. Utilizando
un teléfono con \abbr{Android} y la aplicación \emph{SensorLog} \cite{SLog2016}. 

El protocolo de recolección consiste en alistar un grupo de personas
que porten un teléfono inteligente mientras realizan con conjunto
específico de actividades y registrar datos por medio de la aplicación.
Las actividades de interés descritas en la sección \ref{sec263:actividades-humanas}
deben ser realizadas portando el teléfono en el bolsillo donde cada
individuo realiza una sesión de caminata, trote, bicicleta o conducir
un vehículo por un periodo de tiempo de 10 a 15 minutos. La recolección
de datos llevada a cabo por medio de la aplicación disponible para
la plataforma \abbr{Android} es mostrada en la \figref{fig4:sensor-log}. 

\begin{figure}[!tbph]
\begin{centering}
\includegraphics[scale=0.8]{capitulo-4/graphics/sensorlog1}\includegraphics[scale=0.8]{capitulo-4/graphics/sensorlog2}
\par\end{centering}
\caption[Aplicación de entrenamiento SensorLog]{\label{fig4:sensor-log}Aplicación de entrenamiento \emph{SensorLog
}con interfaces de usuario para configurar y controlar la sesión de
entrenamiento}
\end{figure}

Así como se ve en la figura, la aplicación tiene una interfaz de usuario
simple que permite elegir los sensores de donde recolectar datos (Ej.
\abbr{GPS}, acelerómetro, giroscopio) para la sesión, las acciones
de iniciar y parar la actividad, y las etiquetas de la actividad que
el usuario realiza durante el entrenamiento. Las etiquetas utilizadas
en el proceso de recolección son:
\begin{itemize}
\item Caminar
\item Trotar
\item Quieto\footnote{A pesar de que esta actividad no representa un movimiento se incluye
como parte del estudio}
\item En bicicleta
\item En vehículo
\item \emph{Girando}\footnote{Esta no es una actividad física sino un comportamiento errático de
la orientación del teléfono}
\end{itemize}
La recolección se realiza obteniendo medidas a una razón de 60 a 100
\emph{mili}segundos por lo que se registran aproximadamente 60 a 100
muestras por segundo. Para asegurar la calidad de la recolección inicial
de datos las primeras sesiones de entrenamiento son supervisadas para
preparar las condiciones adecuadas del entrenamiento físico a elección
del participante.

\subsection{Filtro de Señal}

\label{ssec44:filtering}Teóricamente, si un dispositivo equipado
con sensores de movimiento está en reposo la señal de aceleración
$a(t)$ mediría cero (0) en dos ejes. Por ejemplo, en los ejes $x$
e $y$ no habría registro de medidas distintas a cero (0), y el eje
$z$mediría en dirección al suelo $-9.8\,m\,s^{-2}$. Sin embargo,
los sensores electrónicos pueden introducir cierta inestabilidad en
la señal (conocida como \emph{jitter}) provocando una fluctuación
en las lecturas debido a errores en la medición afectando la calidad
de los datos. Por lo tanto, a pesar de que un dispositivo con sensor
de movimiento esté completamente quieto en la mesa, las lecturas podrían
registrar ruido en los datos, Ej. errores del orden de $\pm0.005$. 

Entonces, para reducir ruido de la señal se debe aplicar uno o más
filtros. El filtro permite suavizar la señal por medio de una función
simple como la del cálculo de promedios variables (\emph{moving average})
o por un método como el de \emph{Butterworth} \cite{ReyesOrtiz2015}. 

Para el análisis de este trabajo se utilizó el filtro de señal de
promedios variables debido a su simplicidad y aplicabilidad basado
en otros trabajos como \cite{Yang2009}. El filtro de señal se puede
definir con la siguiente función $M$ descrita a continuación:

\label{def4:moving-average}\newtheorem{defi}{Definición}

\begin{defi}(\emph{Moving average}) Dada una secuencia $\left\{ a_{i}\right\} _{i=1}^{N}$,
un $n$-\emph{moving average} es una nueva secuencia $\left\{ s_{i}\right\} _{i=1}^{N-n+1}$
definida a partir de $a_{i}$ tomando la media aritmética de las \emph{sub}-secuencias
de tamaño $n$ donde,

\begin{eqnarray}
s_{i} & = & \frac{1}{n}\sum_{j=i}^{i+n-1}a_{j}
\end{eqnarray}

Así que las secuencias $S_{n}$ dado los $n$-\emph{moving averages}
serian 

\begin{eqnarray}
S_{2} & = & \frac{1}{2}(a_{1}+a_{2},a_{2}+a_{3},...,a_{n-1}+a_{n})
\end{eqnarray}

\begin{equation}
S_{3}=\frac{1}{3}(a_{1}+a_{2}+a_{3},a_{2}+a_{3}+a_{4},...,a_{n-2}+a_{n-1}+a_{n})
\end{equation}

y así sucesivamente.\end{defi}

Como ejemplo, en la \figref{fig4:filter-maf} se despliegan las gráficas
de una señal de aceleración $a_{y}$ para la dimensión $y$ y su correspondiente
señal filtrada con la función $M(a_{y})$ construida a partir de una
secuencia $S_{5}$ durante un intervalo de $1.6$ segundos. Como se
puede apreciar, la señal resultante está ligeramente suavizada debido
al filtrado de valores extremos resultado de perturbaciones bruscas
o ruido en la señal.

\begin{figure}[!tbph]
\begin{centering}
\includegraphics{capitulo-4/graphics/moving_average}
\par\end{centering}
\caption[Señal filtrada por la función $M$]{\label{fig4:filter-maf}Señal filtrada por promedios variables.}
\end{figure}


\subsection{Muestreo}

\label{ssec44:sampling}La realización de actividades humanas son
efectuadas durante periodos de tiempo de larga duración, en el orden
de los segundos o minutos. Esta taza es mucho mayor comparado con
las tazas de muestreos de los sensores, las cuales pueden llegar hasta
\texttt{250 \abbr{Hz}} (o \texttt{250} muestras por segundo). Una
simple medida capturada en un instante (Ej. la aceleración de $-2.5\,m\,s^{-2}$
en el eje $y$) no provee suficiente información para describir qué
actividad está llevando acabo un persona. Por lo tanto, las actividades
humanas deben ser reconocidas a partir de muestras extraídas en ventanas
de tiempo $w$ en vez de utilizar una sola medida instantánea en $t$. 

En la \tabref{tab4:ex-signal} se muestra una tabla con posibles medidas
instantáneas para la señal de aceleración $a(t)$ en una ventana de
tiempo $w_{j}$ con una taza de muestreo $r$asociada al sensor.

\begin{table}[!tbph]
\begin{centering}
\begin{tabular}{|c|c|c|c|c|}
\hline 
$w$ & $t$ & $a_{x}$ & $a_{y}$ & $a_{z}$\tabularnewline
\hline 
\hline 
$j$ & $0$ & \texttt{$1.3$} & \texttt{$-2.1$} & \texttt{$0$}\tabularnewline
\hline 
$j$ & $1/r$ & \texttt{$1.4$} & \texttt{$-2.3$} & \texttt{$0.1$}\tabularnewline
\hline 
$j$ & $2/r$ & \texttt{$1.1$} & \texttt{$-2.6$} & \texttt{$0$}\tabularnewline
\hline 
$j$ & $\ldots$ & \texttt{$\ldots$} & \texttt{$\ldots$} & \texttt{$\ldots$}\tabularnewline
\hline 
$j$ & $t_{max}$ & \texttt{$1.8$} & \texttt{$2.2$} & \texttt{$-0.4$}\tabularnewline
\hline 
\end{tabular}
\par\end{centering}
\caption[Medidas instantáneas de aceleración ]{\label{tab4:ex-signal}Ejemplo de medidas instantáneas de aceleración
en una ventana de tiempo.}
\end{table}

Las ventanas de tiempo hacen que las señales queden segmentadas en
muestras discretas utilizadas como unidades de reconocimiento de actividad.
De esta manera, cada ventana tiene inequívocamente una actividad humana
asociada y de esta manera podemos satisfacer el requisito planteado
por la definición del problema \ref{def2:harp-rel} en la sección
\ref{sec261:definicion-har}. 

Para incrementar la cantidad de muestras se utiliza una superposición
de $50\%$ con ventanas consecutivas y con tamaño de $2.56$ segundos,
como es recomendado por otros trabajos con técnicas de reconocimiento
\cite{Bao2004,ReyesOrtiz2015}. La superposición evita que ciertos
eventos se pierdan y las actividades se trunquen. La elección de este
tamaño de ventana produce muestras con medidas de tamaño fijo calculadas
aproximadamente con la ecuación \ref{eq4:window-size}. Este tamaño
es bastante conveniente por razones halladas en \cite{ReyesOrtiz2015}.

\begin{equation}
2.56\,\mathrm{sec}\times50\mathrm{\,Hz}=128\,\mathrm{medidas}\label{eq4:window-size}
\end{equation}

Cada muestra debe ser transformada en vectores característicos al
pasar por un proceso de extracción de valores que se describen en
la siguiente sección.

\subsection{Extracción}

\label{ssec44:extraction}Rememorando la definición \ref{def2:harp-rel}
en la sección \ref{sec261:definicion-har}, la incógnita principal
consiste en la comparación de las muestras provenientes de dos ventanas
$w_{1}$ y $w_{2}$, cada muestra con medidas $S_{i}$ que corresponden
a señales completamente distintas. Estas señales no serán idénticas
por más que sean obtenidas del mismo individuo realizando la misma
actividad física. Por lo tanto, para comparar muestras entres sí es
necesario extraer valores característicos en cada ventana $w_{j}$
por medio del filtro de información relevante y el calculo de valores
que identifiquen de cierta manera a las señales. 

El proceso de extracción se traduce en vectores característicos (\emph{\abbr{FV},
feature vectors}) con información relevante que componen varias métricas
calculadas en base a las ventanas $w_{j}$ en el dominio del tiempo.
Las ventanas también pueden transformarse en el dominio de la frecuencia
con métodos discretos de \emph{Fourier} utilizando algoritmos \abbr{FFT}
con números reales. 

Los métodos de cálculo pueden ser de tipos estadísticos y estructurales
\cite{LaraLabrador2013} y en base a lo propuesto ya por otros trabajos
precedentes como \cite{Yang2009,Bao2004}. Las métricas estadísticas
que son incluidas en este trabajo son la media, el máximo, el mínimo,
la desviación estándar, energía, entropía, asimetría, curtosis, rango
intercuantil y coeficientes de autoregresión.

Las métricas son aplicadas a la señal de aceleración utilizando la
magnitud del vector de aceleración calculada a partir de las medidas
de fuerza ejercida en los tres ejes. Se eligió utilizar la magnitud
$\lVert a\rVert$ del vector, y excluir los valores unitarios en cada
dimensión $a_{x}$, $a_{y}$, $a_{z}$ para simplificar el proceso,
cancelar el efecto de variaciones en la orientación del teléfono \cite{Schneider2014}
y reducir la dimensión del vector característico a un tercio ($\frac{1}{3}$).
En la \tabref{tab4:metricas} se resumen las métricas utilizadas con
su formulación matemática tomadas de\cite{ReyesOrtiz2015} para la
señal de ventana $s$ de tamaño $n$ con la magnitud del vector $a$.

\begin{table}[!tbph]
\begin{centering}
\begin{tabular}{|l|l|l|}
\hline 
Función & Descripción & Formulación\tabularnewline
\hline 
\hline 
mag(\textbf{a}) & Magnitud de la aceleración & $\lVert a\rVert=\sqrt{a_{x}^{2}+a_{y}^{2}+a_{z}^{2}}$\tabularnewline
\hline 
mean(\textbf{s}) & Media aritmética & $\overline{s}=\frac{1}{n}\sum_{i=1}^{n}s_{i}$\tabularnewline
\hline 
std(\textbf{s}) & Desviación estándar & $\sigma=\sqrt{\frac{1}{n}\sum_{i=1}^{n}\left(s_{i}-\overline{s}\right)}$\tabularnewline
\hline 
max(\textbf{s}) & Valor máximo en \textbf{n} & $\max(s)$\tabularnewline
\hline 
min(\textbf{s}) & Valor mínimo en \textbf{n} & $\min\left(s\right)$\tabularnewline
\hline 
skewness(\textbf{s}) & Asimetría de señal en frecuencia & $\mathrm{E}\left[\left(\frac{s-\overline{s}}{\sigma}\right)^{3}\right]$\tabularnewline
\hline 
kurtosis(\textbf{s}) & Curtosis de señal en frecuencia & $\frac{\mathrm{E}\left[\left(s-\overline{s}\right)^{4}\right]}{\mathrm{E}\left[\left(s-\overline{s}\right)^{2}\right]^{2}}$\tabularnewline
\hline 
energy(\textbf{s}) & Energía: Promedio de suma de cuadrados & $s_{rms}=\frac{1}{n}\sum_{i=1}^{n}s_{i}^{2}$\tabularnewline
\hline 
entropy(\textbf{s}) & Entropía de la señal & $s_{s}=\sum_{i=1}^{n}c_{i}\log\left(c_{i}\right)\mathrm{\mathtt{,}}c_{i}=s_{i}/\sum_{j=1}^{n}s_{j}$\tabularnewline
\hline 
irq(\textbf{s}) & Rango intercuantil & \selectlanguage{english}%
Q3(\textbf{s}) - Q1(\textbf{s})\selectlanguage{spanish}%
\tabularnewline
\hline 
autoregression(\textbf{s}) & Coeficientes de autoregresión Burg de 4to orden & $ar=arburg\left(s,4\right)\mathtt{,}ar\in\mathbb{R}^{4}$\tabularnewline
\hline 
meanFreq(\textbf{s}) & Promedio ponderado de señal en frecuencia & $\sum_{i=1}^{n}\left(is_{i}\right)/\sum_{j=1}^{n}s_{j}$\tabularnewline
\hline 
\end{tabular}
\par\end{centering}
\caption{Métricas para el calculo de vectores característicos}
\end{table}

Un ejemplo de un proceso de extracción que mapea una ventana $w_{j}$
de señales originales en un vector característico $F_{j}$ de dimensión
$m$ representada como ejemplo en la \tabref{tab4:features}.

\begin{table}[!tbph]
\begin{centering}
\begin{tabular}{|c|c|c|c|}
\hline 
$w$ & $f_{0}$ & $\ldots$ & $f_{m}$\tabularnewline
\hline 
\hline 
$0$ & $2.71$ & \texttt{$\ldots$} & \texttt{$-2.30$}\tabularnewline
\hline 
$\ldots$ & $\ldots$ & \texttt{$\ldots$} & \texttt{$\ldots$}\tabularnewline
\hline 
$j$ & $2.91$ & \texttt{$\ldots$} & \texttt{$-2.11$}\tabularnewline
\hline 
$\ldots$ & $\ldots$ & \texttt{$\ldots$} & \texttt{$\ldots$}\tabularnewline
\hline 
$k-1$ & $2.56$ & \texttt{$\ldots$} & \texttt{$-2.56$}\tabularnewline
\hline 
\end{tabular}
\par\end{centering}
\caption[Métricas de proceso de extracción]{\label{tab4:features}Métricas procesadas a partir de las medidas
de entrenamiento}
\end{table}


\section{Clasificador de actividades }

\label{sec45:clasificador}Utiliza las muestras extraídas para construir
un modelo e predecir qué actividad probable está realizando un individuo
en un determinado instante.

\subsubsection{Clasificación}

\subsubsection{Reconocimiento}

\section{Capacidades deseables}

\subsection{Características no funcionales}

\label{ssec46:caracteristicas}Existen un conjunto de características
deseables que deben ser satisfechas para la construcción efectiva
de los sistemas de reconocimiento. Estas características abordan cuestiones
de diseño importantes que conciernen a la calidad y al funcionamiento
del sistema:
\begin{enumerate}
\item Portabilidad, el sistema utiliza sensores adjuntos a los individuos
(Ej. el acelerómetro) y no deben obstruir las actividades cotidianas
de los usuarios durante su uso. El fin es de evitar que se afecte
la adopción masiva del sistema. 
\item Conectividad, el sistema debe transmitir de manera confiable los datos
recolectados y/o procesados a algún componente desplegado de forma
remota. 
\item Almacenamiento, el sistema debe persistir los datos recolectados y/o
procesados de manera local en el dispositivo móvil con el fin de mantener
la calidad y minimizar la cantidad transferida a otros componentes.
\item Procesamiento, el sistema debe procesar y transformar los datos en
bruto para producir información relevante para el reconocimiento de
actividades.
\item Ubiquidad, el sistema debe operar en cualquier condición y contexto
en que la persona se encuentre sin interferir u obligar al usuario
a interactuar con el sistema.
\item Uso de energía, el sistema debe preservar el uso de energía en los
dispositivos móviles que están implementados. La lectura de datos,
el procesamiento y la conectividad no deben incurrir en gastos excesivos
de energía para que el sistema pueda operar.
\item Privacidad, el sistema debe mantener de manera confidencial los datos
recolectados y/o producidos durante la adopción masiva del sistema,
además de alertar sobre la utilización de datos sensibles que requieran
el consentimiento del usuario.
\end{enumerate}

\subsection{Dispositivos móviles}

\label{ssec46:dispositivos-moviles}Descripción técnica de dispositivos
móviles: procesador, memoria, sensores y almacenamiento

\subsubsection{Teléfonos móviles}

\subsubsection{Relojes inteligentes}

\subsection{Sensores empotrados}

\label{ssec46:sensores-empotrados}Descripción técnica de los sensores
de aceleración, variables, orientación en dispositivo, unidades de
medida, precisión vs consumo.

\subsubsection{Acelerómetro}

\subsubsection{Giroscopio}

\subsubsection{GPS/WIFI}

\section{Conclusión}

Resumen
 	%Cap 4: Sistemas HAR

\chapter{HARDroid: Reconocedor de Actividades Humanas}

\label{chap5:hardroid}

\section{Introducción}

\section{Conclusión}
		%Cap 5: HARDroid: Un Sistema Reconocedor Colaborativo
%Parte IV: Resultados

\chapter{Evaluación y Resultados}

\label{chap6:evaluacion}

\section{Introducción}

En el ámbito de investigación y desarrollo de sistemas \abbr{HAR}
existen dos elementos vitales: la recolección de datos experimentales
y generación del conjunto de entrenamiento. En este capítulo, describimos
de manera general estos elementos en las primeras dos secciones. La
sección \ref{sec6:recoleccion} describe los aspectos relacionados
a la captura de datos, determinando los requisitos mínimos de los
teléfonos móviles utilizados y describiendo el procedimiento guía
para experimentación. Además, la sección \ref{sec6:clasificacion}
describe los resultados obtenidos al aplicar las técnicas descritas
en la sección \ref{sec44:proceso-se=0000F1ales} para transformar
datos sensoriales a un conjunto de entrenamiento para la clasificación
de actividades humanas. Finalmente, en la sección \ref{sec6:resultados}
los resultados producidos al utilizar el conjunto entrenamiento disponible
son presentados. Esto incluye la validación del modelo construido
con el algoritmo de \emph{Machine Learning} (\abbr{ML}) escogido
que confirma su usabilidad en ambientes productivos. También se expone
la evaluación de los resultados producidos por \emph{\abbr{HARDroid}
}con la aplicación \emph{ActivitySurvey} desarrollada para esta tarea.

\section{Datos Experimentales}

\label{sec6:recoleccion}En base a trabajos precedentes en sistemas
\abbr{HAR}, se han dispuesto datos experimentales para entrenar clasificadores
de actividades humanas tales como en \cite{ReyesOrtiz2013}. Estos
datos están disponibles como fuente para diversos estudios de investigaciones
en este ámbito. 

Sin embargo, los datos de sensores capturados con teléfonos móviles
son escasos. Es por esta razón que este trabajo requirió realizar
una colecta acorde a los objetivos de estudio del mismo. El procedimiento
y los datos recolectados por medio de experimentación se describen
a continuación.

\subsection{Instrumentación}

Como es sabido el desarrollo de \emph{\abbr{HARDroid} }se concreto
completamente para la plataforma \emph{\abbr{Android} }donde aquí
se detallan los rasgos técnicos de las herramientas utilizadas para
su concepción y evaluación.

\subsubsection{Teléfonos inteligentes}

Escoger las herramientas apropiadas para desarrollar un sistema \abbr{HAR}
requiere de la evaluación de dispositivos móviles disponibles en el
mercado teniendo en cuenta los criterios citados en la \secref{sec24:dispositivos-moviles}:
\emph{hardware}, sensores y software de plataforma. En el periodo
de evaluación de este trabajo (2016), la cantidad de teléfonos inteligentes
con sensores de aceleración fue vasta, algunos de los cuales se listan
en la \tabref{tab6:dispositivos} con sus características relevantes.

\begin{table}[h]
\begin{centering}
\begin{tabular}{|l|l|l|l|l|}
\hline 
Marca/modelo & CPU & RAM/ROM & Sensor\footnote{Sensor de aceleración} & Android\tabularnewline
\hline 
\hline 
{\small{}LG G2} & {\small{}1.2GHz Cortex-A7} & {\small{}1GB/8GB} & {\small{}BMI160} & {\small{}Nougat 7.0}\tabularnewline
\hline 
{\small{}LG Nexus 5X} & {\small{}Q1.4Ghz Cortex-A53} & {\small{}2GB/32GB} & {\small{}BMC150} & {\small{}Lollipop 5.0.2}\tabularnewline
\hline 
{\small{}Motorola G 2nd} & {\small{}1.2GHz Cortex-A7} & {\small{}1GB/8GB} & {\small{}3-axis Acc} & {\small{}Marshmallow 6.0}\tabularnewline
\hline 
{\small{}Huawei Mate 9} & {\small{}Q2.4GHz Cortex-A73} & {\small{}4GB/64GB} & {\small{}LSM6DSM} & {\small{}Nougat 7.0}\tabularnewline
\hline 
{\small{}Huawei Mate 8} & {\small{}Q2.3GHz Cortex-A72} & {\small{}3GB/32GB} & {\small{}LSM330 3-axis} & {\small{}Marshmallow 6.0}\tabularnewline
\hline 
{\small{}Samsung S6} & {\small{}Q2.1GHz Cortex-A57} & {\small{}3GB/32GB} & {\small{}MPU6500} & {\small{}Nougat 7.0}\tabularnewline
\hline 
{\small{}Samsung A5} & {\small{}Q1.2GHz Cortex-A53 } & {\small{}2GB/16GB} & {\small{}BOSCH Acc} & {\small{}Lollipop 5.1.1}\tabularnewline
\hline 
\end{tabular}
\par\end{centering}
\caption[Especificaciones de los teléfonos inteligentes]{\label{tab6:dispositivos}Especificaciones de los teléfonos inteligentes
de entrenamiento.}
\end{table}

La elección fue en base a los dispositivos móviles disponibles, propiedad
de los voluntarios durante las sesiones de experimento.

\subsubsection{Entorno de desarrollo }

Las aplicaciones móviles desarrolladas en este trabajo están enteramente
construidas en la plataforma \emph{\abbr{Android}} donde fueron utilizados
los programas destinados para el caso \cite{Android2016}:
\begin{itemize}
\item \emph{Android Studio}: Entorno de desarrollo integrado para proyectos\emph{
}de software.
\item \emph{Android} \emph{Software Development Kit }(\abbr{SDK}): Herramientas
y librerías \abbr{API} requeridas para construir aplicaciones.
\item \emph{Gradle}: Programa de automatización de tareas de construcción
de aplicaciones.
\end{itemize}
Las aplicaciones desarrolladas fueron escritas utilizando el lenguaje
\emph{Java }principalmente. Tanto las interfaces de usuario, servicios
de aplicación y las tareas de computación intensivas de acceso a sensores,
procesamiento, algoritmos de \abbr{ML} y almacenamiento de datos
fueron hechos integramente en Java.

\subsection{Procedimiento Guía }

Con el objetivo de obtener un conjunto de datos adecuado a este estudio
de \abbr{HAR} se realizó un experimento de entrenamiento con grupo
de voluntarios. Un grupo de 8 personas entre las edades de 20 y 38
estuvieron dispuestos para esta tarea donde la edad media de la población
esta comprendida en $30.5\pm5$ años. 

El procedimiento guía de captura de datos se instruyó con el uso del
teléfono móvil como prenda sujeta al bolsillo o en la cintura mientras
se realiza una actividad física predeterminada. El planeamiento del
experimento consitió en realizar en orden por un periodo de 10 a 15
minutos las tres actividades básicas y dos de transporte:
\begin{itemize}
\item Caminar (WALKING)
\item Trotar (RUNNING)
\item Quieto (STILL)
\item En bicicleta (ON\_BICYCLE)
\item En auto (ON\_VEHICLE)
\end{itemize}
El experimento de entrenamiento se resume en la \tabref{tab6:sesiones},
donde se incluyen las sesiones de los voluntarios y el tiempo en minutos
invertido en la actividad etiquetada. Cada persona realizo una sesión
de entrenamiento al menos una vez.

\begin{table}[h]
\begin{centering}
\begin{tabular}{|c|c|c|c|c|c|}
\hline 
Sujeto & WALKING & RUNNING & STILL & ON\_BICYLE & ON\_VEHICLE\tabularnewline
\hline 
\hline 
AG & 80 & 33 & 12 & 17 & 7\tabularnewline
\hline 
SG & 15 & 15 & - & - & -\tabularnewline
\hline 
SF & 43 & 25 & - & 13 & -\tabularnewline
\hline 
GA & 30 & 2 & - & - & -\tabularnewline
\hline 
BV & - & 17 & - & - & -\tabularnewline
\hline 
PV & 17 & 19 & 12 & - & 7\tabularnewline
\hline 
SY & 37 & 13 &  & 26 & 11\tabularnewline
\hline 
MD & 29 & 4 & - & - & -\tabularnewline
\hline 
\end{tabular}
\par\end{centering}
\caption{\label{tab6:sesiones}Resumen de sesiones de entrenamiento}
\end{table}


\subsection{Captura de Datos}

TODO:

\section{Generación del Clasificador HAR}

\label{sec6:clasificacion}

\subsection{Etiquetado}

\subsection{Conjunto Entrenamiento }

\subsection{Clasificador de WEKA}

\section{Resultados}

\label{sec6:resultados}Describir resultados de experimento con HARDroid.
		%Cap 6: Evaluación
%Parte V: Final

\chapter{Conclusiones y Trabajos Futuros}

\label{cap8:conclusiones-y-trabajos-futuros}

\section{Conclusiones del presente trabajo}

\label{sec81:conclusiones}El reconocimiento de actividades humanas
es un área de investigación multidisciplinaria con constantes aportes
y diversos usos en la actualidad debido a su relevancia en la computación
contextual y ubicua. Los avances en la miniaturización de los sistemas
de computación y sensores hacen del área un tanto atractiva como desafiante
para explorar y proponer nuevas ideas en este ámbito \cite{LaraLabrador2013}. 

Los esfuerzos realizados en este trabajo han tenido en cuenta además
la colaboración y el uso de Internet de manera a aprovechar los mismos
en la mejora de los sistemas de reconocimiento, y como se mostró en
la sección anterior la aplicación de resultados colaborativos puede
ser aprovechada para mejorar el sistema desarrollado. 

Así, en este trabajo se construyó un sistema de reconocimiento de
actividades colaborativo que utiliza tres elementos principales: teléfonos
móviles inteligentes, una librería que a la par se ha hecho disponible
como software libre y la Internet. 

El desarrollo del trabajo se llevó acabo en dos etapas. La primera
etapa consistió en el entrenamiento de un clasificador y la segunda
etapa consistió en la construcción de un sistema reconocedor de actividades
para teléfonos móviles. Los principales aportes de este trabajo se
produjeron en la segunda etapa:
\begin{itemize}
\item Por una parte se tiene como resultado tangible un componente de código
abierto reutilizable para reconocer actividades humanas. Este componente
es denominado \emph{\abbr{HARDroid}}.
\item Por otra parte este componente posibilita el mejoramiento iterativo
de su desempeño mediante un esquema colaborativo.
\end{itemize}
Respecto al primer punto, es importante destacar que \emph{\abbr{HARDroid}}
puede ser aprovechado por otras aplicaciones móviles de dos maneras
concretas \cite{GimenezYegros2016a}: 
\begin{itemize}
\item el mismo puede ser incluido en tiempo de construcción, o
\item puede ser aprovechado como un servicio a través de una interfaz de
integración.
\end{itemize}
Los experimentos realizados en el presente trabajo y documentados
en la sección previa muestran que \emph{\abbr{HARDroid}} es capaz
de producir un modelo en el cual se observan las siguientes características:
\begin{itemize}
\item Una alta tasa de aciertos (de 91\% a 92\%), y por lo tanto una baja
tasa de errores. Esto se puede verificar por la \tabref{tab6:matriz-confusion}
junto con las métricas de precisión y exhaustividad resultantes.
\item La capacidad de extender el modelo para reconocer diversos tipos de
actividades así como en bicicleta, en vehículo, entre otros.
\item La posibilidad de extender el modelo colaborativamente mediante la
inclusión de muestras de aciertos colectadas en campo.
\end{itemize}
(REVISAR CONCLUSIONES COMPLEMENTANDO CON REF A LOS RESULTADOS)

\section{Trabajos Futuros propuestos}

\label{sec82:trabajos-futuros}Luego de la experiencia obtenida y
documentada en el presente trabajo, teniendo en cuenta la amplia aplicabilidad
del reconocimiento de actividades humanas en especial en aplicaciones
para dispositivos móviles, a continuación se proponen algunos trabajos
que se desprenden del presente:
\begin{enumerate}
\item Segmentar los grupos de individuos distintos por rangos de edad, sexo
y/o factores fisiológicos y otros, de manera a generar modelos específicos
para cada grupo que posibiliten el desarrollo de una nueva gama de
aplicaciones. 
\item Incorporar al sistema \emph{HARDroid} desarrollado otros métodos de
aprendizaje automático que permitan mejorar la tasa de aciertos. Entre
estos se sugieren por ejemplo las técnicas de vectores de soporte
(\abbr{SVM}) o redes neuronales (\abbr{ANN}).
\item Incluir más variables en el procesamiento de señales de entrada para
mejorar las predicciones teniendo en cuenta la orientación del dispositivo. 
\item Extender el sistema \emph{HARDroid} para identificar otras actividades
humanas aplicado a otras disciplinas o contextos. Por ejemplo en medicina
para detectar caídas, posturas o actividades no recomendadas. En el
campo militar para recomendar actividades más favorables para situación.
\item Utilizar o aprovechar otros accesorios que agreguen variables relevantes
en el reconocimiento de actividades dependiendo del contexto. Por
ejemplo, un reloj inteligente (\emph{\abbr{Smartwatch}}).
\end{enumerate}

		%Cap 7: Conclusiones y Trabajos Futuros

%\appendix   
% los capítulos que incluyas a partir de aquí aparecen
% como apendices
%
\chapter{Sincronización de Resultados}

\label{chapA:rest-api}

\subsection{Firma}

%\include{anexos/lecciones-aprendidas}
% estos comandos generan la bilbiografía

% Glosario
\hypertarget{abbr}{\printnomenclature{}}

%referencias
\bibliography{referencias}


\end{document}
