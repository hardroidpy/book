
\chapter*{Resumen}
Human Activity Recognition (HAR) es un tema de investigación ampliamente cubierto en la última década por su relevancia en áreas donde el contexto de los usuarios es importante para construir aplicaciones interactivas. Las aplicaciones móviles para teléfonos inteligentes tienen la capacidad de capturar datos del entorno por medio de sensores y en conjunción con algoritmos que aprovechan la información sensible al contexto, se convierten en una poderosa plataforma de desarrollo. En este trabajo, proponemos un sistema HAR denominado HARDroid que está específicamente diseñado para detectar actividades comunes de los usuarios. Además, los datos recopilados de los usuarios en pruebas de campo se tienen en cuenta para mejorar el clasificador de reconocimiento de actividad. HARDroid está disponible gratuitamente como una biblioteca que puede incluirse en las aplicaciones de Android. Finalmente, se presenta una evaluación que compara el clasificador inicial con un clasificador mejorado, logrando una exhaustividad del 91.34 \% y una precisión del 92.04 \%.


\chapter*{Abstract}
Human Activity Recognition (HAR) is a research topic broadly covered in the last decade for its relevance in areas where the users’ context is important to build interactive applications. Smartphone applications have the capability to collect data from the environment and along with algorithms that take advantage of context-aware information becomes a powerful development platform. In this work, we propose a HAR System denominated HARDroid that is specifically designed to detect common user activities. Furthermore, data collected from users on ground are taken into account to improve the activity recognition classifier. HARDroid is freely available as a library that may be included in Android applications. Finally, an evaluation that comparing the initial classifier with an improved classifier is presented, achieving a recall of 91.34\% and a precision of 92.04\%.

