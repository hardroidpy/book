\nomenclature{Android}{Android es un sistema operativo móvil desarrollado por Google. Está basado en el kernel de Linux.}

\nomenclature{HAR}{Human Activity Recognition} 

\nomenclature{iOS}{Es un sistema operativo móvil creado y desarrollado por Apple Inc.}

\nomenclature{GPS}{Global Positioning System}

\nomenclature{API}{Application Programing Interfaces}

\nomenclature{SaaS}{Sofware as a Service}

\nomenclature{DT}{Decision Trees}

\nomenclature{RF}{Random Forest}

\nomenclature{SVM}{Support-Vector Machines}

\nomenclature{NB}{Naive Bayes}

\nomenclature{HMM}{Hidden Markov Models}

\nomenclature{ANN}{Artificial Neural Networks}

\nomenclature{k-NN}{k-Nearest Neighbors}

\nomenclature{HRM}{Hear Rate Monitor}

\nomenclature{Wearables}{Dispositivos o aparatos que pueden ser atuendos de un individuo}

\nomenclature{ML}{Machine Learning}

\nomenclature{Hz}{Hertz, unidad de medida de ciclos por segundo}

\nomenclature{WIFI}{Wireless Fidelity, permite tener acceso o conectarse a una red utilizando ondas de radio}

\nomenclature{Bluetooth}{Estándar para red personal}

\nomenclature{GPRS}{General Packet Radio Service, servicio de datos moviles en comunicaciones celulares 2G/3G}

\nomenclature{GSM}{Global System for Mobile, sistema de comunicación celular global.}

\nomenclature{3G}{Tercera generación de sistema de comunicación celular.}

\nomenclature{4G}{Cuarta generación de sistema de comunicación celular.}

\nomenclature{SMS}{Short Message Service}

\nomenclature{MMS}{Multimedia Messaging Service}

\nomenclature{Kinect}{Sensor de profundidad de Microsoft}

\nomenclature{WGS}{World Geodetic System}

\nomenclature{FFT}{Fast Fourier Transform}

\nomenclature{FV}{Feature Vectors}

\nomenclature{Smartphones}{Teléfonos inteligentes}

\nomenclature{IPC}{Inter Process Communication}

\nomenclature{CPU}{Central Processing Unit}

\nomenclature{GPU}{Graphical Processing Unit.\\}

\nomenclature{Kernel}{Núcleo principal de un sistema operativo moderno.}

\nomenclature{Linux}{Proyecto de software libre para producir un núcleo de sistema operativo basado en Unix}

\nomenclature{HAL}{Hardware Abstraction Layer}

\nomenclature{DVM}{Dalvik Virtual Machine}

\nomenclature{JVM}{Java Virtual Machine}

\nomenclature{Java}{Lenguaje de programación orientado a objetos}
