
\chapter{Introducción}

\label{introduccion}

Hace apenas una década, en que los teléfonos móviles pasaron de ser
simples aparatos de comunicación, a ser dispositivos de información
que sea han extendido en todo el mundo como el dispositivo electrónico
más utilizado de manera diaria. Desde la aparición de los teléfonos
móviles modernos (\emph{Smartphones}, teléfonos inteligentes) con
pantalla táctil, la convergencia en el uso de teléfonos móviles, las
redes de datos e Internet ha ido aumentando gradualmente \cite{fling2009mobile}.

La capacidad y prestaciones de los teléfonos móviles modernos han
mejorado drásticamente, ya que fusiona los aspectos de teléfonos móviles
y computadoras portátiles, la promesa esperada desde hace más de una
década \cite{Tanenbaum2010}. Muchas de las tareas asociadas comúnmente
a una computadora personal de escritorio pueden ser realizadas con
un teléfono móvil moderno, pero con la ventaja adicional de que el
móvil está más próximo a las actividades diarias del portador.

Desde los teléfonos móviles modernos se tiene acceso a la información
pero en un contexto móvil. El contexto puede darse de dos formas,
la primera forma es el valor que los usuarios mismos generan a partir
de las circunstancias en la que están involucrados \cite{fling2009mobile}.
La información provee un contexto que permite al usuario entender
mejor el momento por el que está pasando. La segunda forma del contexto
es el entorno en el cual el usuario realiza una acción determinada
\cite{fling2009mobile}. El contexto físico de un usuario es la ubicación
y la actividad física. 

El reconocimiento de actividades humanas (HAR\nomenclature{HAR}{Reconocimiento de Actividades Humanas, por sus siglas en inglés},
\emph{Human Activity Recognition}) es un línea de investigación amplia
que busca diseñar algoritmos que detecten el contexto de uno o más
individuos a partir de datos ambiguos de su entorno \cite{Bao2004}.
El proceso de reconocimiento puede realizarse utilizando datos obtenidos
de sensores externos o bien sensores puestos en el individuo. Los
teléfonos móviles pueden ser utilizados como sensores de bolsillo
donde diseñar aplicaciones móviles contextuales que detecten las actividades
físicas de un individuo en su vida diaria\cite{LaraLabrador2013}. 

En el ámbito de los teléfonos móviles existen bastantes usos al reconocer
las actividades básicas ambulatorias, de si el individuo camina, corre,
está quieto o moviéndose rápidamente en algún vehículo \cite{campuzano2015}
\cite{googlio2013}. Actualmente, existen ya varias implementaciones
de fondos privados como las que han hecho los fabricantes \emph{Google},
\emph{Sony} y \emph{Apple} que son los pioneros en la tendencia de
crear aplicaciones ricas en contexto.

Desde el año 2013 que \emph{Google} viene proveyendo como parte de
su plataforma \emph{Android} \cite{google2005and}\nomenclature{Android}{Android es un sistema operativo móvil desarrollado por Google. Está basado en el kernel de Linux.}
un producto denominado \emph{Google Play Services} \cite{googl2016loc},
que es un conjunto de librerías de desarrolladores que posee dentro
una implementación de reconocimiento de actividades. Además, \emph{Sony}
ha lanzado un producto y librerías para desarrolladores llamado \emph{Lifelog}
\cite{sony2016act} que es parte de una suite completa para mantenerse
en forma que también posee reconocedor de actividades. También, \emph{Apple}
en su plataforma \emph{iOS} \cite{apple2007ios}\nomenclature{iOS}{Es un sistema operativo móvil creado y desarrollado por Apple Inc.}
ha dispuesto recientemente una librería para desarrolladores llamada
\emph{HealthKit} \cite{healthkit2016}, de características similares
a los productos antes mencionados. 

Las plataformas de desarrollo móvil de \emph{Android} e \emph{iOS}
e\emph{ }proveen un ecosistema abierto para crear aplicaciones de
contexto y de medios digitales masivos. Estas plataformas son las
que conducen la expansión y crecimiento de las aplicaciones móviles
e inalámbricas \cite{Tanenbaum2010}. Por esta razón es primordial
contar iniciativas abiertas de implementaciones de reconocimiento
de actividades para lo que este trabajo está abocado.

Por último, el teléfono móvil moderno está conectado constantemente
Internet, por lo cual ha popularizando el acceso a la Web\footnote{World Wide Web}.
Como la computación móvil ha crecido rápidamente, también lo ha hecho
el acceso a la Web (cfr. \cite{nyt2008iph}), esto abre las posibilidades
para la colaboración masiva de individuos que formen parte como sujetos
de estudio en el área de reconocimiento de actividades.

\section{Planteamiento del problema}

\label{planteamiento}

El reconocimiento de actividades humanas es un tópico de investigación
que data desde los años 90 y que al día de hoy sigue atrayendo interés
ya que es una línea amplia que abarca varias áreas. Por mencionar
algunas, desde la computación ubicua y móvil, la computación contextual,
la seguridad basada en la vigilancia y la vivienda asistida por el
ambiente \cite{chen2012sensor}. 

Reconocer las actividades físicas consiste en detectar los movimientos
que realiza un individuo o un conjunto de individuos utilizando sensores
internos o externos. Las actividades más básicas serían las ambulatorias,
que son cuando una persona está caminando, corriendo, de pie o sentada
y las de transporte, cuando una persona está en un vehículo en movimiento.

La técnica de reconocimiento consiste en recolectar datos de movimientos
de un conjunto de individuos, donde cada muestra inicial debe ser
etiquetada con la actividad identificada. Luego se procesan los datos
de manera a extraer información relevante para construir un modelo
de inferencia. El resultado es un modelo entrenado con un algoritmo
que detecta con cierta precisión las mismas actividades etiquetadas
teniendo como entrada cualquier conjunto de datos no etiquetados \cite{Bao2004}.

El área de investigación ha crecido con avances sostenidos debido
al acompañamiento de las mejoras tecnológicas de sensores de bajo
costo, las redes inalámbricas de alta velocidad y los teléfonos móviles
de características sin precedentes \cite{chen2012sensor}. A pesar
de los avances en el tópico y las tecnologías, aún siguen latentes
algunos principales desafíos que se presentan al construir sistemas
de reconocimiento de actividades basados en sensores como son: la
selección de las métricas de estudio, la utilización de un recolector
de datos diminuto, los métodos de inferencia y extracción de muestras
significativas, la recolección de datos en condiciones realistas,
la flexibilidad para soportar nuevos usuarios sin re-entrenar el modelo
y la implementación del sistema en dispositivos móviles bajo restricciones
de potencia y procesamiento \cite{LaraLabrador2013}. 

De estas seis cuestiones citadas ponemos énfasis en este trabajo en
cuatro que definen el problema a resolver utilizando teléfonos móviles
como principal herramienta.

\section{Justificación}

\label{justificaciuxf3n}

Los sistemas de reconocimiento de actividades utilizan uno o más sensores
como entrada de datos. Los teléfonos móviles modernos poseen múltiples
sensores, el tamaño y las condiciones adecuadas para la captura de
datos ya que las personas interactúan con estos dispositivos de manera
ubicua. 

Las métricas de estudio son los datos brutos de los sensores de movimiento
del teléfono, el sensor de aceleración\footnote{acelerómetro}, el
más común en todos los aparatos. El sensor de aceleración puede medir
el movimiento en dos o tres ejes, detectar la orientación del dispositivo
y provee información crucial para el reconocimiento de actividades
con un bajo consumo de energía. 

Además, los teléfonos móviles modernos tienen gran capacidad de almacenamiento,
procesamiento y conectividad por lo que son plataformas ideales para
construir sistemas de reconocimiento instantáneos, sin comunicación
con un agente o sistema externo de procesamiento de fondo. 

La combinación de gran capacidad, tamaño diminuto y bajo costo lleva
a las personas a interactuar y usar comúnmente estos dispositivos
en cualquier lugar. Esto aporta al estudio ya que el conjunto de individuos
que formarían la muestra para recolección de datos en condiciones
realistas, no tendrían la naturaleza de muestras de laboratorio.

Esto genera información contextual que hace posible realizar minería
de datos para reconocer actividades para diversos tipos de aplicaciones.
Ej. en medicina, seguridad, entretenimiento o de uso militar, etc.\cite{LaraLabrador2013}.
Los sensores disponibles en un teléfono móvil inteligente en la actualidad
son variados e incluyen: GPS (localización), micrófonos, cámaras,
luz, temperatura, barómetro, dirección (brújula) y aceleración. Existen
otros más variados dependiendo del modelo, el fabricante y los accesorios.
.

Existe bastante información en el estado del arte de esta línea de
investigación sobre técnicas de reconocimiento de actividades, los
métodos de captura y el procesamiento de datos de sensores \cite{LaraLabrador2012},
\cite{Kwapisz2011}. Sin embargo, a pesar de estar bien definida en
la literatura la arquitectura de un sistema de recolección, existe
una necesidad de un componente de librerías para teléfonos móviles
que pueda ser libremente utilizado sin depender de definiciones de
API privadas (Application Programming Interfaces), servicios de en
nube (Software as a Service), o aplicaciones de terceros en Tiendas
(Ej. Google Play Services).

\section{Alcance y Objetivos}

\label{alcance-y-objetivos}

Esta propuesta centra en el estudio del reconocimiento de actividades
humanas utilizando sensores en teléfonos móviles inteligentes, de
manera a aportar una librería de reconocimiento para estos dispositivos.
Durante las pruebas experimentales y la recolección de los datos se
desarrollará una aplicación móvil que demuestre la efectividad de
la librería y evaluará la técnica de reconocimiento sencilla de aprendizaje
automático.

Inicialmente se realizará una revisión del estado del arte con el
objetivo de entender y comprender los métodos de reconocimientos de
actividades actualmente empleados.

Luego de ello se implementaría la librería propuesta con una API y
un servicio Android que atienda a los pedidos de reconocimiento de
actividades estándar. La librería además tendrá incluida la utilización
y actualización del modelo colaborativo para intentar mejorar las
predicciones de las actividades.

Además se implementará un servicio web de captura de las muestras
de la librería y actualice el modelo de predicción para futuras versiones
de la librería. El modelo de aprendizaje se basará en aprendizaje
automático con árboles de decisión (Decision Trees), algoritmo de
clasificación C4.5. Para terminar, se creará una aplicación móvil
de prueba para recolectar muestras y evaluar el resultado de predicción
utilizando la librería de reconocimiento propuesta

\subsection{Objetivo General}

\label{objetivo-general}

Implementar un sistema de reconocimiento de actividades humanas con
teléfonos móviles que aporte componentes de librería y arquitectura
de sistema abiertos.

\subsection{Objetivos Específicos}

\label{objetivos-especuxedficos}
\begin{itemize}
\item Explorar el estado del arte sobre reconocimiento de actividades humanas
(HAR). 
\item Comprender las técnicas de recolección de datos para aprendizaje en
línea en entornos restringidos para bajo consumo de batería. 
\item Comprender el procesamiento de señales en datos de sensores de aceleración
para identificar muestras de ensayo de aprendizaje automático. 
\item Entender las técnicas de clasificación por aprendizaje automático
más apropiados en entornos restringidos para bajo consumo de batería. 
\item Diseñar la arquitectura de un sistema de reconocimiento para la recolección
de datos de manera colaborativa y predicción de actividades humanas
en-linea. 
\item Aportar un componente de librería de código abierto apropiado para
teléfonos móviles con restricciones de bajo consumo de batería. 
\end{itemize}

\section{Organización del trabajo}

\label{organizaciuxf3n-del-trabajo}

El trabajo está organizado como sigue: en el Capítulo 2 se presenta
el reconocimiento de actividades, sus características, dificultades,
y los métodos abordados para el realizar esta reconocimiento.

El Capítulo 3 introduce el aprendizaje automático, los tipos de aprendizajes,
y las distintas técnicas o enfoques aplicados, y sobre todo sobre
los arboles de decisión. Estos capítulos representan el estado del
arte de este trabajo.

El Capítulo 4 se presenta el reconocimiento en dispositivos moviles.
Bla bla.. En el Capítulo 5 se presenta el modelo y la implementación
del reconocedor de actividades colaborativo, su diseño y arquitectura,
y tecnologías utilizadas para su desarrollo.

En el Capítulo 6 se presentan los experimentos realizados, y en el
capitulo 7 los resultados obtenidos. En el Capítulo 8 presentan las
conclusiones de este trabajo y los posibles trabajos futuros resultado
de este trabajo final de grado. 
