
\chapter{Introducción}

\label{chap:introduccion}

Hace apenas una década, en que los teléfonos móviles pasaron de ser
simples aparatos de comunicación, a ser dispositivos de información
que sea han extendido en todo el mundo como el dispositivo electrónico
más utilizado de manera diaria. Desde la aparición de los teléfonos
móviles modernos (\emph{Smartphones}, o teléfonos inteligentes) con
pantalla táctil, la convergencia en el uso de teléfonos móviles, las
redes de datos e Internet ha ido aumentando gradualmente \cite{Fling2009}.

La capacidad y prestaciones de los teléfonos móviles modernos han
mejorado drásticamente, ya que fusiona los aspectos de teléfonos móviles
y computadoras portátiles, la promesa esperada desde hace más de una
década \cite{Tanenbaum2010}. Muchas de las tareas asociadas comúnmente
a una computadora personal de escritorio pueden ser realizadas con
un teléfono móvil moderno, pero con la ventaja adicional de que el
móvil está más próximo a las actividades diarias del portador.

Desde los teléfonos móviles modernos se tiene acceso a la información
pero en un contexto móvil. El contexto puede darse de dos formas,
la primera forma es el valor que los usuarios mismos generan a partir
de las circunstancias en la que están involucrados \cite{Fling2009}.
La información provee un contexto que permite al usuario entender
mejor el momento por el que está pasando. La segunda forma del contexto
es el entorno en el cual el usuario realiza una acción determinada
\cite{Fling2009}. El contexto físico de un usuario es su ubicación
y su actividad física. 

El reconocimiento de actividades humanas (\abbr{HAR}, \emph{Human
Activity Recognition}) es un línea de investigación amplia que busca
diseñar algoritmos que detecten el contexto de uno o más individuos
a partir de datos ambiguos de su entorno \cite{Bao2004}. El proceso
de reconocimiento puede realizarse utilizando datos obtenidos de sensores
externos o bien sensores de atuendo puestos en el individuo \cite{LaraLabrador2013}.
Los teléfonos móviles son utilizados como sensores de bolsillo con
aplicaciones móviles contextuales diseñados para detectar las actividades
físicas de un individuo en su vida diaria. 

En el ámbito de los teléfonos móviles existen bastantes usos al reconocer
las actividades básicas ambulatorias, de si el individuo camina, corre,
está quieto o moviéndose rápidamente en algún vehículo \cite{CampuzanoLopez2015,Google2013l}.
Actualmente, existen ya varias implementaciones de fondos privados
como las que han hecho los fabricantes \emph{Google}, \emph{Sony}
y \emph{Apple} que son los pioneros en la tendencia de crear aplicaciones
ricas en contexto.

Desde el año 2013 que \emph{Google} viene proveyendo como parte de
su plataforma \emph{\abbr{Android} }\cite{Google2005a} un producto
denominado \emph{Google Play Services} \cite{Google2016l}, que es
un conjunto de librerías de desarrolladores que dispone una implementación
de reconocimiento de actividades. Además, \emph{Sony} ha lanzado un
producto y librerías para desarrolladores llamado \emph{Lifelog} \cite{Sony2016l}
que es parte de una suite completa para mantenerse en forma donde
uno de los componentes es un reconocedor de actividades físicas. También,
\emph{Apple} en su plataforma \abbr{iOS} \cite{Apple2007i} ha dispuesto
recientemente una librería para desarrolladores llamada \emph{HealthKit}
\cite{Apple2016h}, de características similares a los productos antes
mencionados. 

Las plataformas de desarrollo móvil de \emph{Android} e \emph{iOS
}proveen un ecosistema abierto para crear aplicaciones personalizadas
de contexto y de medios digitales. Estas plataformas son las que conducen
la expansión y crecimiento de las aplicaciones móviles e inalámbricas
\cite{Tanenbaum2010}. Sin embargo, no existen iniciativas abiertas
que aporten al desarrollo del reconocimiento de actividades.

Por último, el teléfono móvil moderno está conectado constantemente
Internet, por lo cual ha popularizando el acceso a la Web\footnote{World Wide Web}.
Como la computación móvil ha crecido rápidamente, también lo ha hecho
el acceso a la Web (cfr. \cite{NYTimes2008iph}), promoviendo la colaboración
masiva de parte de individuos como sujetos de estudio en el área de
reconocimiento de actividades físicas.

\section{Planteamiento del problema}

\label{planteamiento}

El reconocimiento de actividades humanas es un tópico de investigación
que data desde los años 90 y que al día de hoy sigue atrayendo interés
ya que abarca varias áreas de estudio. Por mencionar algunas, desde
la computación ubicua y móvil, la computación contextual, la seguridad
basada en la vigilancia, la vivienda asistida por el ambiente y otras
\cite{chen2012sensor}. 

Reconocer las actividades físicas consiste en detectar los movimientos
que realiza un individuo o un conjunto de individuos utilizando sensores
internos o externos. Las actividades reconocidas más simples son las
ambulatorias, cuando una persona está caminando, corriendo, de pie
o sentada y las de transporte, cuando una persona está en un vehículo
en movimiento. También existen otras más complejas como las posturas,
de entrenamiento físico y otras más comunes como cepillarse los dientes,
mirar televisión, comer, tomar liquidos, hablar por teléfono, etc.
\cite{LaraLabrador2013}.

El problema del reconocimiento de actividades es abordado utilizando
la técnica de aprendizaje automático supervisado que se basa en una
metodología de inferencia dividida en dos etapas: entrenamiento y
evaluación \cite{LaraLabrador2013,Kwapisz2011}. En resumen, la metodología
consiste en recolectar datos de un conjunto de individuos, donde cada
muestra debe ser etiquetada con la actividad identificada de forma
supervisada. Luego extraer variables con información relevante para
construir un modelo de inferencia entrenado. Finalmente, el resultado
es un algoritmo que predice con cierta precisión las actividades etiquetadas
teniendo como entrada cualquier conjunto de datos no etiquetados \cite{Bao2004}.

Esta área de investigación ha avanzado sostenidamente debido a las
mejoras tecnológicas de sensores de bajo costo, las redes inalámbricas
de alta velocidad y los teléfonos móviles de características sin precedentes\cite{chen2012sensor}.
A pesar de los avances en el área y las tecnologías, aún siguen latentes
algunos principales desafíos que se presentan al construir sistemas
de reconocimiento de actividades basados en sensores como son: la
selección de las métricas de estudio, la utilización de un recolector
de datos diminuto, los métodos de inferencia y extracción de muestras
significativas, la recolección de datos en condiciones realistas,
la flexibilidad para soportar nuevos usuarios sin volver a entrenar
el modelo y la implementación del sistema en dispositivos móviles
bajo restricciones de potencia y procesamiento\cite{LaraLabrador2013}. 

De estas seis cuestiones citadas ponemos énfasis en cuatro puntos
que definen el problema a resolver y donde a continuación se describe
la justificación de este trabajo utilizando los teléfonos móviles
como principal herramienta.

\section{Justificación}

\label{justificaciuxf3n}

Para construir sistemas de reconocimiento de actividades se necesitan
uno o más sensores para capturar los datos del entorno del individuo.
Los teléfonos móviles modernos poseen múltiples sensores, el tamaño
y las condiciones adecuadas para la captura de datos de manera ubicua.
Algunos de los sensores comunes en un teléfono inteligente son: localización
(\hyperlink{abbr}{GPS}, \emph{Global Positioning System}), brújula
y aceleración, como también micrófonos, cámaras, luminosidad, temperatura,
barómetro y otros más variados dependiendo del modelo, el fabricante
y los accesorios del teléfono\cite{Kwapisz2011}.

El sensor de aceleración\footnote{acelerómetro} mide principalmente
el movimiento en dos o tres ejes y puede ser utilizado para detectar
la orientación del dispositivo, estos datos proveen información crucial
para el reconocimiento de actividades con un bajo consumo de energía.
Adicionalmente, los teléfonos inteligentes son de gran capacidad de
almacenamiento, procesamiento y conectividad de manera que son plataformas
ideales para construir sistemas de reconocimiento autónomos.

La combinación de gran capacidad, tamaño diminuto y bajo costo hace
que los dispositivos sean portados a cualquier lugar por las personas,
permitiendo recolectar datos en condiciones realistas, completamente
ajenas a obtener muestras supervisadas en un laboratorio\cite{Bao2004}. 

Existe bastante información en el estado del arte de esta línea de
investigación sobre técnicas de reconocimiento de actividades, los
métodos de captura y el procesamiento de datos de sensores \cite{LaraLabrador2012,Kwapisz2011}.
Sin embargo, a pesar de estar bien definida en la literatura acerca
de la arquitectura de un sistema de reconocimiento, aún existe un
hueco en contar con componente de software para teléfonos móviles
que pueda ser utilizado libremente sin depender de librerías privativas
(Ej. \emph{Google Play Services} API) (\abbr{API}, \emph{Application
Programming Interface}), servicios de Internet (\abbr{SaaS}, \emph{Software
as a Service}), o aplicaciones de terceros (Ej. \emph{Lifelog}).

El modelo de desarrollo de software libre hace posible tener un enfoque
colaborativo en dos dimensiones, por un lado el software en sí, la
librería del sistema de reconocimiento, y por otro el conjunto de
datos de muestras utilizadas para el entrenamiento del reconocedor
de actividades.

\section{Alcance y Objetivos}

\label{alcance-y-objetivos}

Este trabajo estudia el reconocimiento de actividades humanas utilizando
sensores en teléfonos móviles inteligentes, de manera a aportar un
sistema de reconocimiento en forma de librería abierta y libremente
distribuida. Durante la recolección de datos y las pruebas se desarrollará
adicionalmente una aplicación móvil que demuestre la efectividad de
la librería como componente de software independiente. También se
evaluará el sistema de reconocimiento de actividades basado en aprendizaje
automático.

Se incluye la revisión del estado del arte de la materia con el objetivo
de entender y comprender las metodologías de reconocimiento actualmente
empleadas. Luego, se construirá una librería cliente y un servicio
basado en \emph{\abbr{Android}} que atienda a las llamadas de procedimiento
del sistema de reconocimiento de actividades. 

Dentro del sistema de reconocimiento se incluirá un modelo colaborativo
capaz de actualizarse constantemente, a nuestro criterio, esto mejorará
las predicciones basándose en un conjunto de datos de entrenamiento
proveniente de las personas que colaboren de forma anónima y en condiciones
reales.

Además se implementará un servicio web de captura de las muestras
de la librería y actualice el modelo de predicción para futuras versiones
de la librería. El modelo de aprendizaje se basará en aprendizaje
automático por árboles de decisión (\abbr{DT}, \emph{Decision Trees}).
Finalmente, para evaluar la metodología se creará una aplicación móvil
para recolectar muestras y evaluar el resultado de predicción utilizando
el sistema de reconocimiento propuesto.

\subsection{Objetivo General}

\label{objetivo-general}

Implementar un sistema de reconocimiento de actividades humanas con
teléfonos móviles cuyo principal aporte sea un servicio y una librería
libremente distribuidos.

\subsection{Objetivos Específicos}

\label{objetivos-especuxedficos}
\begin{enumerate}
\item \label{enu:obe1}Definir el estado del arte sobre reconocimiento de
actividades humanas (\abbr{HAR}). 
\item \label{enu:obe2}Comprender las técnicas de recolección de datos en
entornos restringidos para bajo consumo energía. 
\item \label{enu:obe3}Comprender el procesamiento de señales de datos de
aceleración inercial para identificar variables significativas de
entrenamiento. 
\item \label{enu:obe4}Comprender la clasificación por aprendizaje automático
en entornos restringidos para bajo consumo energía. 
\item \label{enu:obe5}Diseñar un sistema de reconocimiento de actividades
que comprenda la recolección de muestras de manera colaborativa y
predicción de actividades en-línea. 
\item \label{enu:obe6}Aportar un componente de software de código abierto
para uso en teléfonos móviles inteligentes. 
\end{enumerate}

\section{Organización del trabajo}

\label{organizaciuxf3n-del-trabajo}

El trabajo está organizado en la siguiente manera: en el Capítulo
2 se presenta define el marco teórico del área describiendo los conceptos
principales del reconocimiento de actividades humanas, sus características,
dificultades, y las metodologías existentes. El Capítulo 3 definimos
el aprendizaje automático, los tipos de aprendizajes, y las distintas
técnicas o enfoques aplicados, dando énfasis en los arboles de decisión.
Estos capítulos representan el estado del arte de este trabajo.

El Capítulo 4 se presenta en detalle la construcción de un sistema
de reconocimiento para teléfonos móviles. Continuando, en el Capítulo
5 se discute el modelo y la implementación de un reconocedor de actividades
colaborativo, su diseño, arquitectura, y tecnologías utilizadas.

En el Capítulo 6 se detallan los protocolos de los experimentos realizados,
y en el Capítulo 7 los resultados obtenidos durante el estudio. Para
finalizar, en el Capítulo 8 se incluyen las conclusiones y los posibles
trabajos futuros. 
