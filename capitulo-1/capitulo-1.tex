
\chapter{Introducción}

\label{chap1:introduccion}

Hace apenas una década, que los teléfonos móviles pasaron de ser simples
aparatos de comunicación a ser los dispositivos de información más
utilizados de manera diaria. Desde la aparición de los teléfonos móviles
modernos (\abbr{Smartphones}) con pantalla táctil, la convergencia
en el uso de teléfonos móviles, las redes de datos e Internet ha ido
aumentando gradualmente \cite{Fling2009}.

Desde su aparición, la capacidad y prestaciones de los teléfonos móviles
modernos han mejorado drásticamente, ya que fusiona los aspectos de
teléfonos móviles y computadoras portátiles, la promesa esperada desde
hace más de una década \cite{Tanenbaum2010}. Muchas de las tareas
asociadas comúnmente a una computadora personal de escritorio pueden
ser realizadas con un teléfono móvil moderno, pero con la ventaja
adicional de que el móvil está más próximo a las actividades diarias
del portador.

Desde los teléfonos móviles modernos se tiene acceso a la información
pero en un contexto móvil. El contexto puede darse de dos formas,
la primera forma es el valor que los usuarios mismos generan a partir
de las circunstancias en la que están involucrados. La información
provee un contexto que permite al usuario entender mejor el momento
que está experimentando. Por otro lado, el contexto también es determinado
por la acción que el usuario realiza en su entorno \cite{Fling2009}.
En este último caso, el contexto físico de un usuario involucra a
su ubicación y su actividad humana. 

El reconocimiento de actividades humanas (\emph{Human Activity Recognition},
\abbr{HAR}) es un tópico de investigación que busca diseñar algoritmos
que provean información acerca del contexto de uno o más individuos
a partir de datos ambiguos capturados en su entorno \cite{Bao2004}.
Reconocer el contexto es un componente primordial de los sistemas
inteligentes y cognitivos, este forma parte de un proceso más complejo,
donde contribuye en las etapas de análisis y captura de datos a través
de sensores externos o adjuntos a los individuos \cite{ReyesOrtiz2015,Chen2012}.
Los avances recientes en las tecnologías de computación móvil y sensores
tales como: miniaturización, bajo consumo, buenas prestaciones, buena
conectividad y procesamiento de datos, hizo que proliferara el uso
de los teléfonos móviles modernos con sistemas inteligentes que monitorizan
las acciones del usuario en su vida diaria. 

Diseñar sistemas móviles inteligentes, o aplicaciones móviles de contexto,
que reconozcan las actividades de un individuo, Ej. Si está caminando,
está corriendo, está quieto o moviéndose en algún vehículo tiene diversos
motivos en la actualidad \cite{CampuzanoLopez2015,Google2013l}. Tal
es la motivación, que ha aumentado la popularidad de las aplicaciones
móviles de contexto con funcionalidades en el ámbito del cuidado personal,
la movilidad y la asistencia en la vida diaria que requieren de estas
capacidades. Por citar ejemplos, en la actualidad existen diversas
alternativas en el mercado, como las proveídas por las grandes compañías
\emph{Google}, \emph{Sony} y \emph{Apple}, que son principales fabricantes
de teléfonos móviles e impulsores aplicaciones móviles basadas en
contexto.

Desde el año 2013, \emph{Google} ha proveído para la plataforma \emph{\abbr{Android}
}\cite{Google2005a} el producto \emph{Google Play Services}, este
dispone de una librería para el reconocimiento de actividades humanas
exclusiva para desarrolladores de aplicaciones móviles \cite{Google2013l}.
Saliendo al paso, \emph{Sony} ha lanzado un conjunto completo de productos
(pulseras y relojes) para mantenerse en forma y están acompañados
de la aplicación \emph{Lifelog} \cite{Sony2016l}. Esta aplicación
es capaz de reconocer actividades físicas y registrar las acciones
del usuario simplemente portando el teléfono. Además, \emph{Apple}
para su plataforma\emph{ \abbr{iOS}} \cite{Apple2007i} ha dispuesto
recientemente una librería para desarrolladores llamada \emph{HealthKit}
\cite{Apple2016h}, con un enfoque similar a los productos anteriormente
mencionados. 

Las plataformas para aplicaciones \emph{\abbr{Android}} e \emph{\abbr{iOS}}
proveen un ecosistema abierto para crear aplicaciones personalizadas,
estas conducen a la expansión y el crecimiento de aplicaciones contextuales
\cite{Tanenbaum2010}. Sin embargo, a pesar de promoverse plataformas
abiertas, existe aún una carencia de proyectos colaborativos de código
abierto que contribuyan al desarrollo de sistemas inteligentes de
contexto, a excepción de ciertos precedentes como \cite{Kwapisz2011,LaraLabrador2013}
u otras iniciativas como \cite{FUNF2016} y \cite{SensingKit2016}
más orientados a la colaboración.

Desde que los teléfonos móviles modernos están conectados a Internet
el acceso a la Web\footnote{World Wide Web} se ha popularizando.
Esto impulsó a que la computación móvil haya crecido enormemente,
y también a que haya aumentado el uso de la Web (cfr. \cite{NYTimes2008iph}).
Esto fomenta un ámbito de trabajo colaborativo, distribuido y móvil
donde un grupo masivo de personas pueden aportar a un propósito común,
ya sea como sujetos de estudio en el reconocimiento de actividades
humanas o en el desarrollo de herramientas para aplicaciones móviles
contextuales.

\section{Planteamiento del problema}

\label{sec11:planteamiento}El reconocimiento de actividades humanas
es un tópico de investigación que data desde los años noventa, y en
la actualidad sigue atrayendo bastante interés ya que abarca varias
áreas de estudio orientadas al desarrollo de aplicaciones novedosas.
Algunas de estas áreas son la computación ubicua, la computación móvil,
y la computación contextual; también la seguridad por vigilancia,
las viviendas asistidas y los ambientes inteligentes, entre otros.\cite{Chen2012}. 

Reconocer actividades humanas consiste en comprender las acciones
e interacciones de las personas con su entorno a través de la integración
de sensores y el razonamiento \cite{ReyesOrtiz2015}. Las actividades
humanas se pueden catalogar en diferentes tipos y según el nivel de
detalle \cite{Chen2012}. Poniendo énfasis en las actividades básicas
que un individuo puede realizar, podemos destacar las acciones físicas
más simples, como lo son las actividades ambulatorias y de transporte.
Además, existen otras acciones más complejas como la postura, los
movimientos en ejercicio aeróbico u otras más comunes, Ej. Cepillarse
los dientes, mirar televisión, comer, beber, hablar por teléfono,
etc. \cite{LaraLabrador2013}.

El problema del reconocimiento de actividades humanas conlleva un
proceso complejo que se resume en la utilización de técnicas de aprendizaje
automático (\emph{Machine Learning}, \abbr{ML}), en donde se utiliza
un modelo de inferencia construido en dos etapas: entrenamiento y
evaluación \cite{LaraLabrador2013,Kwapisz2011}. El proceso se puede
resumir en un flujo iterativo que consiste en \cite{Bao2004}:
\begin{itemize}
\item Recolectar datos etiquetados de un conjunto de individuos. 
\item Extraer variables con atributos relevantes al estudio.
\item Construir un modelo de predicción de actividades humanas.
\item Detectar con cierta precisión el conjunto de actividades previamente
etiquetadas utilizando como entrada un conjunto de datos no etiquetados.
\item Retroalimentar y/o evaluar el modelo para obtener una mejora continua.
\end{itemize}
La investigación en el área de estudio de los sistemas \abbr{HAR}
ha progresado de manera sostenida debido a avances tecnológicos de
sensores de bajo costo, redes inalámbricas de alta velocidad y dispositivos
móviles inteligentes \cite{Chen2012}. A pesar de estos avances, aún
existen importantes desafíos por resolver como: seleccionar medidas
relevantes, recolectar datos en forma no invasiva, analizar métodos
de inferencia alternativos y extraer muestras significativas \cite{LaraLabrador2013}.
Además, a los anteriores también se suman restricciones como recolectar
datos en condiciones realistas, flexibilidad para soportar nuevos
individuos sin necesidad entrenar el modelo permanentemente y condiciones
especiales de ahorro de energía \cite{ReyesOrtiz2015}. 

\section{Justificación}

\label{sec12:justificaciuxf3n}Este trabajo se propone hacer énfasis
en cuatro puntos que enmarcan el problema de reconocimiento de actividades
humanas mediante la utilización de teléfonos móviles modernos como
plataforma desarrollo.

Construir un sistema \abbr{HAR} requiere uno o más sensores que observen
oportunamente a sus usuarios. Los teléfonos móviles modernos poseen
múltiples sensores disponibles para capturar localización (\abbr{GPS}),
orientación, movimiento, sonido, vídeo, luminosidad, temperatura,
presión entre otros \cite{Kwapisz2011}.

El sensor de aceleración\footnote{acelerómetro} es el que mide el
movimiento en varias dimensiones y en combinación con otros puede
detectar la orientación. Las señales del movimiento proveen información
crucial para el reconocimiento de actividades humanas con bajo consumo
de energía.

Adicionalmente, los teléfonos modernos poseen capacidades de almacenamiento,
procesamiento y buena conectividad en un tamaño compacto, haciendolos
instrumentos ideales para construir sistemas autónomos. El tamaño
diminuto promueve que estos dispositivos sean portados en cualquier
momento, permitiendo recolectar información de manera ubicua y bajo
condiciones realistas, esto es una ventaja en comparación a obtener
muestras bajo supervisión en un laboratorio \cite{Bao2004}. 

El cuerpo de conocimiento acerca de los sistemas \abbr{HAR} es bastante
extenso, se compone de técnicas de reconocimiento, métodos de recolección
y procesamiento de señales \cite{LaraLabrador2012,Kwapisz2011}. Sin
embargo, a pesar de la extensa fuente bibliográfica, aún es escasa
la disponibilidad del componente de software abierto que pueda ser
utilizado y extendido sin depender de librerías privativas (\emph{Application
Programming Interfac}e, \abbr{API}), como \emph{Google Play Services}
\cite{Google2016l} o aplicaciones de terceros como \emph{Sony Lifelog}
\cite{Sony2016l}.

Cualquier iniciativa de software libre hace posible un enfoque colaborativo,
por un lado la colaboración al construir un componente de software,
la disponibilidad de utilizar el mismo, como también compartir datos
de entrenamiento y evaluación del modelo de predicción para una mejora
continua.

\section{Alcance y Objetivos}

\label{sec13:alcance-y-objetivos}Este trabajo se centra en el reconocimiento
de actividades humanas (\abbr{HAR}) con teléfonos móviles modernos
(\abbr{Smartphones}) utilizando sensores que busca aportar un componente
de software en forma de librería de código abierto que sea de libre
distribución. 

Se abarca la revisión del marco teórico de los sistemas \abbr{HAR}
con el objetivo de exponer las metodologías empleadas en la actualidad.
El contenido principal del trabajo se compone de la construcción y
diseño de una librería cliente y un servicio de reconocimiento de
actividades para la plataforma \emph{\abbr{Android}}. 

Dentro de la implementación del sistema \abbr{HAR} se incluirá un
modelo colaborativo capaz de actualizarse constantemente, de forma
a mejorar las predicciones y basándose en un conjunto de datos de
entrenamiento continuo proveniente de las personas que colaboren de
forma anónima y en condiciones reales. Además se implementará un servicio
en Internet de captura de muestras y predicciones para evaluar y/o
actualizar el modelo de predicción y publicar nuevas versiones. 

El modelo de aprendizaje automático se basará en la técnica por árboles
de decisión (\emph{Decision Tree}, \abbr{DT}). Durante las fases
de entrenamiento y evaluación se desarrollará una aplicación móvil
para recolectar muestras y evaluar el resultado de predicción utilizando
el sistema de reconocimiento propuesto. Esto permitirá demostrar la
efectividad de la librería como componente de software independiente
y evaluar los resultados de reconocimiento de actividades en comparación
con otras soluciones equivalentes. 

\subsection{Objetivo General}

\label{sec13:objetivo-general}El objetivo principal del presente
trabajo es implementar un sistema de reconocimiento de actividades
humanas con teléfonos móviles cuyo principal aporte sea un componente
de software en forma de librería de código abierto y libremente distribuido.

\subsection{Objetivos Específicos}

\label{sec13:objetivos-especuxedficos}
\begin{enumerate}
\item \label{enu:obe1}Definir el marco teórico sobre el reconocimiento
de actividades humanas (\abbr{HAR}). 
\item \label{enu:obe2}Comprender las técnicas de recolección de datos en
entornos restringidos para bajo consumo energía. 
\item \label{enu:obe3}Comprender el procesamiento de señales de datos inerciales
para identificar variables significativas de entrenamiento. 
\item \label{enu:obe4}Comprender la clasificación por aprendizaje automático
en entornos restringidos para bajo consumo energía. 
\item \label{enu:obe5}Diseñar un sistema de reconocimiento de actividades
que se componga de la recolección de muestras colaborativas y predicción
de actividades humanas en-línea. 
\item \label{enu:obe6}Aportar un componente de software en forma de librería
para uso en teléfonos móviles modernos con la plataforma \abbr{Android}. 
\end{enumerate}

\section{Organización del Trabajo}

\label{sec14:organizaciuxf3n-del-trabajo}El trabajo está organizado
como sigue: en el Capítulo \ref{chap2:marco-teorico} se definen las
ideas generales relevantes al área de estudio para la creación de
sistemas \abbr{HAR} con miras a generalizar la perspectiva acerca
del problema de abordado. El Capítulo \ref{chap:Aprendizaje-Automatico}
examina los detalles acerca de las practicas sobre el aprendizaje
automático, los tipos de aprendizajes, las distintas técnicas o enfoques
aplicados, dando un énfasis especial en los arboles de decisión. 

El Capítulo \ref{chap4:sistemas-de-reconocimiento} explica en detalle
la arquitectura de los sistemas \abbr{HAR} para dispositivos móviles
junto con las consideraciones generales de diseño e implementación
tomadas en cuenta en este trabajo. En el Capítulo \ref{chap5:hardroid}
se detalla la propuesta de implementación de un reconocedor de actividades
humanas colaborativo y en linea utilizando la plataforma \abbr{Android}.

El Capítulo \ref{chap6:evaluacion} presentan los protocolos realizados
para obtener datos de entrenamiento para conducir los experimentos
de este trabajo, se incluyen las pruebas supervisadas y la validación
de los datos. También se expone la evaluación en base a los resultados
obtenidos durante las pruebas del sistema \abbr{HAR} propuesto: instalado
en teléfonos móviles y utilizado en condiciones reales. 

Para finalizar, en el Capítulo \ref{cap8:conclusiones-y-trabajos-futuros}
se presentan las conclusiones del presente trabajo así como también
se proponen futuros trabajos relacionados. 
