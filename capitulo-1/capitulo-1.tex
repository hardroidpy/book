
\chapter{Introducción}

\label{introduccion}

Hace apenas una década, en que los teléfonos móviles pasaron de ser
simples aparatos de comunicación, a ser dispositivos de información
que sea han extendido en todo el mundo como el dispositivo electrónico
más utilizado de manera diaria. Desde la aparición de los teléfonos
móviles modernos (\emph{Smartphones}, teléfonos inteligentes) con
pantalla táctil, la convergencia en el uso de teléfonos móviles, las
redes de datos e Internet ha ido aumentando gradualmente \cite{fling2009mobile}.

La capacidad y prestaciones de los teléfonos móviles modernos han
mejorado drásticamente, ya que fusiona los aspectos de teléfonos móviles
y computadoras portátiles, la promesa esperada desde hace más de una
década \cite{Tanenbaum2010}. Muchas de las tareas asociadas comúnmente
a una computadora personal de escritorio pueden ser realizadas con
un teléfono móvil moderno, pero con la ventaja adicional de que el
móvil está más próximo a las actividades diarias del portador.

Desde los teléfonos móviles modernos se tiene acceso a la información
pero en un contexto móvil. El contexto puede darse de dos formas,
la primera forma es el valor que los usuarios mismos generan a partir
de las circunstancias en la que están involucrados \cite{fling2009mobile}.
La información provee un contexto que permite al usuario entender
mejor el momento por el que está pasando. La segunda forma del contexto
es el entorno en el cual el usuario realiza una acción determinada
\cite{fling2009mobile}. El contexto físico de un individuo es su
ubicación y la actividad física. 

El reconocimiento de actividades humanas (HAR\nomenclature{HAR}{Reconocimiento de Actividades Humanas, por sus siglas en inglés},
\emph{Human Activity Recognition}) es un línea de investigación que
busca diseñar algoritmos que detecten el contexto del individuo a
partir de datos ambiguos del entorno \cite{Bao2004}. Diseñar aplicaciones
móviles contextuales que detecten las actividades físicas que un individuo
realiza en su vida diaria es un ejercicio práctico de esta línea de
investigación. 

En el ámbito de los teléfonos móviles existen bastantes usos al reconocer
las actividades básicas ambulatorias, de si el individuo camina, corre,
está quieto o moviéndose rápidamente en algún vehículo \cite{campuzano2015}
\cite{googlio2013}. Actualmente, existen varias implementaciones
privativas como lo han hecho los fabricantes \emph{Google}, \emph{Sony}
y \emph{Apple} que demuestran esta tendencia necesaria de proveer
aplicaciones ricas en contexto con sus respectivos productos \emph{Google
Play Services API} \cite{googl2016loc}, \emph{LifeLog API} \cite{sony2016act}
y \emph{HealthKit} \cite{healthkit2016}. 

Sin embargo, las plataformas como \emph{Android} \cite{google2005and}\nomenclature{Android}{Android es un sistema operativo móvil desarrollado por Google. Está basado en el kernel de Linux.}
e\emph{ iOS} \cite{apple2007ios}\nomenclature{iOS}{Es un sistema operativo móvil creado y desarrollado por Apple Inc.}
proveen un ecosistema abierto para crear aplicaciones de contexto
y de medios digitales masivos. Estas plataformas son las que conducen
la expansión y crecimiento de las aplicaciones móviles e inalámbricas
\cite{Tanenbaum2010}. También es primordial contar iniciativas abiertas
de implementaciones de reconocimiento de actividades para lo que este
trabajo está abocado.

Por último, el teléfono móvil moderno está conectado constantemente
Internet, por lo cual ha popularizando el acceso a la Web\footnote{World Wide Web}.
Como la computación móvil ha crecido rápidamente, también lo ha hecho
el acceso a la Web (cfr. \cite{nyt2008iph}) abriendo posibilidades
para la colaboración continua de individuos como sujetos de estudio
en el área de reconocimiento de actividades.

\section{Planteamiento del problema}

\label{planteamiento}

TODO: Explicar El problema

\section{Justificación\label{justificaciuxf3n}}

El reconocimiento de actividades humanas (Human Activity Recognition,
\nomenclature{HAR}{Reconocimiento de Actividades Humanas, por sus siglas en inglés})
es una línea de investigación de continuo desarrollo que desde hace
más de una década busca recolectar datos de las personas (de manera
ubicua) en interacción con su entorno para proveer información contextual
en su vida diaria \cite{Bao2004}. El ejemplo más común sería reconocer
actividades básicas ambulatorias, cuando una persona está caminando,
corriendo, de pie, sentada a través de algún tipo de sensor o cámara
disponible para dicho efecto.

Desde que empezaron a masificarse los teléfonos móviles inteligentes
con bajo costo, gran capacidad, con múltiples sensores y de adecuado
tamaño, las personas interactúan cada vez más con estos dispositivos.
Esto genera información contextual que hace posible realizar minería
de datos para reconocer actividades para diversos tipos de aplicaciones.
Ej. en medicina, seguridad, entretenimiento o de uso militar, etc.\cite{LaraLabrador2013}.

Los sensores disponibles en un teléfono móvil inteligente en la actualidad
son variados e incluyen: GPS (localización), micrófonos, cámaras,
luz, temperatura, barómetro, dirección (brújula) y aceleración. Existen
otros más variados dependiendo del modelo, el fabricante y los accesorios.
El acelerómetro es el sensor más común en estos dispositivos, puede
medir la aceleración en dos o tres ejes, detectar la orientación del
dispositivo y provee principalmente información crucial para el reconocimiento
de actividades.

Existe bastante información en el estado del arte de esta línea de
investigación sobre técnicas de reconocimiento de actividades, los
métodos de captura y el procesamiento de datos de sensores \cite{LaraLabrador2012},
\cite{Kwapisz2011}. Sin embargo, a pesar de estar bien definida en
la literatura la arquitectura de un sistema de recolección, existe
una necesidad de un componente de librerías para teléfonos móviles
que pueda ser libremente utilizado sin depender de definiciones de
API privadas (Application Programming Interfaces), servicios de en
nube (Software as a Service), o aplicaciones de terceros en Tiendas
(Ej. Google Play Services).

Esta propuesta centra en el estudio del reconocimiento de actividades
humanas utilizando sensores en teléfonos móviles inteligentes, de
manera a aportar una librería de reconocimiento para estos dispositivos.
Durante las pruebas experimentales y la recolección de los datos se
desarrollará una aplicación móvil que demuestre la efectividad de
la librería y evaluará la técnica de reconocimiento sencilla de aprendizaje
automático.

\section{Alcance y Objetivos}

\label{alcance-y-objetivos}

Inicialmente se realizará una revisión del estado del arte con el
objetivo de entender y comprender los métodos de reconocimientos de
actividades actualmente empleados.

Luego de ello se implementaría la librería propuesta con una API y
un servicio Android que atienda a los pedidos de reconocimiento de
actividades estándar. La librería además tendrá incluida la utilización
y actualización del modelo colaborativo para intentar mejorar las
predicciones de las actividades.

Además se implementará un servicio web de captura de las muestras
de la librería y actualice el modelo de predicción para futuras versiones
de la librería. El modelo de aprendizaje se basará en aprendizaje
automático con árboles de decisión (Decision Trees), algoritmo de
clasificación C4.5. Para terminar, se creará una aplicación móvil
de prueba para recolectar muestras y evaluar el resultado de predicción
utilizando la librería de reconocimiento propuesta

\subsection{Objetivo General}

\label{objetivo-general}

Implementar un sistema de reconocimiento de actividades humanas con
teléfonos móviles que aporte componentes de librería y arquitectura
de sistema abiertos.

\subsection{Objetivos Específicos}

\label{objetivos-especuxedficos}
\begin{itemize}
\item Explorar el estado del arte sobre reconocimiento de actividades humanas
(HAR). 
\item Comprender las técnicas de recolección de datos para aprendizaje en
línea en entornos restringidos para bajo consumo de batería. 
\item Comprender el procesamiento de señales en datos de sensores de aceleración
para identificar muestras de ensayo de aprendizaje automático. 
\item Entender las técnicas de clasificación por aprendizaje automático
más apropiados en entornos restringidos para bajo consumo de batería. 
\item Diseñar la arquitectura de un sistema de reconocimiento para la recolección
de datos de manera colaborativa y predicción de actividades humanas
en-linea. 
\item Aportar un componente de librería de código abierto apropiado para
teléfonos móviles con restricciones de bajo consumo de batería. 
\end{itemize}

\section{Organización del trabajo}

\label{organizaciuxf3n-del-trabajo}

El trabajo está organizado como sigue: en el Capítulo 2 se presenta
el reconocimiento de actividades, sus características, dificultades,
y los métodos abordados para el realizar esta reconocimiento.

El Capítulo 3 introduce el aprendizaje automático, los tipos de aprendizajes,
y las distintas técnicas o enfoques aplicados, y sobre todo sobre
los arboles de decisión. Estos capítulos representan el estado del
arte de este trabajo.

El Capítulo 4 se presenta el reconocimiento en dispositivos moviles.
Bla bla.. En el Capítulo 5 se presenta el modelo y la implementación
del reconocedor de actividades colaborativo, su diseño y arquitectura,
y tecnologías utilizadas para su desarrollo.

En el Capítulo 6 se presentan los experimentos realizados, y en el
capitulo 7 los resultados obtenidos. En el Capítulo 8 presentan las
conclusiones de este trabajo y los posibles trabajos futuros resultado
de este trabajo final de grado. 
