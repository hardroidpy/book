
\chapter{Introducción}

\label{chap1:introduccion}

Hace apenas una década, que los teléfonos móviles pasaron de ser simples
aparatos de comunicación, a ser dispositivos de información que se
han extendido en todo el mundo como el dispositivo electrónico más
utilizado de manera diaria. Desde la aparición de los teléfonos móviles
modernos (\emph{Smartphones}, o teléfonos inteligentes) con pantalla
táctil, la convergencia en el uso de teléfonos móviles, las redes
de datos e Internet ha ido aumentando gradualmente \cite{Fling2009}.

La capacidad y prestaciones de los teléfonos móviles modernos han
mejorado drásticamente, ya que fusiona los aspectos de teléfonos móviles
y computadoras portátiles, la promesa esperada desde hace más de una
década \cite{Tanenbaum2010}. Muchas de las tareas asociadas comúnmente
a una computadora personal de escritorio pueden ser realizadas con
un teléfono móvil moderno, pero con la ventaja adicional de que el
móvil está más próximo a las actividades diarias del portador.

Desde los teléfonos móviles modernos se tiene acceso a la información
pero en un contexto móvil. El contexto puede darse de dos formas,
la primera forma es el valor que los usuarios mismos generan a partir
de las circunstancias en la que están involucrados. La información
provee un contexto que permite al usuario entender mejor el momento
que está experimentando. Por otro lado, el contexto también es determinado
por la acción que el usuario realizar en su entorno \cite{Fling2009}.
En este último caso, el contexto físico de un usuario involucra a
su ubicación y su actividad humana. 

El reconocimiento de actividades humanas (\abbr{HAR}, \emph{Human
Activity Recognition}) es un tópico de investigación que busca diseñar
algoritmos que provean información acerca del contexto de uno o más
individuos a partir de datos ambiguos capturados en su entorno \cite{Bao2004}.
Reconocer el contexto es un componente primordial de los sistemas
inteligentes y cognitivos, este forma parte de un proceso más complejo,
donde contribuye en las etapas de análisis y captura de datos a través
de sensores externos o adjuntos a los individuos \cite{ReyesOrtiz2015,Chen2012}.
Los avances recientes en las tecnologías de computación móvil y sensores
tales como: la miniaturización, bajo consumo, buenas prestaciones,
buena conectividad y el procesamiento de datos, hizo que proliferara
el uso de los teléfonos móviles modernos con sistemas inteligentes
que monitorizan las acciones del usuario en su vida diaria. 

Diseñar sistemas móviles inteligentes, o aplicaciones móviles de contexto,
que reconozcan las actividades de un individuo, Ej. Si está caminando,
está corriendo, está quieto o moviéndose en algún vehículo tiene diversos
motivos en la actualidad \cite{CampuzanoLopez2015,Google2013l}. Tal
es la motivación, que ha aumentado la popularidad de las aplicaciones
móviles de contexto con funcionalidades en el ámbito del cuidado personal,
la movilidad y la asistencia en la vida diaria que requieren de estas
capacidades. Por citar ejemplos, en la actualidad existen diversas
alternativas en el mercado, como las proveídas por las grandes compañías
\emph{Google}, \emph{Sony} y \emph{Apple}, que son principales fabricantes
de teléfonos móviles e impulsores aplicaciones móviles basadas en
contexto.

Desde el año 2013, \emph{Google} ha proveído como complemento para
la plataforma \emph{\abbr{Android} }\cite{Google2005a} el producto
\emph{Google Play Services}, este dispone de una librería para el
reconocimiento de actividades humanas exclusivamente para desarrolladores
de aplicaciones móviles \cite{Google2013l}. Saliendo al paso, \emph{Sony}
ha lanzado una suite completa (aplicaciones móviles, pulseras y relojes)
para mantenerse en forma acompañados del producto \emph{Lifelog} \cite{Sony2016l}.
Este también es capaz de reconocedor de actividades físicas y registrar
las acciones de sus usuarios simplemente portando el teléfono en cualquier
lugar. Además, \emph{Apple} para su plataforma\emph{ \abbr{iOS}}
\cite{Apple2007i} ha dispuesto recientemente de una librería para
desarrolladores llamada \emph{HealthKit} \cite{Apple2016h}, con un
enfoque similar a los productos anteriormente mencionados. 

Las plataformas para aplicaciones \emph{\abbr{Android}} e \emph{\abbr{iOS}}
proveen un ecosistema abierto para crear aplicaciones personalizadas,
estas conducen a la expansión y el crecimiento de aplicaciones contextuales
\cite{Tanenbaum2010}. Sin embargo, a pesar de promoverse plataformas
abiertas, existe aún una carencia de proyectos colaborativos de código
abierto que contribuyan al desarrollo de sistemas inteligentes de
contexto, a excepción de ciertos precedentes (véase \cite{Kwapisz2011,LaraLabrador2013})
que han hecho esfuerzos iniciales u otras iniciativas como \cite{FUNF2016}
y \cite{SensingKit2016} orientados a la colaboración.

Desde que los teléfonos móviles modernos conectados constantemente
Internet el acceso a la Web\footnote{World Wide Web} se ha popularizando.
Esto impulsó a que la computación móvil haya crecido enormemente,
y también ha crecido el uso de la Web (cfr. \cite{NYTimes2008iph}).
Esto fomenta un ámbito de trabajo colaborativo, distribuido y móvil
donde un grupo masivo de personas pueden aportar a un propósito común,
ya sea como sujetos de estudio en el reconocimiento de actividades
humanas o en el desarrollo de herramientas para aplicaciones móviles
contextuales.

\section{Planteamiento del problema}

\label{sec11:planteamiento}

El reconocimiento de actividades humanas es un tópico de investigación
que data desde los noventa (90), y a la actualidad sigue atrayendo
bastante interés ya que abarca varias áreas novedosas de estudio.
Algunas áreas de estudio que se pueden mencionar son la computación
ubicua, la computación móvil, y la computación contextual; también
la seguridad por vigilancia, las viviendas asistidas y los ambientes
inteligentes, etc.\cite{Chen2012}. 

Reconocer actividades humanas consiste en detectar las acciones o
interacciones con el entorno que llevan acabo las personas utilizando
sensores. Las actividades humanas pueden ser catalogados en distintos
tipos y según el nivel de detalle \cite{Chen2012}. Poniendo énfasis
en las actividades básicas que un individuo puede realizar podemos
destacar las acciones físicas más simples, como lo son las actividades
ambulatorias y de transportación. Además existen otras acciones más
complejas como la postura, los movimientos en ejercicio aeróbico u
otras más seculares, Ej. Cepillarse los dientes, mirar televisión,
comer, beber, hablar por teléfono, etc. \cite{LaraLabrador2013}.

El problema del reconocimiento de actividades humanas conlleva un
proceso complejo que se resume en la utilización de técnicas de aprendizaje
automático supervisado, con un modelo de inferencia construido en
dos etapas: entrenamiento y evaluación \cite{LaraLabrador2013,Kwapisz2011}.
Es un proceso se puede resumir en un flujo iterativo que consiste
en \cite{Bao2004}:
\begin{itemize}
\item Recolectar datos etiquetados de un conjunto de individuos de forma
supervisada. 
\item Extraer variables predictoras con atributos relevantes al estudio.
\item Construir un modelo de predicción de actividades humanas.
\item Detectar con cierta precisión el conjunto de actividades previamente
etiquetadas utilizando como entrada un conjunto de datos no etiquetados.
\item Retroalimentar y/o evaluar el modelo para obtener una mejora continua.
\end{itemize}
La investigación en el área de los sistemas \abbr{HAR} utilizando
teléfonos móviles ha progresado de manera sostenida debido a los avances
tecnológicas de sensores de bajo costo, las redes inalámbricas de
alta velocidad y los dispositivos móviles inteligentes \cite{Chen2012}.
A pesar de los avances tecnológicos, aún existen importantes desafíos
por resolver como: seleccionar medidas relevantes, recolectar datos
en forma no invasiva, analizar métodos de inferencia alternativos
y extraer muestras significativas \cite{LaraLabrador2013}. Además
de la restricción al recolectar datos en condiciones realistas y tener
la flexibilidad para soportar nuevos individuos sin necesidad entrenar
el modelo permanentemente se debe diseñar el sistema bajo condiciones
especiales de ahorro de energía y procesamiento \cite{ReyesOrtiz2015}. 

De las cuestiones citadas anteriormente, en este trabajo ponemos énfasis
en cuatro puntos que enmarcan el problema a resolver y cuya justificación
responde a utilizar los teléfonos móviles como plataforma de reconocimiento
de actividades humanas.

\section{Justificación}

\label{sec12:justificaciuxf3n}

Para construir sistemas \abbr{HAR} se requiere utilizar uno o más
sensores que midan datos del entorno de un usuario. Los teléfonos
móviles modernos poseen múltiples sensores de bajo consumo, con el
tamaño y las condiciones adecuadas para la capturar de manera ubicua.
Comúnmente los teléfonos modernos se componen de sensores de: localización
(\abbr{GPS}, \emph{Global Positioning System}), brújula, aceleración,
audio (micrófonos) y vídeo (cámaras), luminosidad, temperatura, barómetro
y otros dependiendo del fabricante y/o accesorios adicionales\cite{Kwapisz2011}.

El sensor de aceleración\footnote{acelerómetro} mide el movimiento
en dos o tres ejes y puede ser combinado con otros para detectar la
orientación del dispositivo. Estas medidas proveen información crucial
para el reconocimiento de actividades humanas con un bajo consumo
de energía. El sensor de localización y podómetro (contador de pasos)
también son ideales para su uso pero poseen más restricciones asociadas.
Adicionalmente, los teléfonos modernos tienen capacidad de almacenamiento
y procesamiento, una buena conectividad por lo que son plataformas
ideales para construir sistemas de reconocimiento autónomos.

La combinación de capacidad, tamaño diminuto y bajo costo hace que
los dispositivos sean portados a cualquier lugar por sus usuarios,
esto permite recolectar los datos en condiciones realistas, una ventaja
ideal que tener que obtener muestras supervisadas en un laboratorio
\cite{Bao2004}. 

La información sobre el estado de arte de esta área de investigación
es bastante extensa y abarca las técnicas de reconocimiento, los métodos
de captura y el procesamiento de datos de sensoriales\cite{LaraLabrador2012,Kwapisz2011}.
Sin embargo, a pesar de estar definida la arquitectura los sistemas
\abbr{HAR}, aún existe un hueco de contar con un componente de software
para teléfonos móviles que pueda ser utilizado sin depender de librerías
privadas, Ej. \emph{Google Play Services} API \cite{Google2016l}
(\abbr{API}, \emph{Application Programming Interface}), o servicios
alojados en Internet (\abbr{SaaS}, \emph{Software as a Service}),
o bien aplicaciones distribuidas por terceros, Ej. \emph{Lifelog}.

El modelo de desarrollo de software libre hace posible tener un enfoque
colaborativo en dos dimensiones, por un lado el software como tal,
Ej. Una librería de reconocimiento, y por otro el los de datos sensoriales
utilizadas para el entrenamiento del modelo de predicción de actividades.

\section{Alcance y Objetivos}

\label{sec13:alcance-y-objetivos}

Este trabajo se centra en el reconocimiento de actividades humanas
con teléfonos móviles inteligentes utilizando sensores y aportar una
componente de librería de código abierto que sea libremente distribuida.
Durante la recolección de datos y las pruebas se desarrollará adicionalmente
una aplicación móvil que demuestre la efectividad de la librería como
componente de software independiente. También se evaluará el sistema
de reconocimiento de actividades basado en aprendizaje automático.

Se incluye la revisión del estado del arte de la materia con el objetivo
de comprender las metodologías de reconocimiento actualmente empleadas.
Luego, se construirá una librería cliente y un servicio basado en
\emph{\abbr{Android}} que atienda a las llamadas de procedimiento
del sistema de reconocimiento de actividades. 

Dentro del sistema de reconocimiento se incluirá un modelo colaborativo
capaz de actualizarse constantemente, a nuestro criterio, esto mejorará
las predicciones basándose en un conjunto de datos de entrenamiento
proveniente de las personas que colaboren de forma anónima y en condiciones
reales.

Además se implementará un servicio Web de captura de muestras con
predicciones para evaluar y actualizar el modelo de predicción para
futuras versiones. El modelo de aprendizaje se basará en aprendizaje
automático por árboles de decisión (\abbr{DT}, \emph{Decision Trees}).
Finalmente, para evaluar la metodología se creará una aplicación móvil
para recolectar muestras y evaluar el resultado de predicción utilizando
el sistema de reconocimiento propuesto.

\subsection{Objetivo General}

\label{sec13:objetivo-general}

Implementar un sistema de reconocimiento de actividades humanas con
teléfonos móviles cuyo principal aporte sea una librería de código
abierto libremente distribuido.

\subsection{Objetivos Específicos}

\label{sec13:objetivos-especuxedficos}
\begin{enumerate}
\item \label{enu:obe1}Definir el estado del arte sobre reconocimiento de
actividades humanas (\abbr{HAR}). 
\item \label{enu:obe2}Comprender las técnicas de recolección de datos en
entornos restringidos para bajo consumo energía. 
\item \label{enu:obe3}Comprender el procesamiento de señales de datos de
aceleración inercial para identificar variables significativas de
entrenamiento. 
\item \label{enu:obe4}Comprender la clasificación por aprendizaje automático
en entornos restringidos para bajo consumo energía. 
\item \label{enu:obe5}Diseñar un sistema de reconocimiento de actividades
que comprenda la recolección de muestras de manera colaborativa y
predicción de actividades en-línea. 
\item \label{enu:obe6}Aportar un componente de software de código abierto
para uso en teléfonos móviles inteligentes. 
\end{enumerate}

\section{Organización del trabajo}

\label{sec14:organizaciuxf3n-del-trabajo}

El trabajo está organizado en la siguiente manera: en el Capítulo
2 se presenta define el marco teórico del área describiendo los conceptos
principales del reconocimiento de actividades humanas, sus características,
dificultades, y las metodologías existentes. El Capítulo 3 definimos
el aprendizaje automático, los tipos de aprendizajes, y las distintas
técnicas o enfoques aplicados, dando énfasis en los arboles de decisión.
Estos capítulos representan el estado del arte de este trabajo.

El Capítulo 4 se presenta en detalle la construcción de un sistema
de reconocimiento para teléfonos móviles. Continuando, en el Capítulo
5 se discute el modelo y la implementación de un reconocedor de actividades
colaborativo, su diseño, arquitectura, y tecnologías utilizadas.

En el Capítulo 6 se detallan los protocolos de los experimentos realizados,
y en el Capítulo 7 los resultados obtenidos durante el estudio. Para
finalizar, en el Capítulo 8 se incluyen las conclusiones y los posibles
trabajos futuros. 
