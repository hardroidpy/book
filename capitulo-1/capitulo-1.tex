
\chapter{Introducción}

\label{chap1:introduccion}

Hace apenas una década, que los teléfonos móviles pasaron de ser simples
aparatos de comunicación, a ser dispositivos de información que se
han extendido en todo el mundo como el dispositivo electrónico más
utilizado de manera diaria. Desde la aparición de los teléfonos móviles
modernos (\emph{Smartphones}, o teléfonos inteligentes) con pantalla
táctil, la convergencia en el uso de teléfonos móviles, las redes
de datos e Internet ha ido aumentando gradualmente \cite{Fling2009}.

La capacidad y prestaciones de los teléfonos móviles modernos han
mejorado drásticamente, ya que fusiona los aspectos de teléfonos móviles
y computadoras portátiles, la promesa esperada desde hace más de una
década \cite{Tanenbaum2010}. Muchas de las tareas asociadas comúnmente
a una computadora personal de escritorio pueden ser realizadas con
un teléfono móvil moderno, pero con la ventaja adicional de que el
móvil está más próximo a las actividades diarias del portador.

Desde los teléfonos móviles modernos se tiene acceso a la información
pero en un contexto móvil. El contexto puede darse de dos formas,
la primera forma es el valor que los usuarios mismos generan a partir
de las circunstancias en la que están involucrados \cite{Fling2009}.
La información provee un contexto que permite al usuario entender
mejor el momento por el que está pasando. La segunda forma del contexto
es el entorno en el cual el usuario realiza una acción determinada
\cite{Fling2009}. El contexto físico de un usuario es su ubicación
y su actividad física. 

El reconocimiento de actividades humanas (\abbr{HAR}) es un línea
de investigación amplia que busca diseñar algoritmos que detecten
el contexto de uno o más individuos a partir de datos ambiguos capturados
en su entorno \cite{Bao2004}. El proceso de reconocimiento se realiza
a partir de datos capturados por sensores externos o adjuntos a los
individuos \cite{LaraLabrador2013,Chen2012}. Los teléfonos móviles
modernos han venido siendo sensores de bolsillo con el objetivo de
construir aplicaciones contextuales que detecten las acciones de un
individuo en su vida diaria. 

Dentro del enfoque de las aplicaciones contextuales en los teléfonos
móviles modernos, existen bastantes incentivos para reconocer las
actividades básicas ambulatorias de un individuo, de si el mismo camina,
corre, está quieto o moviéndose rápidamente en algún vehículo \cite{CampuzanoLopez2015,Google2013l}.
Actualmente, ya existen algunas soluciones privadas como las de los
fabricantes de tecnología \emph{Google}, \emph{Sony} y \emph{Apple},
principales protagonistas en la tendencia de crear este tipo de aplicaciones
basadas en contexto.

Desde el año 2013, \emph{Google} ha proveído como complemento para
la plataforma \emph{\abbr{Android} }\cite{Google2005a} dentro del
producto \emph{Google Play Services}, una implementación de reconocimiento
de actividades humanas expuesta en forma de librería para desarrolladores
\cite{Google2013l}. También, \emph{Sony} ha lanzado una suite completa
(aplicaciones, pulseras y relojes) para mantenerse en forma, acompañados
con el producto principal \emph{Lifelog} \cite{Sony2016l}, donde
un componente reconocedor de actividades físicas es incluido. Además,
\emph{Apple} para en su plataform\emph{a \abbr{iOS}} \cite{Apple2007i}
ha dispuesto recientemente una librería para desarrolladores llamada
\emph{HealthKit} \cite{Apple2016h}, de características similares
a los productos anteriormente mencionados. 

Las plataformas de aplicaciones móviles \emph{\abbr{Android}} e \emph{\abbr{iOS}}
proveen un ecosistema abierto para crear aplicaciones personalizadas.
Estas plataformas conducen la expansión y crecimiento de aplicaciones
móviles e inalámbricas \cite{Tanenbaum2010}. Sin embargo, aún no
contamos con proyectos de código abierto que contribuyan al desarrollo
del reconocimiento de actividades humanas a pesar de ciertos precedentes
no disponibles a la fecha \cite{Kwapisz2011,LaraLabrador2013}.

Por último, es notable que desde que los teléfonos móviles modernos
están conectados constantemente Internet, se ha popularizando el acceso
a la Web\footnote{World Wide Web}. Tal es así, que como la computación
móvil ha crecido rápidamente, también ha crecido el uso de la Web
(cfr. \cite{NYTimes2008iph}). Esto promueve la colaboración masiva
de personas a un propósito colaborativo común, ya sea como sujetos
de estudio en el reconocimiento de actividades humanas o en el desarrollo
de aplicaciones móviles contextuales.

\section{Planteamiento del problema}

\label{sec11:planteamiento}

El reconocimiento de actividades humanas es un tópico de investigación
que data desde los años 90 y que hasta la actualidad sigue atrayendo
interés ya que abarca varias áreas de estudio. Algunas áreas y/o aplicaciones
para mencionar son la computación ubicua, la computación móvil, y
la computación contextual; la seguridad por vigilancia, las viviendas
asistidas, los ambientes inteligentes entre otros \cite{Chen2012}. 

Reconocer actividades humanas consiste en detectar las acciones o
interacciones con el entorno que llevan acabo las personas en relación
con la utilización de sensores. Las actividades humanas pueden ser
categorizados en distintos tipos y según el nivel de detalle \cite{Chen2012},
pero poniendo foco en las actividades físicas básicas de un individuo
podemos destacar las acciones más simples como las ambulatorias y
las de transporte. Además existen otras acciones más complejas como
detectar la postura, los movimientos en entrenamiento aeróbico u otras
más seculares como cepillarse los dientes, mirar televisión, comer,
beber, hablar por teléfono, etc. \cite{LaraLabrador2013}.

El problema del reconocimiento de actividades humanas conlleva un
proceso complejo que se resumen en la utilización de la técnica de
aprendizaje automático supervisado, con un modelo de inferencia construido
en dos etapas bien conocidas: entrenamiento y evaluación \cite{LaraLabrador2013,Kwapisz2011}.
Es un proceso iterativo que consiste en \cite{Bao2004}:
\begin{itemize}
\item Recolectar datos etiquetados de un conjunto de individuos de forma
supervisada. 
\item Extraer variables con información relevante.
\item Construir un modelo de predicción (inferencia).
\item Detectar con cierta precisión las actividades previamente etiquetadas
usando como entrada cualquier conjunto de datos no etiquetados.
\item Retroalimentar y evaluar el modelo con miras a una mejora continua.
\end{itemize}
La investigación en el área de los sistemas \abbr{HAR} ha progresado
de manera sostenida debido a los avances tecnológicas de sensores
de bajo costo, las redes inalámbricas de alta velocidad y los teléfonos
móviles modernos \cite{Chen2012}. A pesar de los avances en las tecnologías,
aún existen importantes desafíos por resolver como: seleccionar medidas
relevantes, recolectar datos en forma no invasiva, analizar varios
métodos de inferencia y extraer muestras significativas. Además de
contar con ciertas restricciones como obtener datos en condiciones
realistas, poseer la flexibilidad para soportar nuevos individuos
sin entrenar el modelo permanentemente e implementar el sistema en
dispositivos bajo condiciones especiales de ahorro de energía y procesamiento
\cite{LaraLabrador2013}. 

De las cuestiones citadas ponemos énfasis en los cuatro puntos restrictivos
que definen el problema a resolver y donde a continuación se describe
la justificación de este trabajo en utilizar los teléfonos móviles
como herramienta principal.

\section{Justificación}

\label{sec12:justificaciuxf3n}

Para construir sistemas de reconocimiento es necesario utilizar uno
o más sensores que capturen los datos del entorno de los usuarios.
Los teléfonos móviles modernos poseen múltiples sensores, el tamaño
y las condiciones adecuadas capturar los datos de manera ubicua. Algunos
de los sensores disponibles comúnmente en los teléfonos inteligentes
son: localización (\abbr{GPS}, \emph{Global Positioning System}),
brújula, aceleración, audio (micrófonos), vídeo (cámaras), de luminosidad,
de temperatura, barómetro y otros más variados dependiendo del modelo,
el fabricante y los accesorios del teléfono \cite{Kwapisz2011}.

El sensor de aceleración\footnote{acelerómetro} mide el movimiento
en dos o tres ejes y puede ser combinado con otros para detectar la
orientación del dispositivo, estos son los datos que proveen información
crucial para el reconocimiento de actividades con un bajo consumo
de energía. Adicionalmente, los teléfonos inteligentes tienen gran
capacidad de almacenamiento y procesamiento, buena conectividad de
manera que son plataformas ideales para construir sistemas de reconocimiento
autónomos.

La combinación de capacidad, tamaño diminuto y bajo costo hace que
los dispositivos sean portados a cualquier lugar por los usuarios,
esto permite recolectar los datos en condiciones realistas, completamente
distinto a obtener muestras supervisadas en un laboratorio \cite{Bao2004}. 

Existe bastante información del estado de arte sobre esta línea de
investigación que abarca las técnicas de reconocimiento, los métodos
de captura y el procesamiento de datos de sensores \cite{LaraLabrador2012,Kwapisz2011}.
Sin embargo, a pesar de estar definida la arquitectura de un sistema
de reconocimiento, aún existe un hueco en contar con un componente
de software para teléfonos móviles que pueda ser utilizado libremente
sin depender de librerías privadas (Ej. \emph{Google Play Services}
API \cite{Google2016l}) (\abbr{API}, \emph{Application Programming
Interface}), servicios en Internet (\abbr{SaaS}, \emph{Software as
a Service}), o aplicaciones de terceros (Ej. \emph{Lifelog}).

El modelo de desarrollo de software libre hace posible tener un enfoque
colaborativo en dos dimensiones, por un lado el software en sí, la
librería del sistema de reconocimiento, y por otro el conjunto de
datos de muestras utilizadas para el entrenamiento del modelo de predicción
de actividades.

\section{Alcance y Objetivos}

\label{sec13:alcance-y-objetivos}

Este trabajo estudia el reconocimiento de actividades humanas utilizando
teléfonos móviles inteligentes con sensores, de manera a aportar un
sistema de reconocimiento en forma de librería abierta y libremente
distribuida. Durante la recolección de datos y las pruebas se desarrollará
adicionalmente una aplicación móvil que demuestre la efectividad de
la librería como componente de software independiente. También se
evaluará el sistema de reconocimiento de actividades basado en aprendizaje
automático.

Se incluye la revisión del estado del arte de la materia con el objetivo
de comprender las metodologías de reconocimiento actualmente empleadas.
Luego, se construirá una librería cliente y un servicio basado en
\emph{\abbr{Android}} que atienda a las llamadas de procedimiento
del sistema de reconocimiento de actividades. 

Dentro del sistema de reconocimiento se incluirá un modelo colaborativo
capaz de actualizarse constantemente, a nuestro criterio, esto mejorará
las predicciones basándose en un conjunto de datos de entrenamiento
proveniente de las personas que colaboren de forma anónima y en condiciones
reales.

Además se implementará un servicio Web de captura de muestras con
predicciones para evaluar y actualizar el modelo de predicción para
futuras versiones. El modelo de aprendizaje se basará en aprendizaje
automático por árboles de decisión (\abbr{DT}, \emph{Decision Trees}).
Finalmente, para evaluar la metodología se creará una aplicación móvil
para recolectar muestras y evaluar el resultado de predicción utilizando
el sistema de reconocimiento propuesto.

\subsection{Objetivo General}

\label{objetivo-general}

Implementar un sistema de reconocimiento de actividades humanas con
teléfonos móviles cuyo principal aporte sea un servicio y una librería
libremente distribuidos.

\subsection{Objetivos Específicos}

\label{objetivos-especuxedficos}
\begin{enumerate}
\item \label{enu:obe1}Definir el estado del arte sobre reconocimiento de
actividades humanas (\abbr{HAR}). 
\item \label{enu:obe2}Comprender las técnicas de recolección de datos en
entornos restringidos para bajo consumo energía. 
\item \label{enu:obe3}Comprender el procesamiento de señales de datos de
aceleración inercial para identificar variables significativas de
entrenamiento. 
\item \label{enu:obe4}Comprender la clasificación por aprendizaje automático
en entornos restringidos para bajo consumo energía. 
\item \label{enu:obe5}Diseñar un sistema de reconocimiento de actividades
que comprenda la recolección de muestras de manera colaborativa y
predicción de actividades en-línea. 
\item \label{enu:obe6}Aportar un componente de software de código abierto
para uso en teléfonos móviles inteligentes. 
\end{enumerate}

\section{Organización del trabajo}

\label{sec14:organizaciuxf3n-del-trabajo}

El trabajo está organizado en la siguiente manera: en el Capítulo
2 se presenta define el marco teórico del área describiendo los conceptos
principales del reconocimiento de actividades humanas, sus características,
dificultades, y las metodologías existentes. El Capítulo 3 definimos
el aprendizaje automático, los tipos de aprendizajes, y las distintas
técnicas o enfoques aplicados, dando énfasis en los arboles de decisión.
Estos capítulos representan el estado del arte de este trabajo.

El Capítulo 4 se presenta en detalle la construcción de un sistema
de reconocimiento para teléfonos móviles. Continuando, en el Capítulo
5 se discute el modelo y la implementación de un reconocedor de actividades
colaborativo, su diseño, arquitectura, y tecnologías utilizadas.

En el Capítulo 6 se detallan los protocolos de los experimentos realizados,
y en el Capítulo 7 los resultados obtenidos durante el estudio. Para
finalizar, en el Capítulo 8 se incluyen las conclusiones y los posibles
trabajos futuros. 
