
\chapter{Sistemas HAR }

\label{chap4:sistemas-de-reconocimiento}

\section{Introducción}

\label{sec41:introduccion}El diseño de sistemas con conocimiento
del contexto promueven una interacción novedosa con los usuarios y
diversas aplicaciones en las áreas de ambientes inteligentes, repuesta
a emergencias, vigilancia y otros \cite{Choudhury2008}. Un sistema
con la capacidad de reconocer las actividades humanas por medio del
uso de sensores empotrados posee mecanismos para crear aplicaciones
de cuidado personal, salud y asistencia inteligente. El requerimiento
primordial de un sistema con una aplicación de contexto es que este
pueda ser portado continuamente como atuendo de sus usuarios (un sistema
\abbr{Wearable}). Por lo tanto, un sistema que acompaña continuamente
al usuario puede interaccionar oportunamente con el mismo ya que este
tiene la capacidad de observar en tiempo real las acciones de su portador.
La ventaja adicional de un sistema de este tipo es que puede ser desactivado
fácilmente o removido de la actividad diaria de su usuario.

En este capítulo se definen los componentes principales de un sistema
de reconocimiento de actividades humanas (sistemas \abbr{HAR}). El
objetivo principal del sistema \abbr{HAR}en conjunto es proveer módulo
base para aplicaciones novedosas de contexto. El módulo debe ser capaz
de reconocer varias actividades realizadas rutinariamente de diferentes
maneras, por diferentes usuarios y en diferentes condiciones contextuales.
Las funciones principales de los componentes descritos en la primera
sección exponen los mecanismos para implementar los mismos en base
trabajos relacionados de \abbr{HAR} \cite{Choudhury2008,ReyesOrtiz2015}. 

La última sección, enumera los requisitos no funcionales para lograr
una aplicación de contexto móvil y ubicua. Por un lado, las características
esperadas en una aplicación de esta naturaleza, y por el otro los
requisitos técnicos de los dispositivos móviles y los sensores empotrados
utilizados como instrumentos\footnote{\emph{hardware}} de implementación.

\section{Arquitectura del sistema}

\label{sec42:componentes}El diseño de la arquitectura de componentes
de un sistema \abbr{HAR} se rige de acuerdo a las guías de implementación
de una aplicación de aprendizaje automático (\abbr{ML}). De acuerdo
al proceso definido en la sección \ref{sec262:proceso-har}, se tiene
en cuenta la misma estructura de componentes y las mismas fases de
procesamiento de información. Además, se debe contemplar que el proceso
se divide en dos etapas: la etapa de entrenamiento y la de evaluación
\cite{LaraLabrador2013}. 

Ambas etapas requieren la implementación de los mismos componentes,
pero un sistema \abbr{HAR} práctico debe contemplar principalmente
la fase de evaluación, ya que el reconocimiento de actividades resulta
de una \emph{predicción} basado en un algoritmo de \abbr{ML} en-linea
(\emph{On-line learning}). Sin embargo, la etapa de entrenamiento
es un elemento clave para el sistema ya que es el punto de partida
para el \emph{aprendizaje} basado en modelo de \abbr{ML} y usualmente
se realiza bajo demanda (\emph{Off-line learning}).

Bajo el marco teórico de los sistemas \abbr{HAR} basados en \abbr{ML},
se han identificado unos componentes comunes para realizar las funcionalidades
de aprendizaje y predicción según\cite{Choudhury2008}. Un sistema
de reconocimiento de actividades posee tres componentes:
\begin{itemize}
\item un \emph{recolector }de medidas
\item un\emph{ procesador }de muestras 
\item un \emph{clasificador }de actividades
\end{itemize}
En la \figref{fig42:componentes-har} se muestra una vista general
de los componentes y sus interrelaciones. Las funcionalidades de cada
componente se describen a continuación. 

\begin{figure}[!tbph]
\centering{}\includegraphics[width=1\linewidth]{capitulo-4/graphics/diagrama_4_1}\caption{Componentes de los sistemas \abbr{HAR}}
\label{fig42:componentes-har}
\end{figure}


\subsection{Recolector de medidas}

\label{sec421:recolector-datos}El recolector de datos obtiene medidas
de los sensores y continuamente registrar los datos en forma indexada
en la dimensión del tiempo. La captura de datos para los sistemas
\abbr{HAR} se realiza con instrumentos de medición apropiados: en
nuestro caso con sensores (\abbr{Wearables}, véase \ref{sec23:sensores}).
Las señales de los sensores capturan directamente de los usuarios
por medio de observaciones continuas, por lo que deben estar anexados
al cuerpo; en la cintura, la muñeca, el pectoral, los muslos o en
la cabeza \cite{Bao2004}. Adicionalmente, los sensores puede ser
portados simplemente ya están comúnmente empotrados en dispositivos
de uso regular como los teléfonos móviles modernos, relojes o lentes
inteligentes \cite{LaraLabrador2012,Choudhury2008}.

A continuación se describe el método de registro de datos en flujo
además de algunos ejemplos de variables relevantes utilizadas.

\subsubsection{Registro}

El proceso de registro consiste en capturar las señales de un sensor
y separar las medidas en una o más variables dependientes de cada
tipo de sensor. La organización de los registros se debe realizar
con respecto al tiempo. Es decir, se dispone de un flujo continuo
de datos con una marca de tiempo almacenados de manera secuencial
en un medio permanente para su posterior procesamiento. 

La marca de tiempo se mide milisegundos y dependiendo del tipo de
sensor el intervalo entre medidas de variables puede variar en el
mismo orden, Ej. con tasa de salida de \texttt{60 \abbr{Hz} }se tendrían
60 muestras en un segundo. 

Las señales capturadas de sensores se pueden clasificar en de movimiento,
entorno, fisiológicas y posición.

\subsubsection{Señales de Movimiento}

Las señales de movimiento en sensores son altamente informativos para
las \abbr{HAR}. El acelerómetro y giroscopio son los principales
instrumentos que miden el movimiento por medio de sensores empotrados
y portados por los usuarios.

Los acelerómetros miden señales de acuerdo a diferentes tipos de movimientos,
incluyendo la aceleración lineal y centrípeta, la gravedad y vibración
\cite{Goehl2007} en dos o tres dimensiones. Las variables medidas
están expresadas en la magnitud de la aceleración del dispositivo
con respecto la orientación del mismo (Ej. mide $-9.8\,m\,s^{-2}$
en dirección al suelo). Al representar el atributo medido por la aceleración
con el símbolo$a$ que es el vector de los componentes de la aceleración
en tres ejes $(x,y,z)$ donde cada componente es representado por
$a_{x}$, $a_{y}$ y $a_{z}$. En la figura \figref{fig421a:muestra-ac}
se muestra un ejemplo de la señal de aceleración.

\begin{figure}[!tbph]
\begin{centering}
\includegraphics[bb = 0 0 200 100, draft, type=eps]{capitulo-4/graficos/signal_a3d}
\par\end{centering}
\caption{\label{fig421:muestra-ac}Señal de aceleración en tres dimensiones}
\end{figure}

Los giroscopios, o sensores de taza angular, miden señales de la rapidez
de giro de los objetos en tres dimensiones \cite{Goehl2007}. Las
variables medidas están expresadas en el velocidad angular de rotación
del dispositivo con respecto a los ejes de orientación del mismo (Ej.
mide \foreignlanguage{english}{$-0.1\,rad\,s^{-1}$} en relación a
un eje). Al representar el atributo medido por el giroscopio con el
símbolo$w$ que es el vector de los componentes de la giro en tres
ejes $(x,y,z)$ donde cada componente es representado por $w_{x}$,
$w_{y}$ y $w_{z}$. En el cuadro \ref{fig421:muestra-ac} se muestra
un ejemplo de las señales registradas para el sensor de aceleración.

\begin{table}[!tbph]
\begin{centering}
\begin{tabular}{|c|c|c|c|}
\hline 
$t$ & $a_{x}$ & $a_{y}$ & $a_{z}$\tabularnewline
\hline 
\hline 
$0$ & \texttt{1.3} & \texttt{-2.1} & \texttt{0}\tabularnewline
\hline 
$1/s_{1}$ & \texttt{1.4} & \texttt{-2.3} & \texttt{0.1}\tabularnewline
\hline 
$2/s_{2}$ & \texttt{1.1} & \texttt{-2.6} & \texttt{0}\tabularnewline
\hline 
... & \texttt{...} & \texttt{...} & \texttt{...}\tabularnewline
\hline 
$t_{max}$ & \texttt{1.8} & \texttt{2.2} & \texttt{-0.4}\tabularnewline
\hline 
\end{tabular}
\par\end{centering}
\caption{\label{tab421:ex-se=0000F1ales}Ejemplo de señales medidas de aceleración}
\end{table}


\subsubsection{Señales de Posición}

\subsubsection{Señales de Fisiológicas}

\subsubsection{Señales del Ambiente}

\subsection{Procesador de muestras}

\label{sec422:proceso-se=0000F1ales}Proceso de señales en bruto para
adecuarlos muestras con variables significativas que permitan discriminar
las actividades a reconocer

\subsubsection{Etiquetado}

\subsubsection{Filtro de Señal}

\subsubsection{Muestreo}

\subsection{Clasificador de actividades}

\label{sec423:clasificador}Utiliza las muestras extraídas para construir
un modelo e predecir qué actividad probable está realizando un individuo
en un determinado instante.

\subsubsection{Clasificación}

\subsubsection{Reconocimiento}

\section{Capacidades deseables}

\subsection{Características no funcionales}

\label{sec431:caracteristicas}Existen un conjunto de características
deseables que deben ser satisfechas para la construcción efectiva
de los sistemas de reconocimiento. Estas características abordan cuestiones
de diseño importantes que conciernen a la calidad y al funcionamiento
del sistema:
\begin{enumerate}
\item Portabilidad, el sistema utiliza sensores adjuntos a los individuos
(Ej. el acelerómetro) y no deben obstruir las actividades cotidianas
de los usuarios durante su uso. El fin es de evitar que se afecte
la adopción masiva del sistema. 
\item Conectividad, el sistema debe transmitir de manera confiable los datos
recolectados y/o procesados a algún componente desplegado de forma
remota. 
\item Almacenamiento, el sistema debe persistir los datos recolectados y/o
procesados de manera local en el dispositivo móvil con el fin de mantener
la calidad y minimizar la cantidad transferida a otros componentes.
\item Procesamiento, el sistema debe procesar y transformar los datos en
bruto para producir información relevante para el reconocimiento de
actividades.
\item Ubiquidad, el sistema debe operar en cualquier condición y contexto
en que la persona se encuentre sin interferir u obligar al usuario
a interactuar con el sistema.
\item Uso de energía, el sistema debe preservar el uso de energía en los
dispositivos móviles que están implementados. La lectura de datos,
el procesamiento y la conectividad no deben incurrir en gastos excesivos
de energía para que el sistema pueda operar.
\item Privacidad, el sistema debe mantener de manera confidencial los datos
recolectados y/o producidos durante la adopción masiva del sistema,
además de alertar sobre la utilización de datos sensibles que requieran
el consentimiento del usuario.
\end{enumerate}

\subsection{Dispositivos móviles}

\label{sec432:dispositivos-moviles}Descripción técnica de dispositivos
móviles: procesador, memoria, sensores y almacenamiento

\subsubsection{Teléfonos móviles}

\subsubsection{Relojes inteligentes}

\subsection{Sensores empotrados}

\label{sec433:sensores-empotrados}Descripción técnica de los sensores
de aceleración, variables, orientación en dispositivo, unidades de
medida, precisión vs consumo.

\subsubsection{Acelerómetro}

\subsubsection{Giroscopio}

\subsubsection{GPS/WIFI}

\section{Conclusión}

Resumen
