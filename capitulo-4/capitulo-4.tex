
\chapter{Sistemas de Reconocimiento de Actividades }

\label{chap:sistemas-de-reconocimiento}

Este capítulo describe la arquitectura común de un sistema de reconocimiento
de actividades. Se describe detalladamente los componentes del sistema
además de sus funciones principales de manera general. 

\section{Introducción}

Los sistemas de reconocimiento de actividades humanas son semejantes
a cualquier aplicación de aprendizaje automático. Según el estado
del arte \cite{LaraLabrador2013}, estos sistemas de reconocimiento
poseen la misma estructura y aplican las mismas fases divididas en
dos etapas, la de entrenamiento y la de evaluación (ver Capítulo 2).
A pesar de seguir la mismas guías de diseño, un sistema puede optar
por no implementar en al menos una de las etapas, todos los componentes
citados en este apartado. En otras palabras, un sistema de reconocimiento
dado puede implementar simplemente los componentes necesarios para
la etapa de evaluación dejando de lado cualquier elemento relacionado
a la etapa de entrenamiento y viceversa. 

\section{Componentes}

Dentro del marco teórico de los sistemas de reconocimiento, se han
identificado unos componentes mínimos requeridos para realizar las
actividades de aprendizaje y predicción de manera manual o automatizada
\cite{Choudhury2008}. Un sistema de reconocimiento de actividades
posee tres componentes principales:
\begin{itemize}
\item un \emph{recolector de datos}
\item un\emph{ procesador de muestras} 
\item un \emph{clasificador }
\end{itemize}
Las responsabilidades y detalles de cada componente se describen en
los siguientes apartados. En la siguiente figura se muestra una vista
general de los componentes y sus interrelaciones.

\begin{figure}
\caption{Diagrama de componentes de un sistema HAR}

\end{figure}


\subsection{Recolector de datos}

El componente recolector de datos tiene la función de capturar datos
en bruto de sensores que sean relevantes en el estudio de los sistemas
\hyperlink{abbr}{HAR}. La captura de datos debe ser realizada con
instrumentos de medición apropiados: sensores adjuntos a las personas
en estudio. 

Los sensores se agrupan según los atributos de estudio de los sistemas
\abbr{HAR} \cite{LaraLabrador2013}:
\begin{description}
\item [{Ubicación}] miden datos obtenidos con las redes celulares 3G y
los sistemas \abbr{GPS}. Provee información contextual bastante relevante
de la posición de la persona, y ciertas medidas de movimiento pero
a costa de un consumo considerable de energía.
\item [{Movimiento}] miden datos inerciales como la aceleración y la orientación
respecto a un marco de referencia relativo al dispositivo de medición.
Los sensores de aceleración, giroscopio y la brújula son los más comunes
y utilizados para reconocimiento de actividades con bajo consumo de
energía y buena precisión de reconocimiento \cite{Bao2004,LaraLabrador2012}.
\item [{Fisiológico}] miden signos vitales del individuo como ritmo cardiaco
(\abbr{HRM}, \emph{Hearth Rate Monitor}), temperatura del cuerpo,
ritmo de respiración y otros.
\item [{Ambiental}] miden datos externos que rodean al individuo como el
nivel de ruido, la humedad y/o la temperatura. Los sensores de luz,
cámara, micrófonos y termómetros son utilizados para medir estos atributos. 
\end{description}
Los datos medidos deben ser registrados de manera persistente o transmitidos
para ser procesados y utilizados por otros componentes. Tanto en las
actividades de aprendizaje y predicción se deben recolectar datos
de sensores.

\subsection{Procesador de muestras}

que procesa los datos en bruto para adecuarlos muestras con variables
significativas que permitan discriminar las actividades a reconocer

\subsection{Clasificador}

que utiliza las muestras extraídas para inferir qué actividad probable
está realizando un individuo en un determinado instante.

\section{Capacidades}

Existen un conjunto de características deseables que deben ser satisfechas
para la construcción efectiva de los sistemas de reconocimiento. Estas
características abordan cuestiones de diseño importantes que conciernen
a la calidad y al funcionamiento del sistema:
\begin{enumerate}
\item Portabilidad, el sistema utiliza sensores adjuntos a los individuos,
por ejemplo el sensor de aceleración, y no debe obstruir las actividades
cotidianas de una persona durante su uso para evitar que esto atente
contra la adopción masiva del sistema. 
\item Conectividad, el sistema debe poder transferir de manera confiable
los datos recolectados y procesados a un componente remoto. 
\item Almacenamiento, el sistema debe persistir datos recolectados y procesados
de manera local para mantener la calidad de los mismos y minimizar
la cantidad transferida a otro componente remoto.
\item Procesamiento, el sistema debe realizar tareas de procesamiento y
transformación de datos para producir información relacionada al reconocimiento
de actividades.
\item Ubiquidad, el sistema debe operar en cualquier condición y contexto
en que la persona se encuentre sin interferir u obligar a la persona
a interactuar con el sistema.
\item Uso de energía, el sistema debe preservar el uso de energía en los
dispositivos móviles que están implementados. La lectura de datos
de los sensores, el procesamiento y la conectividad no deben incurrir
en gastos excesivos de energía para que el sistema pueda operar.
\item Privacidad, el sistema debe de mantener de manera confidencial los
datos recolectados y producidos durante la adopción masiva del sistema,
además de avisar sobre la utilización de datos sensibles que puedan
requerir consentimiento del usuario.
\end{enumerate}

