
\chapter{Sistemas de Reconocimiento de Actividades }

\label{chap:sistemas-de-reconocimiento}

Este capítulo describe la arquitectura común de un sistema de reconocimiento
de actividades. Se describe detalladamente los componentes del sistema
además de sus funciones principales de manera general. 

\section{Introducción}

Los sistemas de reconocimiento de actividades humanas son semejantes
a cualquier aplicación de aprendizaje automático. Según el estado
del arte \cite{LaraLabrador2013}, estos sistemas de reconocimiento
poseen la misma estructura y aplican las mismas fases divididas en
dos etapas, la de entrenamiento y la de evaluación (ver Capítulo 2).
A pesar de seguir la mismas guías de diseño, un sistema puede optar
por no implementar en al menos una de las etapas, todos los componentes
citados en este apartado. En otras palabras, un sistema de reconocimiento
dado puede implementar simplemente los componentes necesarios para
la etapa de evaluación dejando de lado cualquier elemento relacionado
a la etapa de entrenamiento y viceversa. 

\section{Componentes}

Dentro del marco teórico de los sistemas de reconocimiento, se han
identificado unos componentes mínimos requeridos para realizar las
actividades de aprendizaje y predicción de manera manual o automatizada
\cite{Choudhury2008}. Un sistema de reconocimiento de actividades
posee tres componentes principales:
\begin{itemize}
\item un \emph{recolector de datos}
\item un\emph{ procesador de muestras} 
\item un \emph{clasificador }
\end{itemize}
Las responsabilidades y detalles de cada componente se describen en
los siguientes apartados. En la siguiente figura se muestra una vista
general de los componentes y sus interrelaciones.

\begin{figure}[!tbph]
\centering{}\includegraphics[width=1\linewidth]{capitulo-4/graphics/diagrama_4_1}\caption{Componentes del sistema HAR}
\label{fig:comphar}
\end{figure}


\subsection{Recolector de datos}

El componente recolector de datos tiene la función de capturar datos
de sensores indexadas en la dimensión del tiempo. La captura de datos
relevantes para los sistemas \abbr{HAR} debe ser realizada con instrumentos
de medición apropiados: sensores adjuntados a sus usuarios. 

Los sensores pueden caracterizarse según los siguientes atributos
\cite{LaraLabrador2013}:
\begin{description}
\item [{Ubicación}] miden datos obtenidos con las redes celulares 3G y
los satélites de navegación \abbr{GPS}. Provee información de contexto
bastante relevante acerca de la posición del individuo, además de
ciertas medidas de movimiento pero con un consumo moderado de energía.
\item [{Movimiento}] miden datos inerciales como la aceleración y la orientación
respecto a un marco de referencia relativo al dispositivo que contiene
los sensores. El acelerómetro, giroscopio y la brújula son los sensores
más comunes y utilizados para reconocimiento de actividades con un
bajo consumo de energía y buena precisión de reconocimiento \cite{Bao2004,LaraLabrador2012}.
\item [{Fisiología}] miden signos vitales del individuo como el ritmo cardíaco
(\abbr{HRM}, \emph{Hearth Rate Monitor}), la temperatura del cuerpo,
el ritmo de respiración, entre otros.
\item [{Ambiental}] miden datos externos que rodean al individuo como el
nivel de ruido, la humedad y/o la temperatura. Los sensores de luz,
cámara, micrófonos y termómetros miden estos datos. 
\end{description}
Los datos de medidos deben ser registrados de manera secuencial y
transmitidos para su posterior procesamiento por otros componentes
del sistema. En ambas etapas, aprendizaje y predicción, se deben recolectar
datos de los sensores.

\subsection{Procesador de muestras}

que procesa los datos en bruto para adecuarlos muestras con variables
significativas que permitan discriminar las actividades a reconocer

\subsection{Clasificador}

que utiliza las muestras extraídas para inferir qué actividad probable
está realizando un individuo en un determinado instante.

\section{Capacidades}

Existen un conjunto de características deseables que deben ser satisfechas
para la construcción efectiva de los sistemas de reconocimiento. Estas
características abordan cuestiones de diseño importantes que conciernen
a la calidad y al funcionamiento del sistema:
\begin{enumerate}
\item Portabilidad, el sistema utiliza sensores adjuntos a los individuos
(Ej. el acelerómetro) y no deben obstruir las actividades cotidianas
de los usuarios durante su uso. El fin es de evitar que se afecte
la adopción masiva del sistema. 
\item Conectividad, el sistema debe transmitir de manera confiable los datos
recolectados y/o procesados a algún componente desplegado de forma
remota. 
\item Almacenamiento, el sistema debe persistir los datos recolectados y/o
procesados de manera local en el dispositivo móvil con el fin de mantener
la calidad y minimizar la cantidad transferida a otros componentes.
\item Procesamiento, el sistema debe procesar y transformar los datos en
bruto para producir información relevante para el reconocimiento de
actividades.
\item Ubiquidad, el sistema debe operar en cualquier condición y contexto
en que la persona se encuentre sin interferir u obligar al usuario
a interactuar con el sistema.
\item Uso de energía, el sistema debe preservar el uso de energía en los
dispositivos móviles que están implementados. La lectura de datos,
el procesamiento y la conectividad no deben incurrir en gastos excesivos
de energía para que el sistema pueda operar.
\item Privacidad, el sistema debe mantener de manera confidencial los datos
recolectados y/o producidos durante la adopción masiva del sistema,
además de alertar sobre la utilización de datos sensibles que requieran
el consentimiento del usuario.
\end{enumerate}

