
\chapter{Conclusiones y Trabajos Futuros}

\label{cap8:conclusiones-y-trabajos-futuros}

\section{Conclusiones del Presente Trabajo}

\label{sec81:conclusiones}El reconocimiento de actividades humanas
es un área de investigación multidisciplinaria con constantes avances
y de diversos usos en la actualidad debido a su relevancia en la computación
contextual y ubicua. Los avances en la miniaturización de los sistemas
de computación y sensores hacen del área un tanto atractiva como desafiante
para explorar y proponer nuevas ideas en este ámbito \cite{LaraLabrador2013}. 

Los esfuerzos realizados en este trabajo han tenido en cuenta además
la colaboración y el uso de Internet de manera a aprovecharlos para
mejorar los sistemas de reconocimiento, y como se demuestra en la
sección anterior, la aplicación de resultados colaborativos pueden
ser aprovechados para mejorar el sistema desarrollado. 

Es así que en este trabajo se construyó un sistema de reconocimiento
de actividades humanas colaborativo que utiliza tres elementos principales:
teléfonos móviles inteligentes, una librería que a la par se ha hecho
disponible como software libre y la Internet. 

El desarrollo del trabajo se llevó acabo en dos etapas. La primera
etapa consistió en el entrenamiento de un clasificador y la segunda
etapa consistió en la construcción de un sistema reconocedor de actividades
para teléfonos móviles. Los principales aportes de este trabajo se
produjeron en la segunda etapa:
\begin{itemize}
\item Por una parte se tiene como resultado tangible un componente de código
abierto reutilizable para reconocer actividades humanas. Este componente
es denominado \emph{\abbr{HARDroid}}.
\item Por otra parte este componente posibilita el mejoramiento iterativo
de su desempeño mediante un esquema colaborativo.
\end{itemize}
Respecto al primer punto, es importante destacar que \emph{\abbr{HARDroid}}
puede ser aprovechado por otras aplicaciones móviles de dos maneras
concretas \cite{GimenezYegros2016c}: 
\begin{itemize}
\item el mismo puede ser incluido en tiempo de construcción, o
\item puede ser aprovechado como un servicio a través de una interfaz de
integración.
\end{itemize}
Los experimentos realizados en el presente trabajo y documentados
en la sección previa muestran que \emph{\abbr{HARDroid}} es capaz
de producir un modelo en el cual se observan las siguientes características:
\begin{itemize}
\item Una tasa de aciertos del 91\%, un coeficiente de confiabilidad alrededor
del 87\%, y una tasa de errores absolutos menores a 1 (\secref{ssec6:evaluacion}). 
\item Las métricas de precisión y exhaustividad se corresponden con los
aciertos con valores entre el 91\% y el 92\% (\secref{ssec6:colaboracion}).
Esto es verificable en la \tabref{tab6:matriz-confusion} cuya cantidad
de falsos negativos alcanza apenas el 3\%. 
\item La capacidad de extender el modelo para reconocer diversos tipos de
actividades así como en bicicleta, en vehículo, entre otros (\secref{ssec6:verificacion}).
\item La posibilidad de mejorar el clasificador colaborativamente mediante
la inclusión de muestras de aciertos colectadas en campo (\secref{ssec6:colaboracion}).
\end{itemize}
Además de los resultados verificables, se aporta una librería reutilizable
\cite{GimenezYegros2016d}, una aplicación como servicio independiente
\cite{GimenezYegros2016a} y una aplicación de encuesta de recolección
de muestras \cite{GimenezYegros2016e} todas disponibles en Internet.

\section{Trabajos Futuros Propuestos}

\label{sec82:trabajos-futuros}Luego de la experiencia obtenida y
documentada en el presente trabajo, teniendo en cuenta la amplia aplicabilidad
del reconocimiento de actividades humanas en especial en aplicaciones
para dispositivos móviles, a continuación se proponen algunos trabajos
que se desprenden del presente:
\begin{enumerate}
\item Segmentar los grupos de individuos distintos por rangos de edad, sexo
y/o factores fisiológicos y otros, de manera a generar modelos específicos
para cada grupo que posibiliten el desarrollo de una nueva gama de
aplicaciones. 
\item Incorporar al sistema \emph{HARDroid} desarrollado otros métodos de
aprendizaje automático que permitan mejorar la tasa de aciertos. Entre
estos se sugieren por ejemplo las técnicas de vectores de soporte
(\abbr{SVM}) o redes neuronales (\abbr{ANN}).
\item Incluir más variables en el procesamiento de señales de entrada para
mejorar las predicciones teniendo en cuenta la orientación del dispositivo. 
\item Extender el sistema \emph{HARDroid} para identificar otras actividades
humanas aplicado a otras disciplinas o contextos. Por ejemplo en medicina
para detectar caídas, posturas o actividades no recomendadas. En el
campo militar para recomendar actividades más favorables para situación.
\item Utilizar o aprovechar otros accesorios que agreguen variables relevantes
en el reconocimiento de actividades dependiendo del contexto. Por
ejemplo, un reloj inteligente (\emph{\abbr{Smartwatch}}).
\end{enumerate}

