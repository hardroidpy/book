
\chapter{Conclusiones y Trabajos Futuros}

\label{cap8:conclusiones-y-trabajos-futuros}

\section{Conclusiones del presente trabajo}

\label{sec81:conclusiones}El reconocimiento de actividades humanas
es un área de investigación multidisciplinaria con constantes aportes
y diversos usos en la actualidad debido a su relevancia en la computación
contextual y ubicua. Los avances en la miniaturización de los sistemas
de computación y sensores hacen del área un tanto atractiva como desafiante
para explorar y proponer nuevas ideas en este ámbito \cite{LaraLabrador2013}. 

Los esfuerzos realizados en este trabajo han tenido en cuenta además
la colaboración y el uso de Internet de manera a aprovechar los mismos
en la mejora de los sistemas de reconocimiento, y como se mostró en
la sección anterior la aplicación de resultados colaborativos puede
ser aprovechada para mejorar el sistema desarrollado. 

Así, en este trabajo se construyó un sistema de reconocimiento de
actividades colaborativo que utiliza tres elementos principales: teléfonos
móviles inteligentes, una librería que a la par se ha hecho disponible
como software libre y la Internet. 

El desarrollo del trabajo se llevó acabo en dos etapas. La primera
etapa consistió en el entrenamiento de un clasificador y la segunda
etapa consistió en la construcción de un sistema reconocedor de actividades
para teléfonos móviles. Los principales aportes de este trabajo se
produjeron en la segunda etapa:
\begin{itemize}
\item Por una parte se tiene como resultado tangible un componente de código
abierto reutilizable para reconocer actividades humanas. Este componente
es denominado \emph{HARDroid}.
\item Por otra parte este componente posibilita el mejoramiento iterativo
de su desempeño mediante un esquema colaborativo.
\end{itemize}
Respecto al primer punto, es importante destacar que \emph{HARDroid}
puede ser aprovechado por otras aplicaciones móviles de dos maneras
concretas \cite{GimenezYegros2016a}: 
\begin{itemize}
\item el mismo puede ser incluido en tiempo de construcción, o
\item puede ser aprovechado como un servicio a través de una interfaz de
integración.
\end{itemize}
Los experimentos realizados en el presente trabajo y documentados
en la sección previa muestran que \emph{HARDroid} es capaz de producir
un modelo en el cual se observan las siguientes características:
\begin{itemize}
\item Una alta tasa de aciertos ($91$\% a $92$\%), y por lo tanto una
baja tasa de errores. Esto se puede verificar en la MATRIZ DE CONFUSION
RESULTANTES
\item La capacidad de extender el modelo para reconocer diversos tipos de
actividades como, en bicicleta, en vehículo, entre otros.
\item La posibilidad colaborativamente el modelo mediante la inclusión de
muestras de aciertos colectadas en campo.
\end{itemize}
(REVISAR CONCLUSIONES COMPLEMENTANDO CON REF A LOS RESULTADOS)

\section{Trabajos Futuros propuestos}

\label{sec82:trabajos-futuros}Luego de la experiencia obtenida y
documentada en el presente trabajo, teniendo en cuenta la amplia aplicabilidad
del reconocimiento de actividades humanas en especial en aplicaciones
para dispositivos móviles, a continuación se proponen algunos trabajos
que se desprenden del presente:
\begin{enumerate}
\item Segmentar los grupos de individuos distintos por rangos de edad, sexo
y/o factores fisiológicos y otros, de manera a generar modelos específicos
para cada grupo que posibiliten el desarrollo de una nueva gama de
aplicaciones. 
\item Incorporar al sistema \emph{HARDroid} desarrollado otros métodos de
aprendizaje automático que permitan mejorar la tasa de aciertos. Entre
estos se sugieren por ejemplo las técnicas de vectores de soporte
(\abbr{SVM}) o redes neuronales (\abbr{ANN}).
\item Incluir más variables en el procesamiento de señales de entrada para
mejorar las predicciones teniendo en cuenta la orientación del dispositivo. 
\item Extender el sistema \emph{HARDroid} para identificar otras actividades
humanas aplicado a otras disciplinas o contextos. Por ejemplo en medicina
para detectar caídas, posturas o actividades no recomendadas. En el
campo militar para recomendar actividades más favorables para situación.
\item Utilizar o aprovechar otros accesorios que agreguen variables relevantes
en el reconocimiento de actividades dependiendo del contexto. Por
ejemplo, un reloj inteligente (\emph{\abbr{Smartwatch}}).
\end{enumerate}

