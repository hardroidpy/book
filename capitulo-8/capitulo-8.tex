
\chapter{Conclusiones y Trabajos Futuros}

\label{conclusiones-y-trabajos-futuros}

\section{Conclusiones}

\label{conclusiones}

El reconocimiento de actividades humanas es un área de investigación
multidisciplinaria con constantes aportes y diversos usos en la actualidad
debido a su relevancia en la computación contextual y ubicua. Los
avances en la miniaturización de los sistemas de computación y sensores
hacen del área tanto atractiva como desafiante para explorar y proponer
nuevas ideas en este ámbito \cite{LaraLabrador2013}. Sin embargo,
los esfuerzos realizados por otros trabajos no han tenido en cuenta
la dimensión de la colaboración y el uso de Internet para contribuir
a la mejora de los sistemas de reconocimiento, tal como se ha hecho
este trabajo.

El problema planteado en la sección \ref{objetivo-general} de diseñar
un sistema de reconocimiento de actividades colaborativo fue abordado
con la utilización de tres elementos principales para lograr el objetivo
principal: los teléfonos móviles, el software libre y la Internet.
Los primeros cuatro objetivos específicos (1 al 4) descritos en la
sección\ref{objetivos-especuxedficos} son cuestiones del marco teórico
en el estudio de HAR donde no se hace una contribución específica.
Los objetivos específicos 5 y 6 son el aporte principal de este trabajo
y completan la solución al problema planteado.

Primeramente, el problema de recolectar datos en condiciones realistas
fue resuelto por medio de una aplicación de captura de datos en forma
de encuesta implementado para los teléfonos móviles que comparte los
datos recolectados para retroalimentar el modelo de aprendizaje del
sistema de reconocimiento.

Además, el aporte principal de este trabajo es el sistema de reconocimiento
que está disponible como proyecto de software libre para la plataforma
\emph{Android} que puede ser incluido por otras aplicaciones móviles
en tiempo de construcción o simplemente puede ser utilizada como un
servicio en los teléfonos móviles \cite{hardroid2016a}.

Para finalizar, la colaboración se desenvuelve en dos aspectos principales.
En el aspecto de los datos para el modelo de aprendizaje automático.
Las personas comparten información relevante al utilizar la aplicación
de encuesta y el sistema de reconocimiento puede actualizar su modelo
periódicamente. En el aspecto del componente de software donde la
implementación del reconocedor está disponible para que cualquiera
lo extienda o mejore según sus necesidades particulares o comunes
de una comunidad \cite{hardroid2016b}.

\section{Trabajos Futuros}

\label{trabajos-futuros}

Luego de la experiencia obtenida y teniendo en cuenta la amplia aplicabilidad
del reconocimiento de actividades y sobre todo en dispositivos móviles:
\begin{enumerate}
\item Poder identificar más actividades en otros contextos. 
\item Segmentar los grupos de individuos distintos por rangos de edad, sexo
y/o factores fisiológicos y generar modelos distintos para cada grupo. 
\item Realizar una comparación de distintos métodos de aprendizaje automático. 
\item Incluir más variables en el procesamiento de la señal teniendo en
cuenta la orientación del dispositivo. 
\item Utilizar un reloj inteligente (Smartwatch) u otros accesorios que
agreguen variables relevantes en el reconocimiento de actividades
dependiendo del contexto. 
\end{enumerate}

