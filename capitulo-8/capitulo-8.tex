\chapter{Conclusiones y Trabajos Futuros}

\subsection{Conclusiones}


\subsection{Trabajos Futuros}
Luego de la experiencia obtenida y teniendo en cuenta la amplia aplicabilidad del reconocimiento de actividades y sobre todo en dispositivos moviles:
\begin{enumerate}
\item Segmentación del modelo generado en distintos rangos de edad, y sexo de las personas.
\item Realizar una comparación de distintos métodos de aprendizaje automático. 
\item Utilizar más variables para mejorar las predicciones y teniendo en cuenta las posición del dispositivo.
\item Identificar otras actividades aplicado a otras disciplinas.
\item <Agregar un 5to trabajo futuro>
\end{enumerate}

Se propone generar modelos segmentados por distintas características de las personas, como ser rango de edades, sexo, etc. Teniendo en cuenta que la motricidad para en las diferentes edades por ejemplo varía para distintas edades o sexo, comprobando si esto crea modelos más preciso, y que cada usuario disponga en su dispositivo móvil el modelo acorde. La segmentación podría llegar a aumentarse hasta tener un modelo por persona.

Entre los métodos de aprendizaje automático existen varias opciones por explorar, como ser las principales el SVM, Redes Neuronales, o kNN, explorando cual obtiene mejores resultados en contrapartida con el consumo de batería del dispositivo movil.

En el proceso de selección de variables estadísticas se debería explorar en detalle un mayor numero de ella, como las tres dimensiones del movimiento para obtener mejores resultados en las predicciones. 

Dentro de las actividades definidas en el modelo de este trabajo pueden extenderse a otras que podrían aplicarse a otro tipo de aplicaciones, como se la medica, militar, etc. Para aplicaciones medicas actividades relevantes serian las posturas del cuerpo como ser parado, sentado, acostado. Esto propone un sin numero de aplicaciones del modelo de reconocimiento de actividades. 

De la misma manera de las actividades, se propone complementar el modelo con diferentes tipos de dispositivos inteligentes como ser un reloj Inteligente, algún tipo de sensor que dé mayor información del contexto de la persona. Esto va muy relacionado al punto anterior, ya que algunas actividades o posturas podrían requerir mayores datos para realizar el reconocimiento.








