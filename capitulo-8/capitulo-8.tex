
\chapter{Conclusiones y Trabajos Futuros}

\label{conclusiones-y-trabajos-futuros}

\section{Conclusiones}

\label{conclusiones}

El reconocimiento de actividades humanas es un área de investigación
multidisciplinaria con constantes aportes y diversos usos en la actualidad
debido a su relevancia en la computación contextual y ubicua. Los
avances en la miniaturización de los sistemas de computación y sensores
hacen del área tanto atractiva como desafiante para explorar y proponer
nuevas ideas en este ámbito \cite{LaraLabrador2013}. Sin embargo,
los esfuerzos realizados por otros trabajos no han tenido en cuenta
la dimensión de la colaboración y el uso de Internet para contribuir
a la mejora de los sistemas de reconocimiento, tal como se ha hecho
este trabajo.

El problema planteado en la sección \ref{objetivo-general} de diseñar
un sistema de reconocimiento de actividades colaborativo fue abordado
con la utilización de tres elementos principales para lograr el objetivo
principal: los teléfonos móviles, el software libre y la Internet.
Los primeros cuatro objetivos específicos (\ref{enu:obe1} al \ref{enu:obe4})
descritos en la sección \ref{objetivos-especuxedficos} son cuestiones
del marco teórico en el estudio de HAR donde no se hace una contribución
específica. Los objetivos específicos \ref{enu:obe5} y \ref{enu:obe6}
son el aporte principal de este trabajo y completan la solución al
problema planteado.

Primeramente, el problema de recolectar datos en condiciones realistas
fue abordado por medio de una aplicación de captura de datos en forma
de encuesta. La aplicación construida para los teléfonos móviles fue
publicada para que las personas su descarga y colaborar aportando
los datos de sensores y las actividades reconocidas mientras utilizan
el sistema de reconocimiento. Periodicamente, con el objetivo de retroalimentar
el modelo de aprendizaje automático, la información recolectada se
sube a Internet para que pueda servir de entrada para entrenar de
nuevo el modelo con actualizaciones constantes.

Además, se construyo el sistema de reconocimiento de actividades,
que es el aporte principal de este trabajo, el mismo está disponible
como un proyecto de software libre para la plataforma \emph{Android.}
Este puede ser utilizado de dos maneras, otras aplicaciones móviles
pueden incluirla en tiempo de construcción o simplemente pueden utilizar
el sistema como un servicio a través de una API de integración \cite{hardroid2016a}.

Finalmente, la colaboración planteada se desenvuelve en dos aspectos
principales. En el aspecto de los datos recolectados para el modelo
de aprendizaje automático donde las personas comparten información
relevante al utilizar la aplicación de encuesta y el sistema de reconocimiento
puede actualizar su modelo periódicamente. En el aspecto del componente
de software donde la implementación del sistema reconocedor está disponible
para que cualquiera comunidad lo extienda o utilice según sus necesidades
particulares \cite{hardroid2016b}.

\section{Trabajos Futuros}

\label{trabajos-futuros}

Luego de la experiencia obtenida y teniendo en cuenta la amplia aplicabilidad
del reconocimiento de actividades y sobre todo en dispositivos móviles:
\begin{enumerate}
\item Poder identificar más actividades en otros contextos. 
\item Segmentar los grupos de individuos distintos por rangos de edad, sexo
y/o factores fisiológicos y generar modelos distintos para cada grupo. 
\item Realizar una comparación de distintos métodos de aprendizaje automático. 
\item Incluir más variables en el procesamiento de la señal teniendo en
cuenta la orientación del dispositivo. 
\item Utilizar un reloj inteligente (Smartwatch) u otros accesorios que
agreguen variables relevantes en el reconocimiento de actividades
dependiendo del contexto. 
\end{enumerate}

