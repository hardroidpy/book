
\chapter{Conclusiones}

\label{conclusiones-y-trabajos-futuros}

\section{Logros}

\label{conclusiones}

El reconocimiento de actividades humanas es un área de investigación
multidisciplinaria con constantes aportes y diversos usos en la actualidad
debido a su relevancia en la computación contextual y ubicua. Los
avances en la miniaturización de los sistemas de computación y sensores
hacen del área tanto atractiva como desafiante para explorar y proponer
nuevas ideas en este ámbito \cite{LaraLabrador2013}. Sin embargo,
los esfuerzos realizados por otros trabajos no han tenido en cuenta
la dimensión de la colaboración y el uso de Internet para contribuir
a la mejora de los sistemas de reconocimiento tal como se ha hecho
en este trabajo. Los logros más relevantes se describen a continuación.

Acorde al objetivo principal descrito en la sección \ref{sec13:objetivo-general}
se construyó un sistema de reconocimiento de actividades colaborativo
utilizando tres elementos principales: los teléfonos móviles modernos,
el software libre y la Internet. El desarrollo del trabajo se llevó
acabo en dos etapas. La primera etapa consistió en el entrenamiento
de un modelo y la segunda etapa consistió en la construcción de un
sistema reconocedor de actividades para teléfonos móviles.

De acuerdo al marco teórico de esta área enunciado en los objetivos
específicos (\ref{enu:obe1} al \ref{enu:obe4}) de la sección \ref{sec13:objetivos-especuxedficos}
ambas etapas siguen las guías comunes para reconocer actividades humanas
y son:
\begin{enumerate}
\item Recolección de datos de sensores
\item Procesamiento de señal en ventanas de tiempo
\item Extracción de muestras y variables características
\end{enumerate}
La primera etapa se caracterizó por la construcción de un clasificador
basado en aprendizaje automático y la exportación del mismo a un modelo
ejecutable en \emph{Java}.

Los objetivos específicos \ref{enu:obe5} y \ref{enu:obe6} engloban
el aporte principal de este trabajo el cual se desenvolvió en la segunda
etapa. Se construyó el sistema de reconocimiento de actividades para
teléfonos móviles para la plataforma \emph{\abbr{Android}} siguiendo
las guías comunes y utilizando el resultado de la primera etapa. Este
sistema puede ser utilizado de dos maneras, otras aplicaciones móviles
pueden incluirla en tiempo de construcción o simplemente ejecutarlo
como un servicio a través de una interfaz de integración \cite{GimenezYegros2016a}.

En parte, para evaluar el sistema construido y además recolectar datos
de sensores en condiciones realistas se construyó una aplicación que
reconoce actividades y realiza una encuesta en línea. Esta aplicación
\emph{\abbr{Android}} construida fue publicada para que los participantes
del estudio la descarguen y colaboren aportando los datos asociados
a las actividades reconocidas mientras utilizan el sistema de reconocimiento.
El objetivo de este ejercicio es retro-alimentar el modelo de aprendizaje
automático periódicamente. La información recolectada se comparte
a través de Internet para que se evalúe y eventualmente sirva como
entrada para construir un nuevo el modelo en etapas posteriores.

Finalmente, la colaboración se desenvuelve en dos aspectos principales.
En el aspecto de la información recolectada para evaluar el modelo
de aprendizaje automático al utilizar la aplicación de encuesta y
la actualización del modelo del sistema de reconocimiento de manera
periódica. En el aspecto del componente de software construido, donde
la implementación del sistema reconocedor está disponible para que
cualquiera lo extienda o utilice según sus necesidades particulares
\cite{GimenezYegros2016b}.

\section{Trabajos Futuros}

\label{trabajos-futuros}

Luego de la experiencia obtenida y teniendo en cuenta la amplia aplicabilidad
del reconocimiento de actividades y sobre todo en dispositivos móviles:
\begin{enumerate}
\item Segmentar los grupos de individuos distintos por rangos de edad, sexo
y/o factores fisiológicos y generar modelos distintos para cada grupo. 
\item Realizar una comparación de distintos métodos de aprendizaje automático. 
\item Utilizar más variables Incluir más variables en el procesamiento de
la señal para mejorar las predicciones teniendo en cuenta la orientación
del dispositivo. 
\item Identificar otras actividades aplicado a otras disciplinas o contextos. 
\item Utilizar un reloj inteligente (\emph{Smartwatch}) u otros accesorios
que agreguen variables relevantes en el reconocimiento de actividades
dependiendo del contexto. 
\end{enumerate}
Se propone generar modelos segmentados por distintas características
de las personas, como ser rango de edades, sexo, etc. Teniendo en
cuenta que la motricidad para en las diferentes edades por ejemplo
varía para distintas edades o sexo, comprobando si esto crea modelos
más preciso, y que cada usuario disponga en su dispositivo móvil el
modelo acorde. La segmentación podría llegar a aumentarse hasta tener
un modelo por persona.

Entre los métodos de aprendizaje automático existen varias opciones
por explorar, como las técnicas de (\abbr{SVM}, \emph{Support Vector
Machine}) o Redes Neuronales, explorando cual obtiene mejores resultados
en contrapartida con el consumo de batería del dispositivo móvil.

En el proceso de selección de variables estadísticas se debería explorar
en detalle un mayor numero de ella, como las tres dimensiones del
movimiento para obtener mejores resultados en las predicciones. 

Dentro de las actividades definidas en el modelo de este trabajo pueden
extenderse a otras que podrían aplicarse a otro tipo de aplicaciones,
como ser la médica, militar, etc. Para aplicaciones médicas actividades
relevantes serian las posturas del cuerpo como ser parado, sentado,
acostado. Esto propone un sin numero de aplicaciones del modelo de
reconocimiento de actividades. 

De la misma manera de las actividades, se propone complementar el
modelo con diferentes tipos de dispositivos inteligentes como ser
un reloj inteligente, algún tipo de sensor que dé mayor información
del contexto de la persona. Esto va muy relacionado al punto anterior,
ya que algunas actividades o posturas podrían requerir mayores datos
para realizar el reconocimiento.
