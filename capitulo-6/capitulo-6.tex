
\chapter{Evaluación y Resultados}

\label{chap6:evaluacion}

\section{Introducción}

En el ámbito de investigación y desarrollo de sistemas \abbr{HAR}
existen dos elementos vitales: la recolección de datos experimentales
y generación del conjunto de entrenamiento. En este capítulo, describimos
de manera general estos elementos en las primeras dos secciones. La
sección \ref{sec6:recoleccion} describe los aspectos relacionados
a la captura de datos, determinando los requisitos mínimos de los
teléfonos móviles utilizados y describiendo el procedimiento guía
para experimentación. Además, la sección \ref{sec6:clasificacion}
describe los resultados obtenidos al aplicar las técnicas descritas
en la sección \ref{sec44:proceso-se=0000F1ales} para transformar
datos sensoriales a un conjunto de entrenamiento para la clasificación
de actividades humanas. Finalmente, en la sección \ref{sec6:resultados}
los resultados producidos al utilizar el conjunto entrenamiento disponible
son presentados. Esto incluye la validación del modelo construido
con el algoritmo de \emph{Machine Learning} (\abbr{ML}) escogido
que confirma su usabilidad en ambientes productivos. También se expone
la evaluación de los resultados producidos por \emph{\abbr{HARDroid}
}con la aplicación \emph{ActivitySurvey} desarrollada para esta tarea.

\section{Datos Experimentales}

\label{sec6:recoleccion}En base a trabajos precedentes en sistemas
\abbr{HAR}, se han dispuesto datos experimentales para entrenar clasificadores
de actividades humanas tales como en \cite{ReyesOrtiz2013}. Estos
datos están disponibles como fuente para diversos estudios de investigaciones
en este ámbito. 

Sin embargo, los datos de sensores capturados con teléfonos móviles
son escasos. Es por esta razón que este trabajo requirió realizar
una colecta acorde a los objetivos de estudio del mismo. El procedimiento
y los datos recolectados por medio de experimentación se describen
a continuación.

\subsection{Instrumentación}

Como es sabido el desarrollo de \emph{\abbr{HARDroid} }se concreto
completamente para la plataforma \emph{\abbr{Android} }donde aquí
se detallan los rasgos técnicos de las herramientas utilizadas para
su concepción y evaluación.

\subsubsection{Teléfonos inteligentes}

Escoger las herramientas apropiadas para desarrollar un sistema \abbr{HAR}
requiere de la evaluación de dispositivos móviles disponibles en el
mercado teniendo en cuenta los criterios citados en la \secref{sec24:dispositivos-moviles}:
\emph{hardware}, sensores y software de plataforma. En el periodo
de evaluación de este trabajo (2016), la cantidad de teléfonos inteligentes
con sensores de aceleración fue vasta, algunos de los cuales se listan
en la \tabref{tab6:dispositivos} con sus características relevantes.

\begin{table}[h]
\begin{centering}
\begin{tabular}{|l|l|l|l|l|}
\hline 
Marca/modelo & CPU & RAM/ROM & Sensor\footnote{Sensor de aceleración} & Android\tabularnewline
\hline 
\hline 
{\small{}LG G2} & {\small{}1.2GHz Cortex-A7} & {\small{}1GB/8GB} & {\small{}BMI160} & {\small{}Nougat 7.0}\tabularnewline
\hline 
{\small{}LG Nexus 5X} & {\small{}Q1.4Ghz Cortex-A53} & {\small{}2GB/32GB} & {\small{}BMC150} & {\small{}Lollipop 5.0.2}\tabularnewline
\hline 
{\small{}Motorola G 2nd} & {\small{}1.2GHz Cortex-A7} & {\small{}1GB/8GB} & {\small{}3-axis Acc} & {\small{}Marshmallow 6.0}\tabularnewline
\hline 
{\small{}Huawei Mate 9} & {\small{}Q2.4GHz Cortex-A73} & {\small{}4GB/64GB} & {\small{}LSM6DSM} & {\small{}Nougat 7.0}\tabularnewline
\hline 
{\small{}Huawei Mate 8} & {\small{}Q2.3GHz Cortex-A72} & {\small{}3GB/32GB} & {\small{}LSM330 3-axis} & {\small{}Marshmallow 6.0}\tabularnewline
\hline 
{\small{}Samsung S6} & {\small{}Q2.1GHz Cortex-A57} & {\small{}3GB/32GB} & {\small{}MPU6500} & {\small{}Nougat 7.0}\tabularnewline
\hline 
{\small{}Samsung A5} & {\small{}Q1.2GHz Cortex-A53 } & {\small{}2GB/16GB} & {\small{}BOSCH Acc} & {\small{}Lollipop 5.1.1}\tabularnewline
\hline 
\end{tabular}
\par\end{centering}
\caption[Especificaciones de los teléfonos inteligentes]{\label{tab6:dispositivos}Especificaciones de los teléfonos inteligentes
de entrenamiento.}
\end{table}

La elección fue en base a los dispositivos móviles disponibles, propiedad
de los voluntarios durante las sesiones de experimento.

\subsubsection{Entorno de desarrollo }

Las aplicaciones móviles desarrolladas en este trabajo están enteramente
construidas en la plataforma \emph{\abbr{Android}} donde fueron utilizados
los programas destinados para el caso \cite{Android2016}:
\begin{itemize}
\item \emph{Android Studio}: Entorno de desarrollo integrado para proyectos\emph{
}de software.
\item \emph{Android} \emph{Software Development Kit }(\abbr{SDK}): Herramientas
y librerías \abbr{API} requeridas para construir aplicaciones.
\item \emph{Gradle}: Programa de automatización de tareas de construcción
de aplicaciones.
\end{itemize}
Las aplicaciones desarrolladas fueron escritas utilizando el lenguaje
\emph{Java }principalmente. Tanto las interfaces de usuario, servicios
de aplicación y las tareas de computación intensivas de acceso a sensores,
procesamiento, algoritmos de \abbr{ML} y almacenamiento de datos
fueron hechos integramente en Java.

\subsection{Procedimiento Guía }

Con el objetivo de obtener un conjunto de datos adecuado a este estudio
de \abbr{HAR} se realizó un experimento de entrenamiento con grupo
de voluntarios. Un grupo de 8 personas entre las edades de 20 y 38
estuvieron dispuestos para esta tarea donde la edad media de la población
esta comprendida en $30.5\pm5$ años. 

El procedimiento guía de captura de datos se instruyó con el uso del
teléfono móvil como prenda sujeta al bolsillo o en la cintura mientras
se realiza una actividad física predeterminada. El planeamiento del
experimento consitió en realizar en orden por un periodo de 10 a 15
minutos las tres actividades básicas y dos de transporte:
\begin{itemize}
\item Caminar (WALKING)
\item Trotar (RUNNING)
\item Quieto (STILL)
\item En bicicleta (ON\_BICYCLE)
\item En auto (ON\_VEHICLE)
\end{itemize}
El experimento de entrenamiento se resume en la \tabref{tab6:sesiones},
donde se incluyen las sesiones de los voluntarios y el tiempo en minutos
invertido en la actividad etiquetada. Cada persona realizo una sesión
de entrenamiento al menos una vez.

\begin{table}[h]
\begin{centering}
\begin{tabular}{|c|c|c|c|c|c|}
\hline 
Sujeto & WALKING & RUNNING & STILL & ON\_BICYLE & ON\_VEHICLE\tabularnewline
\hline 
\hline 
AG & 80 & 33 & 12 & 17 & 7\tabularnewline
\hline 
SG & 15 & 15 & - & - & -\tabularnewline
\hline 
SF & 43 & 25 & - & 13 & -\tabularnewline
\hline 
GA & 30 & 2 & - & - & -\tabularnewline
\hline 
BV & - & 17 & - & - & -\tabularnewline
\hline 
PV & 17 & 19 & 12 & - & 7\tabularnewline
\hline 
SY & 37 & 13 &  & 26 & 11\tabularnewline
\hline 
MD & 29 & 4 & - & - & -\tabularnewline
\hline 
\end{tabular}
\par\end{centering}
\caption{\label{tab6:sesiones}Resumen de sesiones de entrenamiento}
\end{table}


\subsection{Captura de Datos}

TODO:

\section{Generación del Clasificador HAR}

\label{sec6:clasificacion}

\subsection{Etiquetado}

\subsection{Conjunto Entrenamiento }

\subsection{Clasificador de WEKA}

\section{Resultados}

\label{sec6:resultados}Describir resultados de experimento con HARDroid.
