
\chapter{Evaluación y Resultados}

\label{chap6:evaluacion}

\section{Introducción}

En el ámbito de investigación y desarrollo de sistemas \abbr{HAR}
existen dos elementos vitales: la recolección de datos experimentales
y generación del conjunto de entrenamiento. En este capítulo, describimos
de manera general estos elementos organizados en dos secciones. La
sección \ref{sec6:recoleccion} describe los aspectos relacionados
a la captura de datos, determinando los requisitos mínimos de los
teléfonos móviles utilizados para la experimentación y describiendo
el procedimiento guía de pruebas para un grupo de voluntarios. Además,
la sección \ref{sec6:clasificacion} describe el resultado de aplicar
las técnicas descritas en la sección \ref{sec44:proceso-se=0000F1ales}
para manejar datos sensoriales y generar un conjunto de entrenamiento
para caracterización de datos. Finalmente, en la sección \ref{sec6:resultados}
los resultados obtenidos utilizando los datos disponibles son presentados.
Esto incluye la validación de datos con el algoritmo de \emph{Machine
Learning} (\abbr{ML}) escogido que confirma su usabilidad en ambientes
productivos. También se demuestra la evaluación de los datos producidos
por \emph{HARDroid }en conjunto con la aplicación de prueba \emph{ActivitySurvey}
desarrollada para este trabajo.

\section{Recolección de Datos}

\label{sec6:recoleccion}

\subsection{Instrumentación}

\subsubsection{Teléfonos inteligentes}

\subsubsection{Entorno de desarrollo }

\subsection{Procedimiento Guía }

\subsection{Captura de Datos}

\section{Generación del Clasificador \abbr{HAR}}

\label{sec6:clasificacion}

\subsection{Etiquetado}

\subsection{Conjunto Entrenamiento }

\subsection{Clasificador de \abbr{WEKA}}

\section{Resultados }

\label{sec6:resultados}
