
\chapter{Evaluación y Resultados}

\label{chap6:evaluacion}

\section{Introducción}

En el ámbito de investigación y desarrollo de sistemas \abbr{HAR}
existen dos elementos vitales: la recolección de datos experimentales
y generación del conjunto de entrenamiento. En este capítulo, describimos
de manera general estos elementos en las primeras dos secciones. La
sección \ref{sec6:recoleccion} describe los aspectos relacionados
a la captura de datos, determinando los requisitos mínimos de los
teléfonos móviles utilizados y describiendo el procedimiento guía
para experimentación. Además, la sección \ref{sec6:clasificacion}
describe los resultados obtenidos al aplicar las técnicas descritas
en la sección \ref{sec44:proceso-se=0000F1ales} para transformar
datos sensoriales a un conjunto de entrenamiento para la clasificación
de actividades humanas. Finalmente, en la sección \ref{sec6:resultados}
los resultados producidos al utilizar el conjunto entrenamiento disponible
son presentados. Esto incluye la validación del modelo construido
con el algoritmo de \emph{Machine Learning} (\abbr{ML}) escogido
que confirma su usabilidad en ambientes productivos. También se expone
la evaluación de los resultados producidos por \emph{HARDroid }con
la aplicación \emph{ActivitySurvey} desarrollada para esta tarea.

\section{Datos Experimentales}

\label{sec6:recoleccion}En base a trabajos de investigación en \abbr{HAR}
realizados, se han dispuesto algunos datos experimentales para entrenar
clasificadores de actividades humanas \cite{ReyesOrtiz2013,Anguita2013}.
Estos datos están libremente disponibles como origen para estudios
de distintas disciplinas e investigaciones en este ámbito. 

Sin embargo, aún los datos de sensores capturados con teléfonos móviles
son limitados. Es por eso, que en este trabajo se requirió realizar
una colecta de datos personalizada y adecuada a los objetivos de este
estudio. Los datos recolectados por medio de experimentación se describen
a continuación.

\subsection{Instrumentación}

\subsubsection{Teléfonos inteligentes}

\subsubsection{Entorno de desarrollo }

\subsection{Procedimiento Guía }

\subsection{Captura de Datos}

\section{Generación del Clasificador HAR}

\label{sec6:clasificacion}

\subsection{Etiquetado}

\subsection{Conjunto Entrenamiento }

\subsection{Clasificador de WEKA}

\section{Resultados}

\label{sec6:resultados}Describir resultados de experimento con HARDroid.
