
\chapter{Evaluación y Resultados}

\label{chap6:evaluacion}

\section{Introducción}

En el ámbito de investigación y desarrollo de sistemas \abbr{HAR}
existen dos elementos vitales: la recolección de datos experimentales
y generación del conjunto de entrenamiento. En este capítulo, describimos
de manera general estos elementos en las primeras dos secciones. La
sección \ref{sec6:recoleccion} describe los aspectos relacionados
a la captura de datos, determinando los requisitos mínimos de los
teléfonos móviles utilizados y describiendo el procedimiento guía
para experimentación. Además, la sección \ref{sec6:clasificacion}
describe los resultados obtenidos al aplicar las técnicas descritas
en la sección \ref{sec44:proceso-se=0000F1ales} para transformar
datos sensoriales a un conjunto de entrenamiento para la clasificación
de actividades humanas. Finalmente, en la sección \ref{sec6:resultados}
los resultados producidos al utilizar el conjunto entrenamiento disponible
son presentados. Esto incluye la validación del modelo construido
con el algoritmo de \emph{Machine Learning} (\abbr{ML}) escogido
que confirma su usabilidad en ambientes productivos. También se expone
la evaluación de los resultados producidos por \emph{\abbr{HARDroid}
}con la aplicación \emph{ActivitySurvey} desarrollada para esta tarea.

\section{Datos Experimentales}

\label{sec6:recoleccion}En base a trabajos precedentes en sistemas
\abbr{HAR}, se han dispuesto datos experimentales para entrenar clasificadores
de actividades humanas tales como en \cite{ReyesOrtiz2013}. Estos
datos están disponibles como fuente para diversos estudios de investigaciones
en este ámbito. 

Sin embargo, los datos de sensores capturados con teléfonos móviles
son escasos. Es por esta razón que este trabajo requirió realizar
una colecta acorde a los objetivos de estudio del mismo. El procedimiento
y los datos recolectados por medio de experimentación se describen
a continuación.

\subsection{Instrumentación}

Como es sabido el desarrollo de \emph{\abbr{HARDroid} }se concreto
completamente para la plataforma \emph{\abbr{Android} }donde aquí
se detallan los rasgos técnicos de las herramientas utilizadas para
su concepción y evaluación.

\subsubsection{Teléfonos inteligentes}

Escoger las herramientas apropiadas para desarrollar un sistema \abbr{HAR}
requiere de la evaluación de dispositivos móviles disponibles en el
mercado teniendo en cuenta los criterios citados en la \secref{sec24:dispositivos-moviles}:
\emph{hardware}, sensores y software de plataforma. En el periodo
de evaluación de este trabajo (2016), la cantidad de teléfonos inteligentes
con sensores de aceleración fue vasta, algunos de los cuales se listan
en la \tabref{tab6:dispositivos} con sus características relevantes.

\begin{table}[h]
\begin{centering}
\begin{tabular}{|l|l|l|l|l|}
\hline 
Marca/modelo & CPU & RAM/ROM & Sensor\footnote{Sensor de aceleración} & Android\tabularnewline
\hline 
\hline 
{\small{}LG G2} & {\small{}1.2GHz Cortex-A7} & {\small{}1GB/8GB} & {\small{}BMI160} & {\small{}Nougat 7.0}\tabularnewline
\hline 
{\small{}LG Nexus 5X} & {\small{}Q1.4Ghz Cortex-A53} & {\small{}2GB/32GB} & {\small{}BMC150} & {\small{}Lollipop 5.0.2}\tabularnewline
\hline 
{\small{}Motorola G 2nd} & {\small{}1.2GHz Cortex-A7} & {\small{}1GB/8GB} & {\small{}3-axis Acc} & {\small{}Marshmallow 6.0}\tabularnewline
\hline 
{\small{}Huawei Mate 9} & {\small{}Q2.4GHz Cortex-A73} & {\small{}4GB/64GB} & {\small{}LSM6DSM} & {\small{}Nougat 7.0}\tabularnewline
\hline 
{\small{}Huawei Mate 8} & {\small{}Q2.3GHz Cortex-A72} & {\small{}3GB/32GB} & {\small{}LSM330 3-axis} & {\small{}Marshmallow 6.0}\tabularnewline
\hline 
{\small{}Samsung S6} & {\small{}Q2.1GHz Cortex-A57} & {\small{}3GB/32GB} & {\small{}MPU6500} & {\small{}Nougat 7.0}\tabularnewline
\hline 
{\small{}Samsung A5} & {\small{}Q1.2GHz Cortex-A53 } & {\small{}2GB/16GB} & {\small{}BOSCH Acc} & {\small{}Lollipop 5.1.1}\tabularnewline
\hline 
\end{tabular}
\par\end{centering}
\caption[Especificaciones de los teléfonos inteligentes]{\label{tab6:dispositivos}Especificaciones de los teléfonos inteligentes
de entrenamiento.}
\end{table}

La elección fue en base a los dispositivos móviles disponibles, propiedad
de los voluntarios durante las sesiones de experimento.

\subsubsection{Entorno de desarrollo }

Las aplicaciones móviles desarrolladas en este trabajo están enteramente
construidas en la plataforma \emph{\abbr{Android}} donde fueron utilizados
los programas destinados para el caso \cite{Android2016}:
\begin{itemize}
\item \emph{Android Studio}: Entorno de desarrollo integrado para proyectos\emph{
}de software.
\item \emph{Android} \emph{Software Development Kit }(\abbr{SDK}): Herramientas
y librerías \abbr{API} requeridas para construir aplicaciones.
\item \emph{Gradle}: Programa de automatización de tareas de construcción
de aplicaciones.
\end{itemize}
Las aplicaciones desarrolladas fueron escritas utilizando el lenguaje
\emph{Java }principalmente. Tanto las interfaces de usuario, servicios
de aplicación y las tareas de computación intensivas de acceso a sensores,
procesamiento, algoritmos de \abbr{ML} y almacenamiento de datos
fueron hechos integramente en Java.

\subsection{Procedimiento Guía }

Con el objetivo de obtener un conjunto de datos adecuado a este estudio
de \abbr{HAR} se realizó un experimento de entrenamiento con grupo
de voluntarios. Un grupo de 8 personas entre las edades de 20 y 38
estuvieron dispuestos para esta tarea donde la edad media de la población
esta comprendida en $30.7\pm5$ años. 

El procedimiento guía de captura de datos se instruyó con el uso del
teléfono móvil como prenda sujeta al bolsillo o en la cintura mientras
se realiza una actividad física predeterminada. El planeamiento del
experimento consitió en realizar en orden por un periodo de 10 a 15
minutos alguna de las tres actividades básicas y dos de transporte.
Las actividades humanas con sus etiquetas correspondientes se listan
a continuación:
\begin{itemize}
\item Caminar (\emph{WALKING})
\item Trotar (\emph{RUNNING})
\item Estar quieto (\emph{STILL})
\item Andar en bicicleta (\emph{ON\_BICYCLE})
\item Andar en automóvil (\emph{ON\_VEHICLE})
\end{itemize}
El resultado del experimento se resume en la \tabref{tab6:sesiones},
aquí se incluyen las sesiones hechas por los voluntarios y el tiempo
en minutos invertido para cada actividad dada. 

\begin{table}[h]
\begin{centering}
\begin{tabular}{|c|c|c|c|c|c|}
\hline 
\multirow{2}{*}{Voluntario} & \multicolumn{5}{c|}{Actividades}\tabularnewline
\cline{2-6} 
 & \emph{WALKING} & \emph{RUNNING} & \emph{STILL} & \emph{ON\_BICYLE} & \emph{ON\_VEHICLE}\tabularnewline
\hline 
\hline 
AG & 80 & 33 & 12 & 17 & 7\tabularnewline
\hline 
SG & 15 & 15 & - & - & -\tabularnewline
\hline 
SF & 43 & 25 & - & 13 & -\tabularnewline
\hline 
GA & 30 & 2 & - & - & -\tabularnewline
\hline 
BV & - & 17 & - & - & -\tabularnewline
\hline 
PV & 17 & 19 & 12 & - & 7\tabularnewline
\hline 
SY & 37 & 13 &  & 26 & 11\tabularnewline
\hline 
MD & 29 & 4 & - & - & -\tabularnewline
\hline 
\end{tabular}
\par\end{centering}
\caption{\label{tab6:sesiones}Resumen de sesiones de entrenamiento}
\end{table}


\subsection{Captura de Datos}

El experimento requiere de una cantidad de datos recolectados de los
voluntarios mientras realizan las actividades humanas dictadas durante
el entrenamiento. \emph{SensorLog} \cite{Alan2014s} es una aplicación
\abbr{Android} que se utilizó para la captura y persistencia de las
señales del sensor de aceleración como se ve en la \figref{fig4:sensor-log}.
La aplicación permite exportar las señales de sensores almacenadas
en forma agrupada por sesiones y subirlas a la nube. En la \tabref{tab6:captura}
se resumen las cantidades de medidas de señales capturadas.

\begin{table}[h]
\begin{centering}
\begin{tabular}{|c|c|c|c|c|c|}
\hline 
 & \multicolumn{5}{c|}{Actividades}\tabularnewline
\hline 
Cantidades & \emph{WALKING} & \emph{RUNNING} & \emph{STILL} & \emph{ON\_BICYLE} & \emph{ON\_VEHICLE}\tabularnewline
\hline 
\hline 
Medidas & 3.591.788 & 1.604.270 & 387.735 & 819.992 & 365.814\tabularnewline
\hline 
\end{tabular}
\par\end{centering}
\caption{\label{tab6:captura}Resumen de datos capturados}
\end{table}


\subsection{Conjunto de Entrenamiento }

En la \tabref{tab6:muestras} se resumen las cantidades de las muestras
calculadas de acuerdo lo descrito en el proceso de la \secref{ssec44:extraction}.

\begin{table}[h]
\begin{centering}
\begin{tabular}{|c|c|c|c|c|c|}
\hline 
 & \multicolumn{5}{c|}{Actividades}\tabularnewline
\hline 
Cantidades & \emph{WALKING} & \emph{RUNNING} & \emph{STILL} & \emph{ON\_BICYLE} & \emph{ON\_VEHICLE}\tabularnewline
\hline 
\hline 
Muestras & 28.049 & 11.779 & 3.026 & 6.404 & 2.856\tabularnewline
\hline 
\end{tabular}
\par\end{centering}
\caption{\label{tab6:muestras}Resumen de muestras procesadas}
\end{table}


\section{Clasificador de Actividades Humanas}

\label{sec6:clasificacion}En esta sección se describe el procedimiento
de generación de un clasificador \abbr{HAR} utilizando el conjunto
de entrenamiento y sometiendo el mismo a prueba y verificación para
garantizar su usabilidad. Esta tarea se realiza utilizando la herramienta
\emph{Waikato Environment Knowledge Analysis} \texttt{(\abbr{WEKA})}
\cite{Frank2016}. 

\subsection{Clasificador de J48}

En el siguiente cuadro se muestra el resultado de árbol generado.
La evaluación de resultado es un árbol de X nodos y Y ramas .

Number of leaves: 1673 

Size of the tree: 3345

El clasificador generado tiene los siguientes resultados: 
\begin{itemize}
\item Correctly Classified Instances 45.086 86,5142 \% 
\item Incorrectly Classified Instances 7.028 13,4858 \% 
\item Kappa statistic 0,7871 
\item Mean absolute error 0,0653 
\item Root mean squared error 0,2187 
\item Relative absolute error 25,6021 \% 
\item Root relative squared error 61,2282 \% 
\item Total Number of Instances 52.114 
\end{itemize}
La evaluación del clasificador resultante se resume en la \tabref{tab6:matriz-confusion}.

\begin{table}[h]
\begin{centering}
\begin{tabular}{|c|c|c|c|c|c|}
\hline 
 & \multicolumn{5}{c|}{Matriz de Confusión}\tabularnewline
\hline 
Actividad & \emph{WALKING} & \emph{RUNNING} & \emph{STILL} & \emph{ON\_BICYLE} & \emph{ON\_VEHICLE}\tabularnewline
\hline 
\hline 
\emph{WALKING} & 25.334 & 888 & 74 & 1.640 & 113\tabularnewline
\hline 
\emph{RUNNING} & 1.103 & 10.545 & 13 & 115 & 3\tabularnewline
\hline 
\emph{STILL} & 97 & 23 & 2.557 & 74 & 275\tabularnewline
\hline 
\emph{ON\_BICYLE} & 2.018 & 106 & 40 & 4.144 & 96\tabularnewline
\hline 
\emph{ON\_VEHICLE} & 88 & 15 & 156 & 91 & 2.506\tabularnewline
\hline 
\end{tabular}
\par\end{centering}
\caption{\label{tab6:matriz-confusion}Matriz de confusión del clasificador
resultante}
\end{table}

Los valores calculados de las métricas en la \secref{sec3:metricas}.
\begin{itemize}
\item Precisión: 0,8651
\item Exhaustividad: 0,8651
\item Exactitud: 0,9461
\item Valor-F: 0,8651
\end{itemize}

\section{Resultados}

\label{sec6:resultados}Describir resultados de experimento con HARDroid.
