
\subsection{Alcance}
\begin{frame}{Alcance}

\framesubtitle{Propuesta}
\begin{center}
Reconocer actividades humanas utilizando tel�fonos m�viles inteligentes
con un enfoque colaborativo para el aprendizaje autom�tico
\par\end{center}

\end{frame}
%
\begin{frame}{Alcance}

\framesubtitle{Objetivo General}
\begin{center}
El objetivo principal del presente trabajo es implementar un sistema
de reconocimiento de actividades humanas con tel�fonos m�viles cuyo
principal aporte sea un componente de software en forma de librer�a
de c�digo abierto y libremente distribuido.
\par\end{center}

\end{frame}
%
\begin{frame}{Alcance}

\framesubtitle{Objetivos Espec�ficos}
\begin{enumerate}[<+->]
\item \setbeamercovered{transparent}Definir el marco te�rico sobre el
reconocimiento de actividades humanas (HAR). 
\item Comprender las t�cnicas de recolecci�n de datos en entornos restringidos
para bajo consumo energ�a. 
\item Comprender el procesamiento de se�ales de datos inerciales para identificar
variables significativas de entrenamiento. 
\item Comprender la clasificaci�n por aprendizaje autom�tico en entornos
restringidos para bajo consumo energ�a. 
\item Dise�ar un sistema de reconocimiento de actividades que se componga
de la recolecci�n de muestras colaborativas y predicci�n de actividades
humanas en-l�nea. 
\item Aportar un componente de software en forma de librer�a para uso en
tel�fonos m�viles modernos con la plataforma \emph{Android}. 
\end{enumerate}
\end{frame}

