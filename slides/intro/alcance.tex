
\subsection{Alcance}
\begin{frame}{Alcance}

\framesubtitle{Propuesta}
\begin{center}
Reconocer actividades humanas utilizando tel�fonos m�viles modernos
con un enfoque colaborativo
\par\end{center}

\end{frame}
%
\begin{frame}{Alcance}

\framesubtitle{Objetivo General}
\begin{center}
El objetivo principal es desarrollar un sistema \structure{HAR} utilizando
tel�fonos m�viles modernos cuyo aporte principal es un una librer�a
de c�digo abierto y un modelo activo.
\par\end{center}

\end{frame}
%
\begin{frame}{Alcance}

\framesubtitle{Objetivos Espec�ficos}

\setbeamercovered{transparent}
\begin{enumerate}[<+->]
\item Definir el marco te�rico sobre \structure{HAR}. 
\item Revisar las t�cnicas de \structure{captura de se�ales} bajo restricci�n
de consumo de energ�a. 
\item Revisar el \structure{procesamiento de se�ales} para identificar
variables de entrenamiento. 
\item Comprender la clasificaci�n basada \structure{aprendizaje autom�tico}
bajo restricci�n de consumo energ�a. 
\item Dise�ar un \structure{sistema HAR} que incluya la recolecci�n de
muestras colaborativas y clasificaci�n de actividades en-l�nea. 
\item Aportar un componente de \structure{librer�a} para tel�fonos m�viles
modernos con plataforma \emph{\textbf{\emph{Android}}}. 
\end{enumerate}
\end{frame}

