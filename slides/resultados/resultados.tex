
\section{Resultados}
\begin{frame}{Resultados}
\framesubtitle{Procedimiento Gu�a}
\setbeamercovered{transparent}
Para validar y verificar el clasificador se realizaron dos tipos de pruebas siguiendo los siguientes pasos:

\begin{itemize}[<+->]
	\item Creaci�n de un conjunto entrenamiento inicial para el clasificador de \structure{HARDroid} (\emph{off-line}).
	\item Evaluaci�n del reconocedor \structure{HARDroid} en comparaci�n con \emph{Sony Lifelog} (\emph{on-line}).
	\item Generaci�n de un nuevo conjunto entrenamiento \structure{colaborativo} cuyo clasificador es evaluado (\emph{on-line}).
\end{itemize}

\end{frame}
%
\begin{frame}{Resultados}
\framesubtitle{Entrenamiento inicial}
\setbeamercovered{transparent}

\begin{itemize}[<+->]
	\item Voluntarios de 8 personas entre 20 y 38 a�os.
	\item Captura y etiquetado supervisado por actividad de 2 a 15 minutos.
	\item Tiempo acumulado de entrenamiento de 516 minutos (8:36).
\end{itemize}

\begin{table}[h]
	\caption{Muestras etiquetadas en conjunto de entrenamiento inicial}
	\begin{centering}
		\begin{tabular}{|c|c|c|c|c|c|}
			\hline 
			\texttt{\footnotesize{WALKING}} & \texttt{\footnotesize{RUNNING}} & \texttt{\footnotesize{STILL}} & \texttt{\footnotesize{TILTING}} & \texttt{\footnotesize{ON\_BICYLE}} & \texttt{\footnotesize{IN\_VEHICLE}}\tabularnewline
			\hline 
			\hline 
			5.915 &	3.019 &	645 & 485 & 1.338 & 610\tabularnewline
			\hline 
			49\% & 25\% & 6\% & 4\% & 11\% & 5\%\tabularnewline
			\hline 
		\end{tabular}
		\par\end{centering}
\end{table}

\end{frame}
%
\begin{frame}{Resultados}
\framesubtitle{Validaci�n del Clasificador}

El rendimiento del clasificador se eval�a utilizando algunas m�tricas num�ricas de predicci�n:

\begin{itemize}
	\item N�mero de instancias ($N$).
	\item Correctamente clasificadas ($C$).
	\item Incorrectamente clasificadas ($I$).
	\item Estad�sticas Kappa ($\kappa$).
\end{itemize}

\begin{block}<2>{Definici�n}
Es el coeficiente kappa de Cohen que determina el valor de coeficiente 
de concordancia para confiabilidad de los datos.
\end{block}

\end{frame}
%
\begin{frame}{Resultados}
\framesubtitle{Validaci�n del Clasificador}

El costo del clasificador se eval�a utilizando algunas m�tricas num�ricas en t�rminos del error:

\begin{itemize}
	\item Error medio absoluto ($E_{ma}$).
	\item Ra�z de Error cuadr�tico medio ($E_{rms}$).
	\item Error absoluto relativo ($E_{ra}$).
	\item Ra�z de Error cuadr�tico relativo ($E_{rrs}$).
\end{itemize}

\end{frame}
%

\begin{frame}{Resultados}
\framesubtitle{Validaci�n del Clasificador ($\mathrm{A}$)}
\setbeamercovered{transparent}

El clasificador inicial ($\mathrm{A}$) es un �rbol de decisi�n 
 de \textbf{677} nodos con \textbf{339} hojas.


\begin{table}[H]
	\centering
	\caption{M�tricas del clasificador $\mathrm{A}$}
	\resizebox{5cm}{!}{
		\begin{tabular}{|c|c|}
			\hline 
			M�trica & Valor\tabularnewline
			\hline 
			\hline 
			$N$ & 12.012\tabularnewline
			\hline 
			$C$ & 10.942 (91,0922\%)\tabularnewline
			\hline 
			$I$ & 1.070 (8,9078\%) \tabularnewline
			\hline 
			$\kappa$ & 0,8678 \tabularnewline
			\hline 
			$E_{ma}$ & 0,0341 \tabularnewline
			\hline 
			$E_{rms}$ & 0,164 \tabularnewline
			\hline 
			$E_{ra}$ & 15,17\% \tabularnewline
			\hline 
			$E_{rrs}$ & 48,91\% \tabularnewline
			\hline 
		\end{tabular}
	}
\end{table}

\end{frame}
%
\begin{frame}{Resultados}
\framesubtitle{Validaci�n del Clasificador ($\mathrm{A}$)}

\begin{table}[h]
\centering
\caption{Matriz de confusi�n del clasificador generado}
\resizebox{11cm}{!}{
		\begin{tabular}{|l|c|c|c|c|c|c|}
			\cline{2-7} 
			\multicolumn{1}{l|}{} & \multicolumn{6}{c|}{Matriz de Confusi�n}\tabularnewline
			\hline 
			Actividad & \texttt{\footnotesize{WALKING}} & \texttt{\footnotesize{RUNNING}} & \texttt{\footnotesize{STILL}} & \texttt{\footnotesize{TILTING}} & \texttt{\footnotesize{ON\_BICYLE}} & \texttt{\footnotesize{IN\_VEHICLE}}\tabularnewline
			\hline 
			\hline 
			\texttt{\footnotesize{WALKING}} & \textbf{5.643} & 63 & 21 & 28 & 156 & 4\tabularnewline
			\hline 
			\texttt{\footnotesize{RUNNING}} & 122 & \textbf{2.852} & 3 & 6 & 36 & 0\tabularnewline
			\hline 
			\texttt{\footnotesize{STILL}} & 19 & 14 & \textbf{555} & 23 & 15 & 19\tabularnewline
			\hline 
			\texttt{\footnotesize{TILTING}} & 26 & 3 & 21 & \textbf{304} & 46 & 85\tabularnewline
			\hline 
			\texttt{\footnotesize{ON\_BICYLE}} & 157 & 28 & 7 & 47 & \textbf{1.089} & 10\tabularnewline
			\hline 
			\texttt{\footnotesize{IN\_VEHICLE}} & 2 & 2 & 14 & 86 & 7 & \textbf{499}\tabularnewline
			\hline 
		\end{tabular}
	}
\end{table}

\end{frame}
\begin{frame}{Resultados}
\framesubtitle{Validaci�n del Clasificador ($\mathrm{A}$)}

A partir de la matriz de confusi�n se calculan las m�tricas de evaluaci�n:
\begin{itemize}
    \item Exactitud: \textbf{0,9714}
	\item Precisi�n: \textbf{0,9174}
	\item Exhaustividad: \textbf{0,9109}
	\item \emph{Valor-F}: \textbf{0,9142}
\end{itemize}

\end{frame}
%
%
\begin{frame}{Resultados}

\framesubtitle{Verificaci�n del Clasificador ($\mathrm{A}$)}
\setbeamercovered{transparent}

\begin{table}[h]
\centering
\caption{Aciertos y desaciertos de actividades humanas detectadas}
	\resizebox{11cm}{!}{
		\begin{tabular}{|l|c|c|c|c|c|c|}
			\cline{2-5} 
			\multicolumn{1}{l|}{} & \multicolumn{4}{c|}{\textbf{Desaciertos}} & \multicolumn{1}{c}{} & \multicolumn{1}{c}{}\tabularnewline
			\hline 
			Actividad & \texttt{\footnotesize{WALKING}} & \texttt{\footnotesize{STILL}} & \texttt{\footnotesize{ON\_BICYLE}} & \texttt{\footnotesize{IN\_VEHICLE}} & \textbf{\small{Aciertos}} & \textbf{\small{Porcentaje}}\tabularnewline
			\hline 
			\hline 
			\texttt{\footnotesize{WALKING}} &  &  & 13 &  & 151 & 92,07\%\tabularnewline
			\hline 
			\texttt{\footnotesize{RUNNING}} &  &  & 13 &  & 83 & 86,46\%\tabularnewline
			\hline 
			\texttt{\footnotesize{STILL}} &  &  &  & 8 & 140 & 94,59\%\tabularnewline
			\hline 
			\texttt{\footnotesize{TILTING}}\emph{\footnotesize{}}\footnote{{\footnotesize{\scalebox{0.7}{Estar inquieto no se considera una actividad ambulatoria, present�ndose los datos de manera informativa}}}} &  &  &  & 38 & 0 & 0,00\%\tabularnewline
			\hline 
			\texttt{\footnotesize{ON\_BICYLE}} & 8 &  &  & 4 & 59 & 83,10\%\tabularnewline
			\hline 
			\texttt{\footnotesize{IN\_VEHICLE}} &  & 11 &  &  & 83 & 88,30\%\tabularnewline
			\hline 
		\end{tabular}
		}
\end{table}

\begin{block}{Nota}
Resultado de la encuesta con dos individuos.
\end{block}

\end{frame}

%
\begin{frame}{Resultados}
\framesubtitle{Verificaci�n del Clasificador ($\mathrm{A}$)}

\begin{table}[h]
\centering
\caption{\emph{HARDroid} vs \emph{Sony Lifelog}}
\resizebox{11cm}{!}{
		\begin{tabular}{|c|c|c|c|c|c|}
			\hline 
			\emph{Sony Lifelog} & Rango & Duraci�n & \emph{HARDroid} & Aciertos/Desaciertos & Porcentaje \tabularnewline
			\hline 
			\hline 
			\texttt{walking} & 21:26 - 21:44 & 19 & \texttt{WALKING} & 45/54
			& 83\%\tabularnewline
			\hline 
			\texttt{cycling} & 21:44 - 22:00 & 27 & \texttt{ON\_BICYCLE} & 32/40 
			& 80\%\tabularnewline
			\hline 
			\texttt{running} & 22:12 - 22:24 & 13 & \texttt{RUNNING} & 41/41
			& 100\%\tabularnewline
			\hline 
			\texttt{vehicle} & 22:39 - 22:57 & 19 & \texttt{IN\_VEHICLE} & 32/38
			& 84\%\tabularnewline
			\hline 
			\texttt{-} & 22:25 - 22:35 & 10 & \texttt{STILL} & 65/65
			& 100\% \tabularnewline
			\hline 
		\end{tabular}
		}
\end{table}

\begin{block}{Nota}
	Resultados de la encuesta con un individuo.
\end{block}

\end{frame}
%
%
\begin{frame}{Resultados}
\framesubtitle{Clasificador Colaborativo ($\mathrm{B}$)}

El clasificador colaborativo ($\mathrm{B}$) es un �rbol de decisi�n 
de \textbf{699} nodos con \textbf{350} hojas.

\begin{table}[H]
\centering
	\caption{Comparaci�n ($\mathrm{A}$) vs ($\mathrm{B}$)}
	\resizebox{10cm}{!}{
		\begin{tabular}{|c|c|c|c|c}
			\cline{1-4} 
			M�trica & Inicial ($\mathrm{A})$ & Colaborativo ($\mathrm{B})$ & Diferencia & \tabularnewline
			\cline{1-4} 
			\cline{1-4}
			$N$ & 12.012 & 12.578 
			& \textcolor{green}{+566}
			& $\checkmark$ \tabularnewline
			\cline{1-4}
			$C$ & 10.942 (91,09\%) & 11.489 (91,34\%)
			& \textcolor{green}{+0,25\%}
			& $\checkmark$ \tabularnewline
			\cline{1-4}
			$I$ & 1.070 (8,91\%) & 1.089 (8,66\%)
			& \textcolor{red}{-0,25\%}
			& $\checkmark$ \tabularnewline
			\cline{1-4}
			$\kappa$ & 0,8678 & 0,8735
			& \textcolor{green}{+0,57\%}
			& $\checkmark$ \tabularnewline
			\cline{1-4}
			$E_{ma}$ & 0,0341 & 0,0333
			&\textcolor{red}{-0,08\%}
			& $\checkmark$ \tabularnewline
			\cline{1-4}
			$E_{rms}$ & 0,164 & 0,1632
			&\textcolor{red}{-0,08\%}
			& $\checkmark$ \tabularnewline
			\cline{1-4}
			$E_{ra}$ & 15,17\% & 14,58\%
			&\textcolor{red}{-0,59\%}
			& $\checkmark$ \tabularnewline
			\cline{1-4}
			$E_{rrs}$ & 48,91\% & 48,30\%
			&\textcolor{red}{-0,61\%}
			& $\checkmark$ \tabularnewline
			\cline{1-4}
		\end{tabular}
		}
\end{table}

\end{frame}
%
%
\begin{frame}{Resultados}
\framesubtitle{Clasificador Colaborativo ($\mathrm{B}$)}

Las m�tricas de evaluaci�n del clasificador colaborativo resultan en:
\begin{itemize}
	\item Exactitud: \textbf{0,9723} (Antes 0,9714)
	\item Precisi�n: \textbf{0,9204} (Antes 0,9174)
	\item Exhaustividad: \textbf{0,9134} (Antes 0,9109)
	\item \emph{Valor-F}: \textbf{0,9168} (Antes 0,9142)
\end{itemize}
\end{frame}
%
%

