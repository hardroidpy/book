
\section{Resultados}
\begin{frame}{Procedimiento Gu�a}

\framesubtitle{Resultados}

\setbeamercovered{transparent}

Para validar y verificar el clasificador se realizaron dos tipos de
pruebas siguiendo los siguientes pasos:
\begin{itemize}[<+->]
\item Creaci�n de un conjunto entrenamiento inicial para el clasificador
de \structure{HARDroid} (\emph{off-line}, entrenamiento). 
\item Evaluaci�n del reconocedor \structure{HARDroid} en comparaci�n con
\emph{Sony Lifelog} (\emph{on-line}, validaci�n de aporte). 
\item A partir del uso se obtuvo un nuevo conjunto entrenamiento \structure{colaborativo}
a partir de encuestas recolectadas (\emph{on-line}, prueba de concepto). 
\end{itemize}
\end{frame}
%
\begin{frame}{Entrenamiento inicial}

\framesubtitle{Resultados}

\setbeamercovered{transparent}
\begin{itemize}[<+->]
\item Con 8 voluntarios entre las edades de 20 y 38 a�os. 
\item Sesi�n de captura y etiquetado supervisado para cada actividad de
2 a 15 minutos de duraci�n. 
\item Tiempo acumulado de entrenamiento de 516 minutos. 
\end{itemize}
\begin{table}[h]
\caption{Muestras etiquetadas en conjunto de entrenamiento inicial}

\centering{}%
\begin{tabular}{|c|c|c|c|c|c|}
\hline 
\texttt{\footnotesize{}WALKING}{\footnotesize{} } & \texttt{\footnotesize{}RUNNING}{\footnotesize{} } & \texttt{\footnotesize{}STILL}{\footnotesize{} } & \texttt{\footnotesize{}TILTING}{\footnotesize{} } & \texttt{\footnotesize{}ON\_BICYLE}{\footnotesize{} } & \texttt{\footnotesize{}IN\_VEHICLE}\tabularnewline
\hline 
\hline 
5.915  & 3.019  & 645  & 485  & 1.338  & 610\tabularnewline
\hline 
49\%  & 25\%  & 6\%  & 4\%  & 11\%  & 5\%\tabularnewline
\hline 
\end{tabular}
\end{table}

\end{frame}
%
\begin{frame}{Validaci�n del Clasificador}

\framesubtitle{Resultados}

El rendimiento del clasificador se eval�a utilizando m�tricas de predicci�n
propuestas en el estado del arte \emph{{[}Witten et al., 2017{]}}:
\begin{overprint}
\onslide<1> 
\begin{itemize}
\item N�mero de instancias ($N$). 
\item Correctamente clasificadas ($C$). 
\item Incorrectamente clasificadas ($I$). 
\item Estad�stica Kappa ($\kappa$). 
\end{itemize}
\onslide<2> 

\begin{table}
\caption{Estad�stica Kappa ($\kappa$)}
\centering\resizebox{0.7\textwidth}{!}{%
\begin{tabular}{|c|c|c|}
\hline 
Valor & Nivel de acuerdo & \% de datos confiables\tabularnewline
\hline 
\hline 
0 - 0.20 & No & 0 - 4\%\tabularnewline
\hline 
0.21 - 0.39 & M�nimo & 4 - 15\%\tabularnewline
\hline 
0.40 - 0.59 & D�bil & 15 - 35\%\tabularnewline
\hline 
0.60 - 0.79 & Moderado & 35 - 63\%\tabularnewline
\hline 
0.80 - 0.90 & Fuerte & 64 - 81\%\tabularnewline
\hline 
>0.90 & Casi perfecto & 82 - 100\%\tabularnewline
\hline 
\end{tabular}}
\end{table}

\end{overprint}
\begin{block}<2>{Coeficiente $\kappa$ de Cohen}

Coeficiente de concordancia para confiabilidad de los datos.
\end{block}
\end{frame}
%
\begin{frame}{Validaci�n del Clasificador}

\framesubtitle{Resultados}

El costo del clasificador se eval�a utilizando m�tricas de error propuestas
en el estado del arte \emph{{[}Witten et al., 2017{]}}:
\begin{columns}

\column{0.5\textwidth}
\begin{itemize}[<+->]
\item Error medio absoluto ($E_{ma}$). 
\item Ra�z de Error cuadr�tico medio ($E_{rms}$). 
\item Error absoluto relativo ($E_{ra}$). 
\item Ra�z de Error cuadr�tico relativo ($E_{rrs}$). 
\end{itemize}

\column{0.5\textwidth}
\begin{overprint}
\onslide<1> 

\[
E_{ma}=\frac{1}{n}\sum_{i\text{=1}}^{n}\bigl|y_{i}-f(x_{i})\bigr|
\]

\onslide<2> 

\[
E_{rms}=\sqrt{\frac{1}{n}\sum_{i\text{=1}}^{n}\left(y_{i}-f(x_{i})\right)^{2}}
\]

\onslide<3> 
\begin{center}
\[
E_{ra}=\sum_{i\text{=1}}^{n}\frac{\bigl|y_{i}-f(x_{i})\bigr|}{\bigl|y_{i}-\bar{y}\bigr|}
\]
con 
\[
\bar{y}=\frac{1}{n}\sum_{i=1}^{n}y_{i}
\]
\par\end{center}
\onslide<4> 
\begin{center}
\[
E_{rrs}=\sqrt{\sum_{i\text{=1}}^{n}\frac{\left(y_{i}-f(x_{i})\right)^{2}}{\left(y_{i}-\bar{y_{i}}\right)^{2}}}
\]
con 
\[
\bar{y}=\frac{1}{n}\sum_{i=1}^{n}y_{i}
\]
\par\end{center}

\end{overprint}
\end{columns}

\end{frame}
%
\begin{frame}{Validaci�n del Clasificador ($\mathrm{A}$)}

\framesubtitle{Resultados}

El clasificador inicial ($\mathrm{A}$) es un �rbol de decisi�n de
\textbf{677} nodos con \textbf{339} hojas con las siguientes caracter�sticas.

\begin{table}[H]
\caption{M�tricas del clasificador $\mathrm{A}$}
\centering\resizebox{5cm}{!}{ %
\begin{tabular}{|c|c|}
\hline 
M�trica  & Valor\tabularnewline
\hline 
\hline 
$N$  & 12.012\tabularnewline
\hline 
$C$  & 10.942 (91,0922\%)\tabularnewline
\hline 
$I$  & 1.070 (8,9078\%) \tabularnewline
\hline 
$\kappa$  & 0,8678 \tabularnewline
\hline 
$E_{ma}$  & 0,0341 \tabularnewline
\hline 
$E_{rms}$  & 0,164 \tabularnewline
\hline 
$E_{ra}$  & 15,17\% \tabularnewline
\hline 
$E_{rrs}$  & 48,91\% \tabularnewline
\hline 
\end{tabular}} 
\end{table}

\end{frame}
%
\begin{frame}{Validaci�n del Clasificador ($\mathrm{A}$)}

\framesubtitle{Resultados}

\begin{table}[h]
 \caption{Matriz de confusi�n del clasificador generado}
\centering\resizebox{11cm}{!}{ %
\begin{tabular}{l|c|c|c|c|c|c}
 & \texttt{\footnotesize{}WALKING} & \texttt{\footnotesize{}RUNNING}{\footnotesize{} } & \texttt{\footnotesize{}STILL} & \texttt{\footnotesize{}TILTING}{\footnotesize{} } & \texttt{\footnotesize{}ON\_BICYLE}{\footnotesize{} } & \texttt{\footnotesize{}IN\_VEHICLE}\tabularnewline
\hline 
\texttt{\footnotesize{}WALKING}{\footnotesize{} } & \textbf{5.643}  & 63  & 21  & 28  & 156  & 4\tabularnewline
\hline 
\texttt{\footnotesize{}RUNNING}{\footnotesize{} } & 122  & \textbf{2.852}  & 3  & 6  & 36  & 0\tabularnewline
\hline 
\texttt{\footnotesize{}STILL}{\footnotesize{} } & 19  & 14  & \textbf{555}  & 23  & 15  & 19\tabularnewline
\hline 
\texttt{\footnotesize{}TILTING} & 26  & 3  & 21  & \textbf{304}  & 46  & 85\tabularnewline
\hline 
\texttt{\footnotesize{}ON\_BICYLE}{\footnotesize{} } & 157  & 28  & 7  & 47  & \textbf{1.089}  & 10\tabularnewline
\hline 
\texttt{\footnotesize{}IN\_VEHICLE}{\footnotesize{} } & 2  & 2  & 14  & 86  & 7  & \textbf{499}\tabularnewline
\end{tabular}} 
\end{table}

\end{frame}
%
\begin{frame}{Validaci�n del Clasificador ($\mathrm{A}$)}

\framesubtitle{Resultados}

A partir de la matriz de confusi�n se calculan las m�tricas de evaluaci�n
propuestas en el estado del arte \emph{{[}Witten et al., 2017{]}}: 
\begin{columns}

\column{0.4\textwidth}
\begin{itemize}[<+->]
\item Exactitud: \textbf{0,9714} 
\item Precisi�n: \textbf{0,9174} 
\item Exhaustividad: \textbf{0,9109} 
\item \emph{Valor-F}: \textbf{0,9142} 
\end{itemize}

\column{0.6\textwidth}
\begin{overprint}
\onslide<1> 

\[
Exactitud=\frac{TP+TN}{TP+TN+FP+FN}
\]

\begin{center}
\emph{(rendimiento general)}
\par\end{center}
\onslide<2> 

\[
Precisi\acute{o}n=\frac{TP}{TP+FP}
\]

\begin{center}
\emph{(valor predictivo positivo)}
\par\end{center}
\onslide<3> 

\[
Exhaustividad=\frac{TP}{TP+FN}
\]

\begin{center}
\emph{(tasa positiva verdadera)}
\par\end{center}
\onslide<4> 

\[
ValorF=2\cdot\frac{Precisi\acute{o}n\cdot Exhaustividad}{Precisi\acute{o}n+Exhaustividad}
\]

\begin{center}
\emph{(valor ponderado)}
\par\end{center}

\end{overprint}
\end{columns}

\end{frame}
%
\begin{frame}{Verificaci�n del Clasificador ($\mathrm{A}$)}

\framesubtitle{Resultados}

\begin{table}[h]
 \caption{Aciertos y desaciertos en actividades humanas detectadas}
\centering\resizebox{11cm}{!}{ %
\begin{tabular}{|l|c|c|c|c|c|c|}
\cline{2-5} 
\multicolumn{1}{l|}{} & \multicolumn{4}{c|}{\textbf{Desaciertos}} & \multicolumn{1}{c}{} & \multicolumn{1}{c}{}\tabularnewline
\cline{2-7} 
\multicolumn{1}{l|}{} & \texttt{\footnotesize{}WALKING}{\footnotesize{} } & \texttt{\footnotesize{}STILL}{\footnotesize{} } & \texttt{\footnotesize{}ON\_BICYLE}{\footnotesize{} } & \texttt{\footnotesize{}IN\_VEHICLE} & \textbf{\small{}Aciertos} & \textbf{\small{}Tasa de Aciertos}\tabularnewline
\hline 
\hline 
\texttt{\footnotesize{}WALKING}{\footnotesize{} } &  &  & 13  &  & 151  & 92,07\%\tabularnewline
\hline 
\texttt{\footnotesize{}RUNNING} &  &  & 13  &  & 83  & 86,46\%\tabularnewline
\hline 
\texttt{\footnotesize{}STILL} &  &  &  & 8  & 140  & 94,59\%\tabularnewline
\hline 
\texttt{\footnotesize{}ON\_BICYLE}{\footnotesize{} } & 8  &  &  & 4  & 59  & 83,10\%\tabularnewline
\hline 
\texttt{\footnotesize{}IN\_VEHICLE} &  & 11  &  &  & 83  & 88,30\%\tabularnewline
\hline 
\end{tabular}} 
\end{table}

\begin{block}{Nota}

Resultado de la encuesta con dos individuos.
\end{block}
\end{frame}
%
\begin{frame}{Verificaci�n del Clasificador ($\mathrm{A}$)}

\framesubtitle{Resultados}

\begin{table}[h]
 \caption{\emph{HARDroid} vs \emph{Sony Lifelog}}
\centering\resizebox{11cm}{!}{ %
\begin{tabular}{|c|c|c|c|c|c|}
\hline 
Etiqueta & Rango  & Duraci�n  & \textbf{\emph{HARDroid}}  & \textbf{\emph{Sony Lifelog}}  & \textbf{Tasa de Aciertos}\tabularnewline
\hline 
\hline 
\texttt{WALKING}  & 21:26 - 21:44  & 19  & 45 & 54  & 83\%\tabularnewline
\hline 
\texttt{ON\_BICYCLE}  & 21:44 - 22:00  & 27  & 32 & 40  & 80\%\tabularnewline
\hline 
\texttt{RUNNING}  & 22:12 - 22:24  & 13  & 41 & 41  & 100\%\tabularnewline
\hline 
\texttt{IN\_VEHICLE}  & 22:39 - 22:57  & 19  & 32 & 38  & 84\%\tabularnewline
\hline 
\texttt{STILL}  & 22:25 - 22:35  & 10  & 65 & (65)  & 100\% \tabularnewline
\hline 
\end{tabular}} 
\end{table}

\begin{block}{Nota}

Resultados de la encuesta con un individuo. Se asume que \emph{Sony
Lifelog} es 100\% efectivo para la comparaci�n.
\end{block}
\end{frame}
%
\begin{frame}{Clasificador Colaborativo ($\mathrm{B}$)}

\framesubtitle{Resultados }

El clasificador colaborativo ($\mathrm{B}$) es un �rbol de decisi�n
de \textbf{699} nodos con \textbf{350} hojas.

\begin{table}[H]
\caption{Comparaci�n ($\mathrm{A}$) vs ($\mathrm{B}$)}
\centering\resizebox{10cm}{!}{ %
\begin{tabular}{|c|c|c|c|c}
\cline{1-4} 
M�trica  & Inicial ($\mathrm{A})$  & Colaborativo ($\mathrm{B})$  & Diferencia  & \tabularnewline
\hline 
\hline 
$N$  & 12.012  & 12.578  & \textcolor{green}{+566}  & $\checkmark$ \tabularnewline
\cline{1-4} 
$C$  & 10.942 (91,09\%)  & 11.489 (91,34\%)  & \textcolor{green}{+0,25\%}  & $\checkmark$ \tabularnewline
\cline{1-4} 
$I$  & 1.070 (8,91\%)  & 1.089 (8,66\%)  & \textcolor{red}{-0,25\%}  & $\checkmark$ \tabularnewline
\cline{1-4} 
$\kappa$  & 0,8678  & 0,8735  & \textcolor{green}{+0,57\%}  & $\checkmark$ \tabularnewline
\cline{1-4} 
$E_{ma}$  & 0,0341  & 0,0333  & \textcolor{red}{-0,08\%}  & $\checkmark$ \tabularnewline
\cline{1-4} 
$E_{rms}$  & 0,164  & 0,1632  & \textcolor{red}{-0,08\%}  & $\checkmark$ \tabularnewline
\cline{1-4} 
$E_{ra}$  & 15,17\%  & 14,58\%  & \textcolor{red}{-0,59\%}  & $\checkmark$ \tabularnewline
\cline{1-4} 
$E_{rrs}$  & 48,91\%  & 48,30\%  & \textcolor{red}{-0,61\%}  & $\checkmark$ \tabularnewline
\cline{1-4} 
\end{tabular}} 
\end{table}

\end{frame}
%
\begin{frame}{Clasificador Colaborativo ($\mathrm{B}$)}

\framesubtitle{Resultados }

Las m�tricas de evaluaci�n del clasificador colaborativo resultan
en: 
\begin{itemize}
\item Exactitud: \textbf{0,9723} (> 0,9714) 
\item Precisi�n: \textbf{0,9204} (> 0,9174) 
\item Exhaustividad: \textbf{0,9134} (> 0,9109) 
\item \emph{Valor-F}: \textbf{0,9168} (> 0,9142) 
\end{itemize}
\end{frame}

