
\section{Conclusi�n}
\begin{frame}{Conclusi�n}

\framesubtitle{Logros}

\end{frame}
%
\begin{frame}{Conclusi�n}

\framesubtitle{Trabajos Futuros}
\begin{itemize}
\item Segmentar los grupos de individuos distintos por rangos de edad, sexo
y/o factores fisiol�gicos y otros, de manera a generar modelo espec�ficos
para cada grupo.
\item Incorporar al sistema \emph{HARDroid} desarrollado otros m�todos de
aprendizaje autom�tico que permitan mejorar la tasa de aciertos. (SVN,
ANN).
\item Incluir m�s variables en el procesamiento de se�ales de entrada para
mejorar las predicciones teniendo en cuenta la orientaci�n del dispositivo. 
\end{itemize}
\end{frame}
%
\begin{frame}{Conclusi�n}

\framesubtitle{Trabajos Futuros}
\begin{itemize}
\item Extender el sistema \emph{HARDroid} para identificar otras actividades
humanas aplicado a otras disciplinas o contextos. 
\item Utilizar o aprovechar otros accesorios que agreguen variables relevantes
en el reconocimiento de actividades dependiendo del contexto (\emph{Smartwatch}). 
\end{itemize}

\end{frame}
%
\begin{frame}[t]{Congresos y presentaciones}

%\framesubtitle{Congresos y presentaciones}
\begin{itemize}
	\item ACM CHI 2018 - Conference on Human Factors in Computing Systems - Montr�al, Canada - \emph{(Pendiente)}
	\item ICDE 2018 - 34th IEEE International Conference on Data Engineering - Paris, Francia - \emph{(Pendiente)}
\end{itemize}

\end{frame}
%
\begin{frame}
\begin{center}
Gracias!
\par\end{center}
\end{frame}

